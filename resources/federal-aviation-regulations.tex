\chapter{Federal Aviation Regulations}

The Federal Aviation Regulations, or FARs, provide the regulatory framework for aircraft operations within the United States. Outside the United States, the International Civil Aviation Organization, or ICAO, is the United Nations agency that helps to provide a regulatory framework. For now, let's focus on operations inside the United States.

\section{Part 21 - Certification}

\subsection{Airworthiness Certificates}

Special Flight Permits are covered in \href{https://www.ecfr.gov/current/title-14/section-21.197}{14 CFR 21.197}. There are two reasons why we might want one:

\begin{enumerate}
\item Moving a non-airworthy aircraft. We might do this in order to make repairs, to store it, to deliver or export it, to perform production flight testing, evacuation, or customer demonstration; and
\item Ferry flights, for which we may need additional fuel tanks and weight for added range.
\end{enumerate}

The local FSDO can provide such a permit.

\section{Part 23 - Airworthiness for Normal (non Transport) Airplanes}

Airworthiness is the measure of an aircraft's suitability for safe flight. FAR parts 21 through 39 concern themselves with airworthiness. In this section, we'll touch upon a few common areas.

\subsection{Aircraft Categories}

There was a big shake up of FAR 23 in 2016. ``With respect to the number of categories on a TC, the FAA is eliminating the commuter, utility, and acrobatic airplane categories in part 23 for the reasons explained in the NPRM. Therefore, airplanes certified under new part 23 have only one category: normal." See \url{https://www.federalregister.gov/documents/2016/12/30/2016-30246/revision-of-airworthiness-standards-for-normal-utility-acrobatic-and-commuter-category-airplanes} for more information.

Therefore, to really understand what these are, we need to look at the state of FAR 23 in early 2017, before the sweeping changes. \href{https://www.ecfr.gov/on/2017-01-03/title-14/chapter-I/subchapter-C/part-23}{14 CFR 23 2017-01-03} provides a good historical reference as of this writing.

\begin{center}
\begin{tabular}{ |c|c|c| }
\hline
Category & Positive G Limit & Negative G Limit \\
\hline
Normal    & 3.8 & 1.52 \\
Commuter  & 3.8 & 1.52 \\
Utility   & 4.4 & 1.76 \\
Acrobatic & 6.0 & 3.00 \\
\hline
\end{tabular}
\end{center}

\subsection{Category, Class, and Type}

My mnemonic for remembering these is: sort alphabetically and you naturally get the longest word to the shortest word. This is good because we're going from less specific to more specific, and, broadest inclusion to narrowest inclusion. These definitions come straight from \href{https://www.ecfr.gov/current/title-14/chapter-I/subchapter-A/part-1/section-1.1}{14 CFR 1.1}. Focusing on what this means for airman certification, we have:

Category - (1) As used with respect to the certification, ratings, privileges, and limitations of airmen, means a broad classification of aircraft. Examples include: airplane; rotorcraft; glider; and lighter-than-air.

Class - (1) As used with respect to the certification, ratings, privileges, and limitations of airmen, means a classification of aircraft within a category having similar operating characteristics. Examples include: single engine; multiengine; land; water; gyroplane; helicopter; airship; and free balloon; 

Type - (1) As used with respect to the certification, ratings, privileges, and limitations of airmen, means a specific make and basic model of aircraft, including modifications thereto that do not change its handling or flight characteristics. Examples include: DC–7, 1049, and F–27. 

\section{Part 39 - Airworthiness Directives}

\href{https://www.ecfr.gov/current/title-14/chapter-I/subchapter-C/part-39}{14 CFR 39} discusses the topic of Airworthiness Directives or ADs. ADs are regulatory and compliance is mandatory. The aircraft owner or operator is responsible for compliance.

§ 39.3 Definition of airworthiness directives. FAA's airworthiness directives are legally enforceable rules that apply to the following products: aircraft, aircraft engines, propellers, and appliances.

§ 39.5 When does FAA issue airworthiness directives? FAA issues an airworthiness directive addressing a product when we find that: (a) An unsafe condition exists in the product; and (b) The condition is likely to exist or develop in other products of the same type design.

\section{Part 43 - Maintenance, Preventive Maintenance, Rebuilding, and Alteration}

\href{https://www.ecfr.gov/current/title-14/chapter-I/subchapter-C/part-43}{14 CFR 43} discusses maintance that we are able to perform on the aircraft.

14 CFR Appendix-A-to-Part-43(c) provides an extensive listing of precisely what \emph{preventive maintenance} entails.

A pilot is one of the individuals capable of performing the work (43.3(g)) and must make a log book entry consistent with 43.9.

\section{Part 47 - Aircraft Registration}

\href{https://www.ecfr.gov/on/2017-01-03/title-14/chapter-I/subchapter-C/part-47}{14 CFR 47} deals with the various and sundry details of aircraft registration.

By default, aircraft registrations are valid for three years, as of 2010-07-20. 14 CFR 47.40(b).

But they can be nullified as per the conditions in 14 CFR 47.41. See acronym 30 FT DUC in the Alphabet Soup section later on in this book.

\section{Part 61 - Certification: Pilots, Flight Instructors, and Ground Instructors}

\subsection{Medical Certificates}

\href{https://www.ecfr.gov/current/title-14/chapter-I/subchapter-D/part-61/subpart-A/section-61.23}{14 CFR 61.23} discusses the topic of medical certificates.

Explicitly called out are the kinds of medical certificates required for certain operations:

\begin{itemize}
\item First Class Medical - PIC of airline transport certificate, plus some additional Part 121 scenarios;
\item Second Class Medical - commercial pilot certificate (other than glider, special rules for balloons), some additional Part 121 scenarios;
\item Third Class Medical - private, recreational, or student pilot; flight instructor; checkride; examiner, with exceptions for BasicMed.
\end{itemize}

A higher class medical certificate will roll over to the lower classes as time passes. ``Months'' refers to months after the month of the date of the examination. So a June 1st issuance expires on July 31st.

\begin{center}
\begin{tabular}{ |c|c|c|c|c| }
\hline
Class & $>$ 40 & 1st Class & 2nd Class & 3rd Class \\
\hline
First & Yes      & 12 mo & 12 mo & 60 mo \\
First & No       &  6 mo & 12 mo & 24 mo \\
\hline
Second & Yes     & --        & 12 mo & 60 mo \\
Second & No      & --        & 12 mo & 24 mo \\
\hline
Third & Yes      & --        & --        & 60 mo \\
Third & No       & --        & --        & 24 mo \\
\hline
\end{tabular}
\end{center}



\subsection{Currency}

\href{https://www.ecfr.gov/current/title-14/chapter-I/subchapter-D/part-61/subpart-A/section-61.57}{14 CFR 61.57} discusses the currency requirements for private and commercial pilots, and also discusses instrument currency requirements. For VFR flight taking passengers, the requirements are as follows.

\begin{itemize}
\item Three takeoffs and landings in the past 90 days, as sole manipulator, in the same category, class, and type of aircraft.
\item For tailwheel aircraft, those three landings must be to a full stop.
\item For night time operations (an hour past sunset to an hour before sunrise), those three landings must be to a full stop.

\end{itemize}

\section{Part 91 - General Operating and Flight Rules}

\subsection{Inoperative Equipment}

\href{https://www.ecfr.gov/current/title-14/chapter-I/subchapter-F/part-91/subpart-C/section-91.213}{14 CFR 91.213} discusses inoperative equipment and the Minimum Equipment List or MEL. The Piper Lance II is old enough that we don't have an MEL. The local FSDO will have a copy of the MEL.

It used to be that Advisory Circular 91-67, dated 1991-06-28, gave additional guidance on following an MEL but it was cancelled on 2017-11-03 for not being aligned with current ICAO standards. There is not a more current version of that AC. So it is still a helpful reference for domestic US operations.

\subsection{Inspections}

\href{https://www.ecfr.gov/current/title-14/chapter-I/subchapter-F/part-91/subpart-E/section-91.409}{14 CFR 91.409} discusses inspection requirements for our aircraft.

We must have an annual inspection every 12 calendar months.

We also need 100-hour inspections. ``...no person may operate an aircraft carrying any person (other than a crewmember) for hire, and no person may give flight instruction for hire in an aircraft which that person provides, unless within the preceding 100 hours of time in service the aircraft has received an annual or 100-hour inspection...''








