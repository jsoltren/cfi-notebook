\chapter{Federal Aviation Regulations}

The Federal Aviation Regulations, or FARs, provide the regulatory framework for aircraft operations within the United States. Outside the United States, the International Civil Aviation Organization, or ICAO, is the United Nations agency that helps to provide a regulatory framework. For now, let's focus on operations inside the United States.

\section{Airworthiness}

Airworthiness is the measure of an aircraft's suitability for safe flight. FAR parts 21 through 39 concern themselves with airworthiness. In this section, we'll touch upon a few common areas.

\subsection{Aircraft Categories}

There was a big shake up of FAR 23 in 2016. ``With respect to the number of categories on a TC, the FAA is eliminating the commuter, utility, and acrobatic airplane categories in part 23 for the reasons explained in the NPRM. Therefore, airplanes certified under new part 23 have only one category: normal." See \url{https://www.federalregister.gov/documents/2016/12/30/2016-30246/revision-of-airworthiness-standards-for-normal-utility-acrobatic-and-commuter-category-airplanes} for more information.

Therefore, to really understand what these are, we need to look at the state of FAR 23 in early 2017, before the sweeping changes. \href{https://www.ecfr.gov/on/2017-01-03/title-14/chapter-I/subchapter-C/part-23}{14 CFR 23 2017-01-03} provides a good historical reference as of this writing.

\begin{center}
\begin{tabular}{ |c|c|c| }
\hline
Category & Positive G Limit & Negative G Limit \\
\hline
Normal    & 3.8 & 1.52 \\
Commuter  & 3.8 & 1.52 \\
Utility   & 4.4 & 1.76 \\
Acrobatic & 6.0 & 3.00 \\
\hline
\end{tabular}
\end{center}

\subsection{Category, Class, and Type}

My mnemonic for remembering these is: sort alphabetically and you naturally get the longest word to the shortest word. This is good because we're going from less specific to more specific, and, broadest inclusion to narrowest inclusion. These definitions come straight from \href{https://www.ecfr.gov/current/title-14/chapter-I/subchapter-A/part-1/section-1.1}{14 CFR 1.1}. Focusing on what this means for airman certification, we have:

Category - (1) As used with respect to the certification, ratings, privileges, and limitations of airmen, means a broad classification of aircraft. Examples include: airplane; rotorcraft; glider; and lighter-than-air.

Class - (1) As used with respect to the certification, ratings, privileges, and limitations of airmen, means a classification of aircraft within a category having similar operating characteristics. Examples include: single engine; multiengine; land; water; gyroplane; helicopter; airship; and free balloon; 

Type - (1) As used with respect to the certification, ratings, privileges, and limitations of airmen, means a specific make and basic model of aircraft, including modifications thereto that do not change its handling or flight characteristics. Examples include: DC–7, 1049, and F–27. 








