\chapter{Emergency Procedures}

Dealing with emergencies is a crucial part of operating any aircraft. As a commercial pilot, operations that push the comfort zones and limits of both the pilot and the aircraft will become that much more common. Further, abiding by the law of large numbers, accidents are more likely to happen with more hours in the aircraft. Thus it is critical to be familiar with general emergency procedures, as well as the specific procedures for the airplane. Like the rest of this book, this chapter focuses on the Piper Lance II, P32T.

Systems knowledge, and systems familiarity, play a large role in emergency procedures. Not every emergency will have a dedicated checklist or flow. Sometimes, the pilot may be required to think, slow down, and understand the situation.

The heart of all emergency procedures comes down to this list, provided by Mr. Roger Sharp:

\begin{enumerate}
\item What can go wrong?
\item How would I know?
\item How do I fix it?
\item If I can't fix it, how do I minimize impact?
\end{enumerate}

The first question involves some knowledge of aircraft systems. Thus, unlike the procedures in the POH, this chapter will focus more heavily on systems knowledge as it relates to emergency procedures.

The second question involves knowledge of general aerodynamics as well as the systems and quirks of the particular aircraft.

The third question involves following book procedure.

The fourth question involves aeronautical decision making, or ADM.

\section{Landing Gear}

\subsection{System Overview}

For starters, we begin reading about the landing gear in Chapter 7 of the POH. ``The Lance II is equipped with a retractable tricycle landing gear, which is hydraulically actuated by an electrically powered reversible pump. The pump is controlled by a selector switch on the instrument panel.''

The Landing Gear Electrical Schematic (Figure 7-5) provided an overview of the electrical systems connected to the gear. The landing gear actuator circuit breaker is a 25 amp circuit breaker which regulated the hydraulic pump motor, which is connected via relays to the actuator handle. The Gear Unsafe lamp is wired to the gear actuator and the up limit switches on each gear. The left, nose, and right bulbs are wired directly to the down lock switches.

Though not called out as such in the POH, the ``three greens'' pull right out of the instrument panel in case we need to swap them during emergency operations.

We do have a checklist for the landing gear in Section 3, Emergency Procedures. However, the title of the checklist, ``Emergency Landing Gear Extension'', is something of a misnomer. I would rename it to ``Landing Gear Malfunctions''.

Our airplane, thankfully, has the landing gear auto-extension system deleted. It is safe to ignore any sections of the POH that mention it. I confirmed this explicitly with Piper.

\subsection{Landing Gear Malfunctions - Checklist}

Prior to emergency extension procedure:

\begin{itemize}
\item Master Switch: Check ON.
\item Circuit Breakers: Check.
\item Radio Lights: Off (in daytime).
\item Gear Indicator Bulbs: Check.
\end{itemize}

If landing gear does not check down and locked:
\begin{itemize}
\item Airspeed: Below 87 KIAS.
\item Landing Gear Selector: Down.
\item Emergency Gear Lever: Override Engaged (while fishtailing airplane).
\end{itemize}

If landing gear still does not check down and locked:
\begin{itemize}
\item Emergency Gear Lever: Override Engaged (while fishtailing airplane).
\end{itemize}

If all electrical power has been lost, the landing gear must be extended using the above procedures. The gear position indicator lights will not illuminate.

\subsection{Landing Gear Malfunctions Discussion}

The POH, Section 3.27, provides a short discussion of the above checklist I find this discussion insufficient. So, I will expand upon it here.

The first part of the checklist involves dealing with electrical issues. After all, the hydraulic gear system has an electrical pump that actuates it. Maybe the master switch is off. Maybe the pump burned out.

On this airplane, the radio dimmers also dim the three greens of the landing lights. The suggestion to turn the radio lights off makes certain that the bulbs will be as bright as possible.

The POH says to ``check the landing gear indicators for faulty bulbs''. But it doesn't dare describe \emph{how} to do this. Fro crying out loud. They pull right out of the panel, and can be swapped. Note, from the electrical diagram, that the lights do offer some modicum of redundancy.

Has the airplane been flying in icing conditions? It's possible that the landing gear is frozen into place. Check the outside air temperature gage and fly into an area that has temperatures warm enough to melt the ice if able.

The recommendation to slow down - interestingly, to $V_Y$ in the dirty configuration (it would be helpful if they mentioned that in the POH) - is to pull aerodynamic drag off a weak mechanism.

The recommendation to fishtail the airplane is to help swing a faulty landing gear into place to lock it. But this will only work with the main gears. With the nose gear, some nose dives and recoveries may help to snap the gear into place.

If none of these procedures work, it's helpful to get some eyes on the ground to have a look. Phone a friend, call the tower, or go somewhere else. If it's night time - go to a big airport, maybe they have search lights. A big airport will have more runways and more emergency services. You did bring a 45 minute reserve, did you not?

At this point, it's time to consider a forced landing. Gear up or gear down? That is the question. With two wheels down out of three we can attempt a landing. Try to use aileron or elevator to keep pressure off of the ``bad'' tire for as long as possible, and expect a severe yawing or pitching moment when it finally does catch. Try to turn the engine off to minimize damage.

The POH does not give any specific guidance for a partial gear landing. It DOES give guidance for whether to attempt a power off landing with gear down or gear up. But it doesn't tell us to look there, now does it? The guidance in both cases is the same: lowest possible airspeed with full flaps, ignition off, master switch off, fuel selector off, mixture idle cutoff. Tighten seat belts and shoulder harnesses. I would also open the door and prop it with a jacket, and be ready to evacuate immediately. If there is time, I would prep the fire extinguisher.








