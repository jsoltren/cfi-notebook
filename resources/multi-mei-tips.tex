
\chapter{Multiengine Instructor Tips}

\emph{By an MEI, for MEIs.}

This chapter contains some items of particular interest to multiengine instructor (MEI) candidates.
The material in this chapter is of general interest to all multiengine pilots.

\section{Instructional Safety}

The job of the multiengine instructor is to maintain a safe learning environment for the learner.

Whenever manipulating certain controls - the flaps, any control under simulated or actual engine inoperative conditions -
it is important to make sure the correct control is being identified. The learner reaches for the control in question and
says, ``identify''. The instructor visually confirms: ``verify''. The learner then exercises the flap control.

Feathering the wrong engine can turn a training scenario into a real emergency very quickly. Raising the gear
instead of the flaps, is similarly disastrous.

When simulating an engine failure on the ground, the instructor should use brake or rudder, preferably brake,
to simulate a failed engine condition.

In lieu of feathering a propeller, the instructor may set the ``dead'' engine to about 12'' of manifold pressure
and 2000 RPM to simulate a zero thrust condition.

Consider that the \vmc roll has claimed some very high time pilots. The only thing that will save your life in that
situation is to reduce power on the good engine, and possibly extending the landing gear to further reduce \vmc at the
expense of further reduced performance. With a more powerful twin like the Baron, I'm going to be either reducing power
on the good engine during \vmc demonstration maneuvers (which reduces actual \vmc), or blocking the rudder (which
serves to raise effective \vmc, giving us a healthy margin of 5-10 knots between loss of directional control and actual \vmc).

See \url{http://www.kathrynsreport.com/2022/09/fatal-accident-occurred-september-04.html}.

\section{Teaching Single Engine Aerodynamics}

We can use the SMACFUM acronym, the COMBATS acronym, or other memory aids.

The learner needs to understand that certain items are for \emph{standardization}, meaning
certification purposes. Other items are things that the pilot-in-command can control, either prior
to the flight (who sits where?) or during the flight (do we use full power on the good engine?).

Things that may save a life: put passengers up front, put the gear down to lower \vmc, lower power, increase speed, lower the nose to accelerate up past \vmc. That's why these items are in the drag demo.

A physical model of the aircraft makes it easier to teach and visualize the pitch and roll moments
associated with a dead engine.

The lesson concerning zero side slip should incorporate a yaw string taped to the windscreen to demonstrate
the effectiveness of correct yaw and roll corrections when operating on a single engine.

Some examiners place a special emphasis on the PAST acronym, and specifically, torque. My trick for teaching torque
is to use a battery powered hand drill! By squeezing the trigger on the drill, we can very easily visualize both the
roll and yaw moments associated with torque. Roll is the obvious one, but why yaw? It's because we're holding
the drill, which constrains it, and precession causes it to yaw.

Work to find teaching opportunities and teachable moments.

It can be tricky to find the ``old'' 14 CFR 23 that was the certification basis for the Duchess. ecfr.gov can provide
a historical view of 14 CFR 23 as of January 1, 2017 that contains 23.149 and other relevant regulations.

When possible, we want to avoid turning across the dead engine, since we may not be able to
stop the turn. Generally, a standard rate turn across the good engine, and a half standard rate turn across
the bad engine, is safe. This varies per airplane, confirm in your POH.

% Remove the above since we can do it in the Duchess.
% Put it back in because Beth cautioned about turns across the dead engine.

\section{Teaching the One Engine Inoperative (OEI) Traffic Pattern}

There are three ways to simulate the failure of an engine in a piston twin. They are:

\begin{enumerate}
    \item Pull the power.
    \item Pull the mixture.
    \item Fuel cutoff.
\end{enumerate}

Of these, only one - pulling the power - is safe close to the ground. Why? It's the easiest to undo when we need
two working engines again.

If planning to teach an OEI traffic pattern, brief it on the ground. No specific briefing is necessary. The generic
pre takeoff briefing already covers the situation. Make sure to bug runway heading for takeoff, this will be important later.

The ACS spells out that the engine must not be failed below 400 AGL. I would wait until at least 500 AGL. If at all
possible, pull the engine on upwind, and certainly NOT during a turn.

Pull a throttle. Keep a finger on it so you remember what you did. Call out ``simulated engine failre''. Demonstrate,
or have the student perform, the flow. If having the student perform, it's critical to confirm that each step is done
promptly, and to continue to cross check: altitude, airspeed, and heading where they need to be, controls staying where they need
to say. Students and examiners love to omit steps, or to undo correct things that they did.

\begin{itemize}
    \item Airspeed: pitch for \vyse. In the Duchess, this means immediately lowering the nose by 5 degrees.
    \item Directional control: maintain. You did bug runway heading, didn't you? Good. Step on that heading bug! Bank 3-5 degrees into the good engine.
    \item All available power! Mixtures RICH, props FORWARD, throttles FULL (but you're blocking one so that it doesn't go forward accidentally).
    \item Gear: identify, verify, up. Confirm no greens, no reds. Flaps: identify, verify, up.
    \item Identify: dead foot, dead engine. Verify: touch the dead throttle. The instructor is still blocking it, so it won't move.
    \item Feather. Simulate feather close to the ground: blue lever to the detent. Consider blocking the prop lever for the good engine.
\end{itemize}

We do not attempt to fix the engine close to the ground. We just feather it. For training, we simulate feather with: prop to detent, power
to 11 inches of manifold pressure.

Assess the situation. Are we maintaining altitude or climbing? Great! Reduce power on the good engine if possible to lower \vmc,
and circle back to land. Otherwise, accept the drift down, and descend to the ground, hopefully in a controlled fashion.

Turns across the good engine on full power may be done at up to standard rate. Turns across the dead engine with the good
engine at full power should not exceed half standard rate if at all possible. If we reduce power on the good engine, we can
perhaps tolerate a turn up to standard rate.

Watch that altitude! If you are below 500 AGL and the airplane is not under positive control, the maneuver is over. Put the landing
gear down, bring the ``failed'' engine back online, reduce power on the ``good'' engine, and try again.

With the airplane under positive control, fly a normal to somewhat wider traffic pattern. Manage power carefully to maintain blue
line. Drop the landing gear abeam the numbers if possible. Some particularly weak light twins may not be able to descend gracefully
with the gear down. The Duchess can, and benefits from the \vmc reduction that comes with dropping the gear. Keep the flaps up until
the runway is made. Confirm the landing gear is down (three greens, no red) on downwind, base, and final.

The landing standard is: first third of the runway.

DO NOT attempt a go around with flaps fully extended in the Duchess. If at all possible, don't attempt one at all.

\section{Teaching the In-Flight Engine Feather and Air Restart}

In this maneuver, we are actually feathering an engine. To ensure safety, it is important to brief this maneuver. We risk the
single greater killer of light piston twin pilots: the feathering of the wrong engine.

The flight instructor's challenge is not simply to fly the maneuver. It is to be about ten minutes
ahead of the maneuver, anticipating common problem areas, and immediately terminating the
maneuver if safety cannot be maintained.

\subsection{Maneuver Setup}

The Private and Commercial ACS specify that this maneuver shall be completed at least 3000' AGL.

I would recommend a higher altitude. Consider the following:

We want to plan for up to 1000 feet of altitude loss if altitude cannot be maintained, plus potentially 500 feet for recovery. This might suggest a 4500' AGL entry.

The service ceiling for a light twin is usually somewhere around 6000 to 8000 feet MSL. Doing the maneuver above the service ceiling will
ensure that altitude may not be maintained. If the pilot attempts to pitch to maintain altitude, we are effectively performing a \vmc demo,
in which case we will either stall or enter a \vmc roll.

However, the higher we go, the less power the working engine has to ``bully'' us into an unusual attitude.

Multi engine time can be very expensive. Students may become annoyed spending on the order of ten dollars a minute, climbing to an excessive altitude for the maneuver.

The maneuver entry should be performed at a VFR altitude.

Putting all this together, assuming the ground is at 1000 MSL, an entry altitude of 5500 MSL seems to work quite well. Set up the maneuver in the typical fashion: cruise power configuration, best landing site selected, clearing turns completed.

Weather may make this maneuver challenging. In multi engine training, it is somewhat common to file an IFR flight plan, get above a cloud layer, and perform VFR on top maneuvering. Consider that recovery from this maneuver may be difficult in IMC, and that the student may not be instrument rated. A risk assessment is ultimately left to the instructor.

\subsection{Failing an Engine}

Recall our options for failing an engine:

\begin{enumerate}
    \item Pull the power.
    \item Pull the mixture.
    \item Fuel cutoff.
\end{enumerate}

We have enough altitude to actually fail an engine, and we intend to actually halt and secure the engine, so we can consider the second and third items on this list. Some considerations:

Pulling the mixture is pretty obvious. It gives a clear indication of the failed engine.

Pulling the fuel cutoff is more subtle. It makes the student figure out which engine has failed.

Generally, I prefer to pull the mixture on a checkride, and pull the cutoff in actual training.

Whichever engine fails: make sure YOU, the instructor, remembers which one it is! You might consider pre-emptively guarding the correct rudder in anticipation of a sudden loss of directional control. This is less of an issue in our docile, counter-rotating light piston twins, but it's still good practice for the instructor.

With fuel not flowing to the engine, it will stutter in about ten seconds. That's plenty of time to get ahead of the immediate action items and set up to guard the relevant controls.

\subsection{Immediate Action Items}

The action items correspond to our usual flow again. We will reiterate the flow here once again, with amplifications as relevant for this procedure.

\begin{itemize}
    \item Airspeed: pitch for \vyse. In the Duchess, this means immediately lowering the nose by 5 degrees.
    \item Directional control: maintain. You did bug the prior heading, didn't you? Good. Step on that heading bug! Bank 3-5 degrees into the good engine. A competent instructor is already guarding this rudder.
    \item All available power! Mixtures RICH (except for the ``failed'' mixture, which is guarded by a finger on the throttle track), props FORWARD, throttles FULL.
    \item Gear: identify, verify, up. Confirm no greens, no reds. Flaps: identify, verify, up.
    \item Identify: dead foot, dead engine. The instructor might ask, ``what is your left/right foot doing right now?'' Verify: move the suspected dead throttle. The student touches it and says ``left/right throttle identify''. The instructor says ``verify''. The student then moves the suspect throttle to idle.
    \item Fix or Feather? Attempt to fix if we are above 3000' AGL, which we are, pursuant to the earlier discussion.
\end{itemize}

Fixing needs to be a flow, not a checklist, as these are immediate emergency action items. In the BE-76 Duchess, we start at the mixture control, and go down, focusing on the controls for the dead engine. The instructor's job is to make sure that the correct controls are cycled or manipulated, lest we induce a second failure.

\begin{enumerate}
    \item Mixture: full rich. Try to restore fuel flow.
    \item Carb Heat: on. Try to remove any carb ice that may have formed.
    \item Cowl Flap: close. Try to prevent shock cooling of this sick engine. Open the ``good'' engine cowl flap all the way as it is now working as hard as it can.
    \item Fuel Selector: on. This may need to be simulated, if the instructor pulled the mixture. At least touch the control.
\end{enumerate}

Then, we start at the mixture control again, and flow right to left:

\begin{enumerate}
    \item Mixture: confirm full rich. Try to restore fuel flow.
    \item Auxiliary Fuel Pump: on. Perhaps an engine driven fuel pump has failed?
    \item Magnetos: cycle right-left-on. Perhaps one bank of magnetos is failing?
    \item Alternator: cycle off-on. Perhaps an alternator has seized?
\end{enumerate}

No joy. Proceed to feather.

This is a great time for the instructor to double check airspeed, altitude, and heading. Are we getting close to that 3000' AGL limit? If so,
we need to consider the maneuver incomplete, recover, and try again. Whether it was airspeed mismanagement, poor configuration, or
insufficient engine power on the good engine, if we're at 3000' AGL now, we've already lost a bunch of altitude and will lose a
bunch more. There is no emergency that cannot be made worse by rushing.

\subsection{Feathering and Securing of the Engine}

Finally,

\begin{itemize}
    \item Feather. Actually feather the engine at sufficient altitude: blue lever to the detent. Consider blocking the prop lever for the good engine.
\end{itemize}

The student identifies the propeller lever of the ``dead'' engine. The instructor verifies it. The student pulls the engine to feather. All gets quiet on one side of the airplane.

At this point, run through the same exact flow once again: mixtures DOWN, then mixtures LEFT:

\begin{itemize}
    \item Mixture: identify, verify, OFF.
    \item Carb heat: identify, verify, OFF.
    \item Cowl flap: identify, verify, CLOSE, and OPEN on opposite engine. They do like to slip on the Duchess, so make sure.
    \item Fuel selector: identify, verify, OFF. Consider CROSSFEED for extended single engine operations, but leave to OFF in training.
\end{itemize}

Thence:

\begin{itemize}
    \item Mixture: confirm OFF.
    \item Propeller: confirm FEATHER.
    \item Throttle: confirm IDLE.
    \item Fuel boost pump: identify, verify, OFF.
    \item Magnetos: identify, verify, OFF.
    \item Alternator: identify, verify, OFF.
\end{itemize}

Follow up with the checklist. After the flow, this is a great time for the challenge-response method of checklist execution, with the
instructor performing the challenge. Or, the candidate/student can perform the challenges, and the instructor can perform responses.

We might be here for a little while, so verify any single engine conditions. In the Duchess, this means monitoring fuel (that working engine is very thirsty right now), monitoring electrical load, and watching the pressure system to make sure the gyroscopes are still spinning. And, of course, verify airspeed, altitude, and heading one more time.

It is good to at least mention that if we are planning to stay here for a while, we want to trim accordingly. This is a three dimensional maneuver, so we need pitch trim for airspeed, rudder trim for heading, and aileron trim to bank into the good engine, raise the dead engine, and establish a zero sideslip condition. Phew!

For a student, this is a very popular time to take a selfie.

\subsection{Restart of the Engine}

Remembering once again that feathering or failing the ``good'' engine is about the most dangerous thing we can do, it's important to take our time with the air restart of the engine.

Fortunately, the manufacturer's checklist is written as a ``do'' list. Follow the checklist to restart the engine. Use crew resource management and identify/verify each individual step.

A common gotcha in the Duchess is that the air restart needs to be performed at an airspeed of at least 100 KIAS. This means we need to pitch
DOWN, and we will LOSE airspeed in the process! But without enough airspeed, the propeller may not come out of feather.

Congrats! The canonical multi engine simulated emergency maneuver is complete.

\section{Who can get a multiengine rating?}

\subsection{Sport Pilot}

Can a sport pilot get a multiengine rating? No. 61.311 spells out the categories and classes
of aircraft available to the sport pilot. The list does not include multiengine airplanes.

\subsection{Recreational Pilot}

Can a recreational pilot get a multiengine rating? No. 61.101 specifically prohibits
a recreational pilot from operating an aircraft with more than one powerplant.

\subsection{Private Pilot}

Can a private pilot get a multiengine rating as their initial rating?
Yes! 61.109(b) spells out the aeronautical requirements
for the private pilot airplane multiengine rating. A student pilot could choose to start in multi
engine airplanes. It may be difficult for the student to get insurance coverage for the 10 hours
of solo time required for this rating. Legally possible, practically impossible due to insurance.

If we can make it work, they will need these endorsements at a minimum, referencing AC 61-65J, current as of this writing. Take note of A.68, as most multis are complex, and A.69, as many are high performance as well.

\begin{itemize}
\item A.1 Prerequisites for practical test: § 61.39(a)(6)(i) and (ii).
\item A.2 Review of deficiencies identified on airman knowledge test: § 61.39(a)(6)(iii), as
required.
\item A.3 Pre-solo aeronautical knowledge: § 61.87(b).
\item A.4 Pre-solo flight training: § 61.87(c)(1) and (2).
\item A.6 Solo flight (first 90-calendar-day period): § 61.87(n).
\item A.9 Solo cross-country flight: § 61.93(c)(1) and (2).
\item A.10 Solo cross-country flight: § 61.93(c)(3).
\item A.14 Endorsement of U.S. citizenship recommended by the Transportation Security
Administration (TSA): Title 49 of the Code of Federal Regulations (49 CFR)
part 1552, § 1552.15(c).
\item A.32 Aeronautical knowledge test: §§ 61.35(a)(1), 61.103(d), and 61.105.
\item A.33 Flight proficiency/practical test: §§ 61.103(f), 61.107(b), and 61.109.
\item A.68 To act as PIC in a complex airplane: § 61.31(e).
\item A.69 To act as PIC in a high-performance airplane: § 61.31(f). (as needed)
\item A.82 Review of a home-study curriculum: § 61.35(a)(1). (unusual)
\end{itemize}

\subsection{Commercial Pilot}

Can a commercial pilot get a multiengine rating? Absolutely. This is the most common path.
61.129(b) lists the aeronautical experience requirements for the commercial multiengine rating.
Of note, 61.129(b)(4) has a specific allowance for completing solo flight time, not truly solo, but
with an authorized instructor on board.

What is the minimum training requirement for a pilot with a Commercial ASEL certificate
to add a multiengine rating? There is none! 61.63 lists the specific requirements for adding an
additional category or class to a pilot certificate.

What additional endorsements are needed for the C-AMEL add on?

\begin{itemize}
\item A.1 Prerequisites for practical test: § 61.39(a)(6)(i) and (ii).
\item A.68 To act as PIC in a complex airplane: § 61.31(e).
\item A.69 To act as PIC in a high-performance airplane: § 61.31(f). (as needed)
\item A.72 To act as PIC of an aircraft in solo operations when the pilot does not hold an
appropriate category/class rating: § 61.31(d)(2).
\item A.74 Additional aircraft category or class rating (other than ATP): § 61.63(b) or (c).
\end{itemize}

For the C-AMEL initial, we also need these endorsements:

\begin{itemize}
\item A.34 Aeronautical knowledge test: §§ 61.35(a)(1), 61.123(c), and 61.125.
\item A.35 Flight proficiency/practical test: §§ 61.123(e), 61.127, and 61.129.
\end{itemize}

\subsection{Airline Transport Pilot}

Can an airline transport pilot get a multiengine rating? Yes, absolutely. 61.156 lists the requirements
for the multiengine class or multiengine airplane type rating for the airline transport pilot certificate.
Of note, the only way to get this is to go through a training course approved by the Administrator. There is
no way around this: 61.165 says that to add the multiengine class to an airline transport pilot certificate,
the requirements of 61.156 must be met.

The vast majority of ATPs are multi engine ATPs.

\subsection{Additional Scenarios}

A client has a private pilot certificate with an ASEL category/class and a commercial pilot
certificate with a rotorcraft/helicopter category class. Can they get their multiengine airplane
rating? Yes. Two paths are available: private or commercial privileges. See 61.63
for further details.

Is an instrument rating required for the commercial multiengine pilot? No. The certificate will have
a VFR Only limitation in this situation.

What if the learner takes their checkride in a multi engine airplane with two engines in a straight line, such as the Cessna 337?
They will be ``limited to centerline thrust'' and
can have this endorsement removed with an abbreviated checkride in the future.

\section{MEI Add-On Checkride}

The Flight Instructor ACS spells out the requirements. In short, there will be ground lessons on
single engine aerodynamics, weight and balance, performance charts, and systems. The flying
portion will include the usual maneuvers plus the drag demo. The examiner will typically have the
candidate walk them through a number of maneuvers, and ask them to teach a single-engine
traffic pattern as well as an air shutdown and restart of an engine..

\section{Resources}

Airplane Flying Handbook, Chapter 13, discusses multiengine airplanes in detail.

Constant speed propellers and other systems are covered in greater detail in other chapters of the PHAK and AFM.

FAA-P-8740-66, Flying Light Twins Safely

FAASTeam Light Twin Takeoff Control \& Performance Briefing checklist.

FAASTeam ALC-30: Multi-Engine Safety Review.\\\url{https://www.faasafety.gov/files/helpcontent/Courses/ALC-30/content/index.html}


