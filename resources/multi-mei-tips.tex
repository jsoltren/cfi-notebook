
\chapter{Multi Engine Instructor Tips}

This chapter contains some items of particular interest to multi engine instructor (MEI) candidates.
The material in this chapter is of general interest to all multi engine pilots.

\section{Instructional Safety}

The job of the multi engine instructor is to maintain a safe learning environment for the learner.

Whenever manipulating certain controls - the flaps, any control under simulated or actual engine inoperative conditions -
it is important to make sure the correct control is being identified. The learner reaches for the control in question and
says, ``identify''. The instructor visually confirms: ``verify''. The learner then exercises control.

Feathering the wrong engine can turn a training scenario into a real emergency very quickly. Raising the flaps
instead of the gear, or vice versa, is similarly problematic.

When simulating an engine failure on the ground, the instructor should use brake or rudder, preferably brake,
to simulate a failed engine condition.

When simulating an engine failure in the air, particular in high density altitude conditions where \vmc may be
below the stalling speed, the instructor may block a rudder with their foot to limit rudder travel. This allows the
learner to reach ``full rudder deflection'' earlier than they would otherwise.

In lieu of feathering a propeller, the instructor may set the ``dead'' engine to about 12'' of manifold pressure
to simulate a zero thrust condition.

\section{Teaching Single Engine Aerodynamics}

We can use the SMACFUM acronym, the COMBATS acronym, or other memory aids.

The learner needs to understand that certain items are for \emph{standardization}, meaning
certification purposes. Other items are things that the pilot-in-command can control, either prior
to the flight (who sits where?) or during the flight (do we use full power on the good engine?).

Things that may save a life: put passengers up front, put the gear down to lower \vmc, lower power, increase speed, lower the nose to accelerate up past \vmc. That's why these items are in the drag demo.

A physical model of the aircraft makes it easier to teach and visualize the pitch and roll moments
associated with a dead engine.

The lesson concerning zero side slip should incorporate a yaw string taped to the windscreen to demonstrate
the effectiveness of correct yaw and roll corrections when operating on a single engine.

The instructor should remember that teaching a student or private pilot is different than teaching
a commercial pilot. For the student pilot, more demonstrations and abbreviated explanations may be
appropriate. When in doubt, keep it simple.

Work to find teaching opportunities and teachable moments.

It can be tricky to find the ``old'' 14 CFR 23 that was the certification basis for the Duchess. ecfr.gov can provide
a historical view of 14 CFR 23 as of January 1, 2017 that contains 23.149 and other relevant regulations.

When possible, we want to avoid turning across the dead engine, since we may not be able to
stop the turn. Generally, a standard rate turn across the good engine, and a half standard rate turn across
the bad engine, is safe. This varies per airplane, confirm in your POH.

\section{Who can get a multi engine rating?}

Can a sport pilot get a multi engine rating? No. 61.311 spells out the categories and classes
of aircraft available to the sport pilot. The list does not include multi engine airplanes.

Can a recreational pilot get a multi engine rating? No. 61.101 specifically prohibits
a recreational pilot from operating an aircraft with more than one powerplant.

Can a private pilot get a multi engine rating? Yes! 61.109(b) spells out the aeronautical requirements
for the private pilot airplane multi engine rating. A student pilot could choose to start in multi
engine airplanes. It may be difficult for the student to get insurance coverage for the 10 hours
of solo time required for this rating.

Can a commercial pilot get a multi engine rating? Absolutely. This is the most common path.
61.129(b) lists the aeronautical experience requirements for the commercial multi engine rating.
Of note, 61.129(b)(4) has a specific allowance for completing solo flight time, not truly solo, but
with an authorized instructor on board.

What is the minimum training requirement for a pilot with a Commercial ASEL certificate
to add a multi engine rating? There is none! 61.63 lists the specific requirements for adding an
additional category or class to a pilot certificate.

Can an airline transport pilot get a multi engine rating? Yes, absolutely. 61.156 lists the requirements
for the multiengine class or multiengine airplane type rating for the airline transport pilot certificate.
Of note, the only way to get this is to go through a training course approved by the Administrator. There is
no way around this: 61.165 says that to add the multiengine class to an airline transport pilot certificate,
the requirements of 61.156 must be met.

A client has a private pilot certificate with an ASEL category/class and a commercial pilot
certificate with a rotorcraft/helicopter category class. Can they get their multi engine airplane
rating? Yes. Two paths are available: private or commercial privileges. One requires a written test. See 61.63
for further details.

Is an instrument rating required for the commercial multi engine pilot? No. The certificate will have
a VFR Only limitation in this situation.

What if the learner takes their checkride in a Cessna 337? They will be ``limited to centerline thrust'' and
can have this endorsement removed with an abbreviated checkride in the future.

\section{MEI Add-On Checkride}

The Flight Instructor ACS spells out the requirements. In short, there will be ground lessons on 
single engine aerodynamics, weight and balance, performance charts, and systems. The flying
portion will include the usual maneuvers plus the drag demo. The examiner will typically have the
candidate walk them through a number of maneuvers.

\section{Resources}

Airplane Flying Handbook, Chapter 13, discusses multi engine airplanes in detail.

Constant speed propellers and other systems are covered in greater detail in other chapters of the PHAK and AFM.

FAA-P-8740-66, Flying Light Twins Safely

FAASTeam Light Twin Takeoff Control \& Performance Briefing checklist.

FAASTeam ALC-30: Multi-Engine Safety Review.\\\url{https://www.faasafety.gov/files/helpcontent/Courses/ALC-30/content/index.html}


