\chapter{The Commercial Pilot}

So you want to become a commercial pilot? What are the privileges and limitations for such a pilot? What does that even mean?

One thing is for certain: the private pilot checkride is NOT simply a ``glorified commercial pilot checkride''. The expectations - and risks - for a commercial pilot are much higher.

Think of it this way. The private pilot checkride represents your first flight with a passenger (albeit a fairly picky one at that). The commercial pilot checkride is meant to simulate your first \emph{job interview} as a pilot. It's all about professionalism, polish, and positive control.

The private pilot checkride is your change to demonstrate competence. The commercial pilot checkride demonstrates fluency, professionalism, experience, and finesse. Of course, the ATP checkride takes this to another level, but we'll talk about that another day.

The PHAK \cite{PHAK} is a great reference. So is the AFH \cite{AFH}. It's good for commercial pilots to be familiar with hazardous attitudes \cite{adm} as well.

\section{Training}

The Commercial Pilot is held to a higher standard than the Private Pilot. Many of the maneuvers on the private pilot checkride make an appearance on the commercial pilot checkride as well. Quantitatively, the commercial pilot has less margin: tighter limits for airspeed, altitde, heading, bank, landing distance, etc. Qualitatively, the checkride examiner wants to see that the commercial pilot candidate is the clear master of the aircraft. They are looking for positive control, smoothness, rudder coordination (and cross-coordination when appropriate!), and appropriate use of trim or checklists.

With the private pilot we ask, would I trust the candidate to take my spouse or child flying? With the commercial pilot we ask, would I trust the candidate to take my entire family flying? Are they the clear master of the aircraft? Can they explain the aircraft's, as Doug De Muro would say, quirks and features? Do they make the airplane do what they want it to do, when they want it, for reasons they can clearly explain?

For the author, the process of commercial pilot training was humbling. I already had about 100 hours in my airplane, was already instrument rated, and felt like a safe, competent pilot. How hard could it possibly be? I thought I could knock out the training in about five hours of dual instruction (in addition to all the other requirements in 14 CFR 61.129 of course). I thought wrong.

Commercial pilot training forced me to be a student pilot again. It was frustrating. It was humbling. I was practicing many of the same things that I had already demonstrated as a private pilot, but in a larger, faster airplane that was not a Cessna 172. (The author feels that there is an implicit bias toward the Cessna 172 in pilot training materials.) I was being forced to look outside - after learning to fly as an instrument pilot. I was being asked to put G forces on my body and airplane that were not necessarily comfortable, and would be extremely concerning in an instrument environment. Not only was I having to re-learn things that I had previously learned in a different airplane with a different instructor, but I was also being held to a higher standard. That standard, of course, is the Commercial Pilot ACS \cite{acs-commercial}.

The ``good news'', at least for me, was that I was not immediately dependent on passing my checkride to gain any new privileges. I didn't have an airline job lined up. I could still fly myself and my family around if I wanted to. After some initial frustration, I learned to re-embrace the learning process, and that it's okay to postpone a checkride once, twice, or even more times. Cancellations are free, but disqualifications will cost you.

If I were to go back in time, and talk to myself when beginning, I would offer the following pieces of advice:

\begin{itemize}
    \item This is going to take longer than you expect.
    \item This is VFR training with an instructor. Schedules, airplane maintenance, and weather all need to align. Often times they won't.
    \item For every hour in the air, spend three hours on the ground, split evenly between debriefing the previous flight and preparing for the next flight.
    \item Try to remember to have fun with this! Flight training is an incredible privilege that not everyone is able to do.
    \item Just because examiners book a month out doesn't mean you need to book immediately. A concrete date can be a good thing (motivation) and a bad thing (an unnecessary stressor).
    \item The difficulty of commercial pilot training is a direct function of competency at the end of the private pilot checkride. If soft and short field operations, steep turns, and emergency descents remained regular operations, things will go easier. If, however, it has been two years since the words ``steep turns'' came out of your mouth, this might take a little while.
\end{itemize}

Ultimately, my own goals include being a flight instructor (for which the commercial rating is a prerequisite) and being competent in my airplane (commercial training certainly explores the flight envelope).

\section{Ground Operations}

TODO. Spend some time talking here about what makes a commercial pilot more senior than a private pilot on the ground. Do so in ACS order. Talk about airworthiness requirements, weather information, cross country flight planning, the National Airspace System, Performance and Limitations, Operation of Systems, Human Factors, Preflight Assessment, Flight Deck Management. This ends up reading like a checkride cheatsheet.

\subsection{Engine Starting}

At the commercial pilot we kick everything up a notch, even something as humdrum as starting the engine. Surely we've started an ending before?! But now, we should be able to talk a little more about what's going on.

TODO: talk about how in starting the engine, the prime makes the fuel mixture rich, and cranking makes the mixture progressively leaner until we catch.

TODO: talk about gear driven starters and how to not damage them.

TODO: discuss whether or not to leave the alternator on while starting. Depends on POH. Belt vs gear driven alternators.

TODO: discuss importance of ground lean.

\subsection{Taxiing}

On the checkride, make sure to use an appropriately slow taxi speed. Rushing on the ground will only make the checkride end faster by getting you disqualified. 10 knots is a good ground speed. Feet off the brakes and minimal RPM for this. Don't forget crosswind controls.

\subsection{Before Takeoff Check}

Follow the checklist.

What would you do if the engine ran rough?

Takeoff briefing. Describe what we would do if there were an obstruction on the runway. What kind of takeoff is this? Where do we expect to be airborne? What do we do if our engine quits on the ground roll? On rotation? At 100 AGL? At 500 AGL? Making a 360 is a bad idea. Making a 180 might be a good idea. Using a crosswind runway is an option. Maybe scope out landing sites just off the departure end of the runway ahead of time.

The Lance has an STC for GAMIjectors, which explicitly allow us to lean the mixture to peak EGT for takeoff. I use this as my justification for leaning the mixture to get the RPM drop in the mag check to where I need it to be. At this point, in the Lance, I might point out that the D in Lycoming IO-540-K1G5D stands for ``dual magneto'', meaning both magnetos are driven by a single gear. This is a single point of failure for the magnetos, which is an unfortunate design.

\section{Flight Operations}

The Commercial Pilot ACS \cite{acs-commercial} is the bible for the Commercial Pilot checkride. We refer to it frequently in this section.

The author has, as of this writing, spent the most time in the P32T type, a T-tailed Piper Lance II. The detailed procedures here implicitly refer to the Lance, and its airspeeds and quirks. Some details will differ based upon the particular aircraft.

The Lance has a wide range of flight envelopes. It flies very differently when nose loaded with a forward center of gravity (CG), as opposed to when aft loaded with a rear CG. The numbers and profiles in this section assume a ``checkride profile''. This includes a pilot and an observer in the front seat totaling 440 lbs, 50 gallons of fuel in the tanks, 20 lbs of baggage in the nose, 40 lbs of bags on the second row seats, and 50 lbs of baggage in the tail. This should give a CG station of about 86 inches. When practicing maneuvers for the checkride, it's important to get the plane set up almost the same every time. For me, this meant flying with a number of safety pilots, who were more than happy to provide ballast in the front right seat.

I did these maneuvers in Texas winter and early spring weather. This generally meant field elevations of 1000 AGL and density altitudes of about 1500 AGL when landing. I've found the Lance to be extremely sensitive to density altitude, particularly when landing. At density altitudes approaching sea level, the Lance wants to float on the runway, and it may be necessary to cut power four or more hundred feet early to make a mark. Conversely, at high density altitudes, the Lance becomes even more of a flying brick and wants to sink through ground effect.

On checkride day I would recommend doing the maneuvers that day to see what the airplane wants to do.

But first a thought. Why do we even do these maneuvers? Doing a chandelle or steep turns with paying passengers who don't expect those maneuvers is not a great idea. The lazy eights and eights on pylons have almost no practical application. The best rationale I can come up with is this: they demonstrate mastery of a complex, albeit fairly arbitrary, maneuver. As a commercial pilot you won't do these maneuvers. But you could be doing crop dusting. Or, you could be doing aerial surveillance, which requires flying back and forth along the ground. Or, you could be flying an aerial photographer around, who needs to keep a photographic subject in one place. Thinking of it as a job interview, the mastery of these maneuvers gives a potential employer a \emph{baseline} of performance. It's sort of like inverting a binary tree on a programming interview: you'll never do it in practice, but you show it to the interviewer to demonstrate mastery.

For consistency, I will explore these topics in the same order in which they appear in the Commercial Pilot ACS in the subsections that follow.

\subsection{Maneuvering Speed}

Read \url{https://www.planeandpilotmag.com/article/understanding-maneuvering-speed/}.

Read \url{https://www.federalregister.gov/documents/2010/08/16/2010-20195/maneuvering-speed-limitation-statement}. Even if this applies primarily to transport category airplanes, the same rules of physics govern our normal category single engine airplane. The discussion is relevant.

Maneuvering speed is defined at the speed at which a full control deflection will cause the airplane to stall before it breaks. Since it incorporates the airplane's intertia, a lighter airplane will have a lower maneuvering speed than a heavier airplane.

Na\"ively, we expect that the maneuvering speed varies with weight. But, since there is a regulatory aspect to this, we need to check the POH. In the Lance, somewhat confusingly, we're told to just linearly interpolate.

Maneuvering speed is a maximum, not a minimum.

It's possible that the placarded maneuvering speed for the Lance - 132 knots at max gross - is an optimistic number determined by a test pilot with a new airplane in ideal, controlled conditions. For the reasons described in those articles linked above, this may not be realistic.

In flight, the airplane can respond, with feedback, on how smoothly it handles.

I've foind that, in the aforementioned ``checkride profile'', a maneuvering speed of 120 knots is safe for the airplane and occupants, is in keeping with all available maneuvering speed guidance, and allows us to fly the commercial pilot maneuvers adequately. So, that's the speed we'll use for the rest of the chapter.

\subsection{Traffic Patterns}

By now, traffic patterns should be old hat. But there are still some improvements I made along the way.

I sometimes had a tendency to fly tight traffic patterns. While this is fine for a long succession of take offs and landings for currency, it's not ideal and reinforces bad habits. The traffic pattern should not be a racetrack. (Well, it should for the power off 180, but that's a topic for later.) It should have a clear crosswind, downwind, base, and final leg. The crosswind leg should be long enough so that the runway is off the Lance's wing tip, and so that the four legs of the traffic pattern hane a distinct ground track (since it's so easy to check with ForeFlight or FlightAware now).

A wider traffic pattern gives us more time to spot other aircraft and more time to get set up for short and soft field landings. The Lance has more energy than a 172 and doesn't want to slow down right away so we need to plan that out some.

The turn to base should be with the runway threshold 45 degrees behind.

Pay close attention to the wind. In the pattern, is the wind blowing you toward or away from the runway? Crosswind and base should have the same ground track distance. But, one will have a longer duration on account of winds.

Pay close attention to an RP on the sectional chart for a right hand traffic pattern. For the checkride, scope out potential right patterns ahead of time. Especially if much time has been spent doing left handed takeoffs, landings, and power off 180s, a right pattern could really cause a problem if they are unfamiliar!

A traffic pattern likely involves a landing. It may be a normal landing, a soft field landing, or a short field landing. Refer to the appropriate section for each landing.

\subsection{Normal Takeoff and Climb}

This one should be a gimme. We've taken off as many times as we've landed, and most of those have been normal take offs. So, instead of belaboring the basics, we'll jump to the gotchas.

Crosswind controls and left rudder application will be critical. The examiner will want to see that the airplane takes off smoothly and keeps going. No wing dip. No hunting for the center line. No getting blown off course.

We're looking for two callouts. ``Airspeed alive'' once the airspeed indicator is off the stops. ``Engine good'' to make sure the gages are in the green. It's okay for oil pressure to be momentarily high. The Lance uses Aeroshell W100 Plus, which has the consistency of treacle (molasses) at normal temperatures. Even if the oil temperature indicator is still in the green, the entire oil system is still coming up to temperature. We'll see oil pressure up to 107 PSI.

We'll also see the engine get up to 2720 RPM, 20 RPM over redline. This could be a governor or engine tweak that needs to be done or a slightly miscalibrated sensor. These are normal in the Lance.

In a retract, we're looking for an extra callout. ``Positive rate, no usable runway, gear up.'' Be familiar with the max landing gear retraction speed (109 knots in the Lance) and the maximum landing gear extended and extension speed (both are 129 knots in the Lance). When I take off in any airplane, I like to tap the brakes once a departure is assured. With a retract, there's no sense in storing spinning wheels, it just wears out the wheel brakes in the wheel wells. In any airplane, stopping the landing gear from spinning early on helps to avoid an unexpeted resonance through the landing gear mechanism. This was most prominent on a particular Cessna I used to fly: about ten seconds after takeoff, the aircraft would shake violently as an unbalanced wheel spinning down triggered a resonant mode in the fixed landing gear.

But! In the Lance we need a healthy dose of right rudder on takeoff. So, just before I reach for the gear knob, and with my right foot still on the rudder, I grab the parking brake to stop the wheels.

In the Lance, dirty configuration Vy is the same as clean configuration Vx: 87 knots.

Since the Lance is special I would make sure that the examiner is familiar with the quirks of a takeoff in the Lance. I would explain that the horizontal stabilizer, being small and out of the slipstream, doesn't become really active until about 80-85 knots.

My normal takeoff in the Lance involves keeping the plane on the ground (with neutral or slightly forward elevator) until 85, then smoothly rotating. At that point, even the tricky T-tail Lance can fly itself off the ground like a Cessna.

We don't need an excessive climb rate. Nose to the horizon or just above it in the Lance will give us a nice 500 to 1000 foot per minute climb. That's all we need.

As a personal minimum, I fly runway heading at Vy of 92 knots until we are 500' AGL.

In the Lance, we pull it back to 25-25 - meaning 25" of manifold pressure and 2500 RPM - once we're through 500' AGL. This is for noise abatement. There is some controversy as to how necessary this is for engine longevity, but I like to make things just a little quieter for passengers and for people on the ground. This does mean the black knob keeps coming forward as we get up to altitude.

Once we're 500' AGL, the fuel pump and landing light can come off. At night, I might keep the landing light on just a little longer if there is heavy traffic in the area. When switching the fuel pump off, watch the engine gages. If any substantial loss of power is indicated, get the fuel pump back on and consider a precautionary landing.

At this point, I either get ready to enter the traffic pattern, or establish a cruise climb. A good cruise climb in the Lance is 105 knots. That provides a good climb rate and plenty of airflow over the engine.

Yep. At least an entire page on just a normal takeoff. You can see how the rest of this section will go. Lots to think about as a professional pilot!

\subsection{Normal Approach and Landing}

A DPE friend once shared his thoughts on transitioning to a new airplane, such as a Cirrus. He said it would be better to do a hundred landings in the plane than fly it around for a bunch of hours.

I will start by saying that the Lance is a fairly difficult airplane to land. It's a little plane that flies like a big plane. It has high wing loading and relatively fast approach speeds for a single engine airplane. Unlike a Cessna, which will gladly float in ground effect for a long while, the Lance will gladly sink right through ground effect, depositing itself on the ground with an unwelcome thud.

At the commercial pilot level, we're all about precision. Even though the ACS gives us 100-200 feet of leeway for many of our landings, we want to practice being on the mark, every time. The 1000 foot markers are going to be our target for these and all other landings in this chapter.

Pundits on the Internet think that the Lance's landing quirks are due to that pesky T-tail, with a small horizontal stabilizor out of the wind. That may be somewhat true. I believe the relatively simple wing cross section and high wing loading are bigger factors.

Bigger airplanes - airliners - are flown ``by the book'' and ``by the numbers''. I've found, experimentally from flight, that the Lance has three landing profiles. Let's call them fast, medium, and slow. A normal landing is a medium landing profile.

Fast: We're coming down final at 105 knots. The airplane lands with a decent amount of energy and floating down the runway is somewhat inevitable. This is the instrument approach profile, and suitable for use at large airports with long runways and an abundance of fast, heavy traffic. If the pilot is successful in flying the airplane in ground effect - achieved by cutting the power and lifting the nose to the horizon or just above - the pilot will be rewarded with a gentle touchdown. The tradeoff is that we use a ton of runway - expect a 3000 foot ground roll on a standard day at 1000' MSL.

Since the Fast approach is the staple of the instrument approach, I'll relate it to an instrument approach here. At the initial approach fix (IAF) fly normally and become established on course and on glide slope. At the intermediate fix (IF) slow to 129 KIAS and get the landing gear down. At the final approach fix (FAF) complete a GUMPFSS check, which will include getting the first notch of flaps down and flying a stabilized approach at 105 KIAS. Put in the second notch of flaps at or by the decision point when the field is sighted (or go missed). Put in the third notch of flaps once close enough to the airfield that a landing is assured.

We probably won't see this one on a commercial checkride for a single engine airplane, since they don't make you fly an approach, but it will be a staple for the instrument checkride.

Medium: We're coming down final at 95 knots, slowing to 90 and then 85 knots as we round out. This is a typical VFR traffic pattern and visual approach. It's the approach the plane wants to fly without being too far on the backside of the power curve (recalling that L/D max in the Lance is 92 knots clean).

Let's walk through an entire normal approach and landing in the Lance.

The approach begins on the down wind leg. I like to give myself plenty of time to get set up for this correctly. This means I'm entering the traffic pattern well before the departure end of the runway of intended landing, already at altitude. My spacing is such that the runway is off and just past my wing tip. Wider approaches give us more room in the Lance. If challenged by an examiner, explain that the Lance is a bigger, heavier airplane than a Cessna and needs more time to bleed off energy. Airspeed should be gear extension speed at most.

Abeam the numbers, we are on airspeed (105-129 knots), on altitude (1000' AGL in most cases; this procedure may need to be scaled for airports that have nonstandard pattern altitudes like KPAO), on heading (reciprocal of runway heading plus wind correction), and at the correct spacing. A good landing begins with a good approach, and a good approach begins with a good set up.

Still abeam the numbers, we reduce the power and complete our GUMPFSS check. Gas - most full tank. Undercarriage - confirm 129 kias, gear comes down, plane starts slowing down. Mixture - full rich. Prop - forward for max speed. Flaps - confirm 109 kias, first notch of flaps comes down. Seatbelts and harnesses fastened. Fuel pump and landing lights on.

We should be established in a smooth 500-700 foot per minute descent at about 100-105 knots.

With the runway numbers 45 degrees behind us, it's time to turn from downwind to base. Make a smooth, coordinated turn, making sure to keep the descent and not exceeding 30 degrees of bank in the pattern. Once wings are level, put in the second notch of flaps. Still descending at 500-700 feet per minute, the airspeed should be right on 95 knots. Pitch for airspeed, power for altitude and sink rate. The power setting here varies wildly based on density altitude and airplane loading, it can be as low as 12 and as high as 20 inches.

When approaching the extended runway centerline, it's time to begin the turn to final. Maintain coordination and maintain the descent. Become aligned with the runway and establish a glide slope based on a VASI, PAPI, or visual estimation. (Be aware of airfields that have nonstandard PAPI angles. Runway 15 at KRYW is set up for a 4.0 degree glide slope. But the RNAV RWY 15 approach is configured for a three degree approach, so we can fly down at three degrees safely. Just be aware that you'll see all red lights when you do so.)

There will inevitably be a crosswind on final. It will probably be from the left (in the northern hemisphere). In the Lance, I strongly prefer to fly a crab angle as opposed to a forward slip. There are a few reasons for this. One, a crab angle is easier to stabilize - it is what I would use on a long instrument approach. Two, a crab angle doesn't require cross coordination of controls and is somewhat safer, especially as we have a more aft CG in the Lance. Three, a crab angle is naturally coordinated! I'll cross control closer to the ground, using yoke for left-right alighment on the runway centerline and rudder for rotational alignment of the nose with the runway. The Lance has enough adverse yaw that it takes less rudder than you might expect - still plenty, but not a ton. On the other hand, it does take slightly more yoke than I usually expect, and it needs to keep coming in as the airplane slows to a stop.

Once a landing is assured and the runway is made, put in the final notch of flaps. That last bit of flaps induces a decent amount of pitch-up moment. Be prepared for that. We should be slowing so that we're 90 knots across the runway threshold and 85 knots at wheels down. Confirm three greens here - the landing gear had better be down or else we're going around. A power adjustment may be necessary. Be careful not to get too slow.

Here's the secret to a perfect spot landing on the 1000 foot marks. Cut power 2-3 stripes before the marks. If we're on glideslope and on airspeed, this will work perfectly.

The roundout in the Lance for a normal landing involves bringing the nose to or a few degrees above the horizon and trying to keep the wheels off the runway as long as possible. Think about slow flight at 3' AGL with the engine out in the dirty configuration for as long as possible. Assuming we didn't allow ourselves to get super slow on final, the wheels will let down with light to moderate force and we can proceed to brake.

In the Lance, the flap selector and gear selector are very easy to tell apart. One is a small round knob. The other is a big long lever. Once we're on the ground, we can retract that flaps to get more weight on the wheels for braking.

I'll have more to say about go-arounds later, but for now, the procedure is: all three throttle controls forward, then in order: flaps, gear, flaps, flaps.

\subsection{Soft-Field Takeoff and Climb}

The soft field takeoff and climb demonstrates that the pilot is able to operate the aircraft safely on a dirt, grass, or other similar off-pavement airstrip. The constant theme is unweighting: keeping weight off of the tires, particularly the nosewheel. A secondary theme is smoothness.

We configure the aircraft with two notches of flaps and one or two turns of nose-up trim past the neutral point. Once we begin moving the aircraft, we continue to move until we are airborne or abort the landing. Back pressure on the yoke will transfer as much weight as possible from the nose wheel to the mains. The Lance has very little elevator authority at low speeds so this effect will be minimized.

Taxi smoothly and continuously onto the runway. Apply takeoff power, at perhaps half the rate as usual (taking 6-8 seconds to advance the throttles instead of 3-4) to help prevent kicking up debris. Holding firm backpressure, accelerate down the runway.

The Lance's $V_x$ speed with gear down is only 68 knots. The Lance usually doesn't climb until 70 knots. So, in this aircraft, it may not be necessary to accelerate in ground effect, as we might with many other aircraft. The pilot must be mindful of this.

Once we are airborne at rotation speed above $V_x$, we hold that to clear a 50- or 100-foot obstacle. Then, once positive rate is confirmed and the obstacle is clear, we retract the gear, lower the flaps one notch at a time, and continue accelerate to our gear up $V_x$ of 87 knots, which just so happens to be $V_y$ with gear down and flaps up. After that, we continue to accelerate to a clean $V_y$ of 92 knots, and continue to climb from there.

A soft-field takeoff is not necessarily a short-field takeoff as well. If it is we need best rate to clear an obstacle. If not, we don't. Best ti clarify this before beginning the maneuver.

Once airborne, the airspeed should be monotonically increasing. Don't sink! Don't stall! Don't let the stall warning go off. Keep the nose coming down and the airspeed and altitide coming up. Once 500' AGL, proceed as you would for a normal takeoff and departure.

Disqualifications might include: porpoising on the ground due to incorrect transition to ground effect flight (which we shouldn't need in the Lance), stopping once we begin the ground roll, inappropriate use of controls, incorrect airspeeds.

\subsection{Soft-Field Approach and Landing}

The soft field landing demonstrates that the pilot is able to manage the aircraft on a surface other than a paved runway. Much like the soft field take off, the goal of the maneuver is to keep as much weight off of the tires, particularly the nose tire, for as long as possible. Note that, by default, the soft field approach is NOT also a short field approach. However, the Lance's POH does not differentiate between them.

The landing begins as a typical short landing: wheels down at 75 knots, full flaps. In the Lance, a blip of throttle is needed to allow the main tires to gently set down on the ground as opposed to slamming down and potentially digging in. Then, we continue to hold back pressure and leave the flaps down to gently bring the nose down, doing so as late as able. We can lower the flaps to two notches as soon as able, and have the option of leaving them down as we taxi. We must not stop until we are clear of the active.

Another option is to do a power on approach and landing. This takes longer, but you can really take your time to grease the wheels onto the runway when doing so. This has become my preferred approach.

In the Lance, we typically do these maneuvers with half a tank of fuel and two front seat occupants pushing the station weight limit of 440 lbs. This leads to a fairly fore CG condition where the aircraft is nose heavy. When the aircraft is aft loaded, the nose becomes MUCH lighter, and it is possible to wheelie all the way down the runway unless we are very careful.

Note that the soft field landing has no aiming point specified. This is good, since a true grass strip won't have markings.

Disqualifications might include: bouncing, slamming the nose wheel, stopping on the runway.

\subsection{Short-Field Takeoff and Maximum Performance Climb}

In the short field takeoff, we are looking to use a minimum of runway, lifting off at the earliest spot possible, and to clear a 50 foot obstacle. To use the minimum runway, we position the aircraft as close to the very end of the runway as possible, using a displaced threshold to our advantage. In the Lance we use two notches of flaps. We hold the brakes, apply full power, lean if appropriate, and release the brakes. Seriously - feet off the brakes!

We recall that $V_x$ in the Lance is a mere 68 knots in the dirty configuration. We're lucky if we are airborne at 70-75. Maintain that airspeed until clear. Then it's gear up and flaps up as we accelerate through $V_x$ of 87 knots to $V_y$ of 92 knots.

Disqualifications might include: forgetting to position the airplane as far as possible on the end of the runway, incorrect flap settings, forgetting to run up with brakes held, incorrect airspeeds.

\subsection{Short-Field Approach and Landing}

The Lance doesn't like to be slow as this maneuver reminds us.

The short field approach is an important practical maneuver. On a checkride, we usually have the examiner give us an aiming point, and it's usually the 1000 foot markers. In real life, that aiming point is the numbers as we're looking to truly use minimal runway.

In the Lance, we set up for a full flaps stabilized approach diung 75 knots over the numbers. The plane will want to sink quickly when we cut power so a small dose of throttle will set us up for success. However, it's imporant to be within 100 feet of the required spot, so a hard landing is better than a failed checkride. Recalling that the 1000 foot markers are 200 feet ling (source?), the goal is to be ON the markers.

Once down, retract flaps (easy in the Lance, lower the lever to the ground), apply full back pressure, and call out ``maximum braking''. It's not necessary to actually brake hard, we don't need to leave tire marks on the runway.

Disqualifications - and there are many for this one - might include: improper configuration, missing the aiming point, bouncing so hard that we porpoise and have to go around, forget to call out ``maximum braking''.

\subsection{Power-Off 180$^\circ$ Accuracy Approach and Landing}

This maneuver is probably responsible for more failed commercial pilot and CFI checkrides than all the others combined. It is unforgiving and requires a very high degree of precision, particularly to correct for winds. The candidate gets one shot at this one. If they land short, or land long, or have to go around, that's a disqualification. It still makes sense to continue the checkride, but if this happens to you, expect to lick your wounds, spend an hour with a CFI for an additional signoff, and to take another shot in the coming weeks.

Recalling the old joke, ``you are never too high or too fast in a Piper'', the power off 180 will be faster and steeper than it would on a Cessna. I've had some safety pilots get really quiet and nervous as we do this maneuver, thinking we're going to crash and die, only to watch me put the plane down perfectly on the 1000 foot marks. I would be sure to brief the intricacies of this maneuver in the Lance prior to execution.

For this maneuver, we fly a traffic pattern with a closer than usual downwind leg ($\frac1 3$ of the way up the Lance's wing instead of just off the tip) and cut the power abeam our touchdown point, the 1000 foot markers. We immediately pitch for our best glide speed of 92 knots, turn a fairly tight pattern, put the gear down once the field is made, and execute a normal, possibly no flaps landing. This ends up being a fiarly aggressive, tight, rounded approach, with no real base leg, as downwind transitions smoothly to base and final.

DO NOT FORGET TO PUT THE GEAR DOWN. If you're less than 50' AGL and don't see three greens, GO AROUND. This isn't worth a new engine.

With respect to the aiming point, there are three possible outcomes:

\begin{itemize}
\item{Too short.} We've lost too much energy and have no hope of making the aiming point. Perhaps our pattern was too wide, we did not correct for wind, or we otherwise failed to manage energy. Identify and call out the situation, and immediately GO AROUND. Whether or not this is disqualifying, this is the correct course of action.
\item{On point.} This is ideal. But getting it magically right by chance means we're not really in control of the aircraft. So, even though this will pass the test, it's not the best place to be.
\item{Too far.} We have not lost enough energy. As long as we don't have \emph{such} an excess of energy that we cannot manage the airplane and the landing, this is fine. We have some tricks to get lower: cut the airspeed (down to 75 instead of 87-92), lower the flaps, slip with flaps extended (approved in the Lance!).
\end{itemize}

You've got 200 feet from the aiming point for this one. If the aiming point is the near side of the 1000 foot markers, the limit will be the end of them. You'll know immediately when you've done this one correctly, and so will your examiner.

Assuming you do this correctly, this is arguably the hardest part of the checkride. So spend plenty of time practicing.

Setup is absolutely critical for this one. If the setup isn't right, it's technically not a go around if you do a 360 in the pattern for spacing. The examiner may frown on this. Make sure to allow plenty of time for a complete, stabilized downwind leg. Also, we don't want to be \emph{too} fast downwind. Best glide is 92. Being at about 105 is right. If we're at 130, that's all that much more extra energy that we need to manage.

Even the Lance will float a tiny bit in ground effect, so we've started aiming for two lines prior, flaring and floating to the marks.

In the Lance, the bank angles and sink rates may seem excessive. I've had at least one safety pilot / right seat CFI get nervous. So, with an examiner, make certain to brief this.

Disqualifications include: failing to make the aiming point (obviously), failing to go around when appropriate.

Slamming the plane on the ground on the marks isn't great. Being off center line can be disqualifying. Failing to make a crosswind correction can be a huge problem. If the wind is blowing you toward the runway, consider a wider pattern. If there is a strong wind blowing away from the runway, consider an even tighter pattern. We've practiced these almost exclusively to the left, which is preferred with the airplane's left turning tendency (partially caused by even a windmilling propeller).

Instructors and examiners get really nervous if you take your hand off the throttle for this one. Keep the hand on the throttle, ready to go around.

\subsection{Go-Around/Rejected Landing}

Flaps, gear, flaps, flaps.

Go-arounds in the Lance are, for the most part, a non-event. The airplane has plenty of reserve power at lowland altitudes and ample power at higher density altitudes.

All of the knobs come forward at once. Full power, high RPM or low pitch propeller, full mixture. If it's known to be a high density altitude the mixture can come back slightly. As part of the landing checklist we should have already set the propeller and the mixture so it's just the throttle that is moving smoothly forward. (Smoothly means, about 4 seconds from idle to full power. Don't slam it forward in half a second and risk de-tuning the engine counterweights.)

Offset from the runway as appropriate. I will offset if there is any traffic on the runway at all. For all I know, they are not talking on the radio and could decide to take off. I will offset on the side opposite the traffic pattern and maintain visual separation.

In the Lance, since that last notch of flaps gives us so very much induced drag and includes a pitching moment, the very first thing we do is to get the first notch of flaps out. We confirm a positive rate of climb, then bring in the first notch of flaps.

Then the landing gear comes up. Making sure we are below the landing gear extension speed of 109 knots, we confirm a positive rate of climb and bring the landing gear up.

Now, I bring back the remaining two notches of flaps, one at a time, confirming a positive rate of climb before and after each one.

While doing all of this, I'm devoting most of my attention outside the cockpit, looking for traffic.

I'll maintain Vy of 92 knots up to 500 AGL.

\subsection{Steep Turns}

First of our maneuvers.

Clearing turns are critical or else we fail. S turns or 360.

Maneuvering speed. Power comes in. Pick reference points.

50 degrees bank. Overbanking tendency.

Critical to maintain sight picture.

HEAVY backpressure in the Lance. Two hands! Hard to trim in time. Don't enter a dive, fight for it.

You know you've done well when you hit your own wake turbulence. Bump bump.

Once to the left, once to the right.

\subsection{Steep Spiral}

Old joke: ``In a Piper, your best field is directly under you.''

Not the same as emergency descent. This is a ground reference maneuver. Purpose: getting set up for an emergency landing from altitude.

Power comes to idle. Pitch and trim for best glide in the clean configuration.

Which way is the wind coming from?

Tip: clear engine into the wind to help correct wind drift.

Easy to bust altitude on this one. We lose about 1200' every 360 degrees.

Easy to lose track of how many turns we've done. Set the heading indicator. Keep track on a kneeboard if need be. We're looking for three turns. We're ending on a particular heading or pointed at a reference, without busting 1500' AGL.

This is not an emergency descent. We'll talk about that later.

\subsection{Chandelles}

Ah, the chandelle. That classic commercial pilot maneuver. I wonder how long it's been part of the ACS, the PTS, and whatever came before it. I wonder if my grandfather-in-law had to demonstrate one of these in the early 1940s.

At its core, the chandelle is a combination of a 180-degree turn and a maximum performance climb. Our goal is to enter the maneuver at maneuvering speed, and to end the maneuver with a blip of the stall horn pointed in the opposite direction.

Watch \url{https://www.youtube.com/watch?v=Ml8YI7oj2Q8}. The UND commercial pilot maneuvers are pretty good. They are close to how I would teach the maneuver. Of course, like many training materials, they are heavily biased toward the Cessna 172. In the rest of this section we'll add some refinements and take the quirks of the Lance into consideration.

Just like the standard traffic pattern, the key to success in the chandelle - and most any other maneuver - is to break it up into sections. This isn't a maneuver where we make one control input, wait a minute, and hope we got it right. This is a maneuver where we have some key reference points.

It helps me to break this maneuver into two independent axes: the pitch axis, and the roll axis. Let's analyze in turn what each of these are doing before bringing them back together. (I picked this order of axes intentionally.)

In the pitch axis, we're looking to climb as much as possible without stalling. We smoothly lift the nose to about 12-15 degrees of pitch in the Lance, come in with a little power, hold right rudder to maintain coordination, and hold that pitch angle until the stall warning horn goes off. As the climb progresses, more and more backpressure will be needed to keep the nose up, and more rudder will be needed, as the speed bleeds off. Once the stall warning goes off, we level off and maintain altitude, allowing the nose to continue falling without ballooning or losing altitude.

In the roll axis, this is somewhat more than a 180 degree turn. The airplane rolls immediately into a coordinated 30 degree turn. The complication here is that we cannot maintain a 30 degree bank angle the entire way through. If we were to do so, we would stall the airplane, since our pitch change is causing us to lose airspeed. Further, this would sharpen our turn rate, since turn rate increases for constant bank and decreasing airspeed. So, starting about halfway through the turn, we need to start rolling to wings level. Indeed, through the second half of the turn, we're chasing a corner of the flight envelope. As we get slower, we roll out slightly, which drops our stall speed and lets us fly a little slower still. As a result, the first 90 degrees of this maneuver will take much less time than the second 90 degrees.

The common instruction technique breaks this maneuver down into two 90 degree segments. I found, in my training, that this was not sufficiently granular for me to enjoy consistent success with this maneuver. So, we found 45 degree increments worked much better.

In the Lance, as usual, we enter this maneuver at 120 knots. Our exit speed is 70 knots. This is maybe a knot or two above (power-on, clean) stall speed and sufficient for giving us a little blip of the stall warning horn when the maneuver is complete. For power, we do not use full power for this maneuver. Unlike the Cessna 172, the high-performance Lance, with a 300 HP engine, would have a really difficult time finding a stall at full power. We could have full backpressure and just hang off the stall warning horn indefinitely. So instead, we do this maneuver at 65-75\% power. That means, maybe adding 2-4 inches of manifold pressure. After that initial power adjustment, this is a constant power maneuver. I fly all of these maneuvers, including the chandelle, at 2400 RPM.

We usually enter this maneuver at 4500' MSL or about 3500' AGL in our area. We can gain 1000 feet in the maneuver. We normally do one to the left, and one to the right. Be warned, at 6500' MSL, the air is thinner and the Lance's performance suffers. So, if doing these for training, remember to descend between attempts.

Begin by flying straight and level trimmed for 120 knots at the desired altide on heading. I set my heading bug as a reminder. Pick visual reference points for 45, 90, and 135 degrees. An intersection of roads works well but make certain the roads are distinct. Don't forget about clearing turns or our GUMPFSS check. In particular, the clearing turns need to clear the airspace above and behind us as well, and in the Lance, the fuel pump needs to come on..

In the first 45 degrees, we immediately roll into 30 degrees of bank, then come in with power, and then start pitching up to 12-15 degrees. This will take quite a lot of backpressure with both hands in the Lance. Moreover, as the speed drops off, this will take more and more backpressure. It's not really worth trying to trim this out - we need both hands for pitch control.

45 degrees in, we're at maximum bank. We call this out. We may not be at maximum pitch yet, that's fine.

The second quarter of the maneuver, 45 to 90 degrees, goes by the most quickly. Make sure that the 90 degree visual reference point is in sight as we'll start doing something once we see it. Backpressure will be increasing as the aircraft slows.

At the 90 degree point, call out ``90 degrees, max pitch, max bank, starting to roll out''. Immediately release 10 degrees of bank, resulting in a 20 degree bank. The maneuver, in terms of time elapsed, is about one third done. This is where the fun begins.

Through the second half of the maneuver, we need to actually fly the airplane. The airspeed will continue to decrease until we're at 70 knots. The bank angle will continue to roll out from 20 degrees to zero. The heading will continue to come around until we've turned 180 degrees. The trick is to get to 70 knots, zero bank, and 180 degrees all at once, without hanging off the stall warning horn. During this time, the pitch remains constant (12-15 degrees of bank) and the altitude will do whatever it does.

It should take about twice as long to get from 90 to 135 degrees as it does from 135 to 180 degrees. Remember to keep the airplane coordinated. Proper coordination can help keep the nose coming around to the correct heading. I found that undershooting 180 degrees of turn was much more common than overshooting it.

The first 90 degrees of the maneuver is about speed and smoothness. The second 90 degrees are about slow flight and managing three variables (airspeed, bank angle, and heading) to converve.

The 135 degree point is a good checkpoint. Are we at about 10-15 degrees of bank? Are we about halfway between 70 knots and whatever our airspeed was at the 90 degree point?

That last quarter will feel like the longest. Fly the airplane. Look inside and outside. Pay attention to all the variables. I call this quarter ``gaming it'' because you're actually flying the airplane here, not just waiting for the maneuver to be over.

While doing all of this, it's more important to look outside the cockpit. Become familiar with the sight pictures throughout the maneuvers.

At the 180 degree point, keep the power in and push the nose over to stay on altitude. As we accelerate, power can start to come out. The Lance likes to balloon - don't let it.

It's common to do one chandelle in one direction, and another in the opposite direction. Give the plane time to get back up to maneuvering speed before attempting another or else the maneuver will come up short.

In the Lance, the chandelle's combination of high power setting, high bank angle, and low speed, results in CHTs getting as high as we ever see them: about 350 F. I would acknowledge this. If it's a particularly warm day, I would give the engine a chance to cool down before doing one in the other direction. Remember, on the checkride, you are the PIC. You can tell the instructor that the CHTs are too high to safely complete the maneuver, and fix this by flying straight and level for about a minute.

\subsection{Lazy Eights}

A lazy eight is a combination of a 180 degree turn, done at the same time as a shallow climb and descent. We turn one way, then turn the other. Its nearest living relative is the private pilot maneuver ``S-turns across a road''. The aerobatic maneuver ``wingover'' is a cousin.

The standard literature describes this as a ``graceful'' maneuver. It is supposed to be ``beautiful''. It is supposed to be ``ballet in the air''. It is supposed to demonstrate ``mastery of the aircraft'' or somesuch.

Such subjective descriptions are of zero use to the poor pilot who attempts this maneuver. We need some more rigor to derive and describe this maneuver.

The lazy eight is NOT a chandelle. It is a slow maneuver. The standard literature describes it as ``lazy''. Again, that's subjective. How slow? About 40 seconds for a Cessna 152, and about 60 seconds for the Lance. This largely comes down to how many knots of airspeed separate $V_A$ and $V_S0$ on your aircraft. I suspect it has to do with the square of that since we're talking about energy that needs to be dissipated. I'll get back to that.

Before we go too much further, let's at least talk through the conventional way of teaching this maneuver. It's not a bad start, and for many student pilots more skilled at the controls than the author, it's often enough.

First, let's consider a skateboarder shredding a half pipe. If not familiar, have a look at this video: \url{https://www.youtube.com/watch?v=FJfqcpnH6Ys}. The half-pipe in this case is a U shaped channel, about twice as tall as the skateboarder and about 30 times as long as the skateboard. The skateboarder starts at the top of the pipe. They skate down the ramp, raching maximum speed at the bottom. They climb up the other side, going about as high as the top. Once they reach the top, they turn and go back down. As they go up the pipe, they convert kinetic for potential energy, slowing down as they climb. The minimum speed is at the top of the pipe, where they turn around and go back down.

Usually, to stay on the pipe, the skateboarder makes all their turns in the same direction - to the left or to the right. But, if the pipe were sufficiently long, the skateboarder could do one to the left, then one to the right, continuing on indefinitely down the half-pipe.

One thing. The half-pipe happens pretty quickly. What if the pipe were more shallow? Well, then we could slow it down, and take our time turning around at the top.

A lazy eight is similar to shredding half-pipes (in alternating directions) in the sky with the airplane, except we make it take longer to demonstrate that we can control the aircraft.

In the airplane, we start out by flying straight and level. We pitch up and start turning. The plane climbs. As the plane climbs, it slows down, and our rate of turn naturally increases. At some point about halfway through, we've reached our minimum speed. The plane noses over and starts descending. We end up flying straight and level in the opposite direction.

Seems easy enough. What's so tricky about it? A couple of things. If we do the maneuver too quickly, we risk stalling or spinning the aircraft. If we do it too slowly, it's not really a maneuver, so much as just flying the airplane. So we pick a time scale somewhere in the middle.

Another tricky thing is that it's easy to get disoriented. For this we need good visual references. Not terribly surprising since the commercial pilot certificate is a visual certificate. I'll come back to this.

Another challenge is consistency. We need to impose some order on the maneuver so we can reproduce it consistently. I group being on the correct airspeed and heading under consistency.

The final challenge for now is airspeed management and airframe protection. Too fast and we could rip the wings off the airplane. Too slow and we could stall and spin.

Once you put all those constraints together, you arrive at the following procedure, which is usually how lazy eights are taught.

The pilot selects a clear visual reference line on the ground. A major highway, river, railroad, right of way for a power line, or other sufficiently long, straight feature is suitable for this maneuver. The keys are that the reference line be visible from all angles, and be distinct enough from other things on the ground to avoid confusion, disorientation, and ultimately, maneuver failure. A short road is not great. A small road among many other small roads is not great. The edge of a field is not great. If necessary, scope this out ahead of time using a sectional chart, Google Earth, or any other available resource.

The pilot plans to enter the maneuver flying perpendicular to the ground reference line. To protect the airframe, we want to be flying no faster than $V_A$, the design maneuvering speed for the aircraft. We likely want to be flying slower since we're probably not at gross weight. In the Lance, we enter at 120 knots indicated.

While flying straight at the reference line, but before crossing it, the pilot selects some key reference points. One is at 45 degrees off the nose in the direction of the turn. This point will represent the one-quarter point through the turn. One is at 90 degrees. More than likely this will simply be the reference line itself. Alignment with the reference line will represent the halfway point through the turn. The last reference point is at 135 degrees, or 45 degrees behind the reference line. That point represents the three-quarter point through the turn. Since the turn in one direction will immediately be followed by a turn in the other direction, it would be a good idea to get 45 and 135 degree references on both sides. Try to pick something obvious: a quarry, a silo, a castle, a lake, a river bend. Clouds aren't the worst choice but they do move. A particular field surrounded by other similar fields is probably not a great choice.

We break the maneuver down into quarters. We begin execution once we cross the reference line. This is a constant power maneuver, so don't touch the throttle - 18" of manifold pressure and 2400 RPM in the Lance on a standard day at 4500' MSL is about perfect. We shouldn't need to touch the trim either. Let's assume calm winds for now. Here we go!

In the first quarter, we are, at the same time, entering a shallow turn of about 5 degrees, and beginning to increase our pitch. Control inputs smoothly increase. At the 45 degree point, we are at 10 degrees of pitch, representing maximum pitch, and at about 15 degrees of bank, representing half of our maximum bank. We've gained about 200 feet.

In the second quarter, we are starting to let the nose down while increasing bank to the maximum. Just before the 90 degree point - let's say ten degrees prior - an interesting thing starts to happen with the nose. We are approaching maximum turn rate and minimum speed. Our lift vector is no longer pointing straight up. The nose wants to nose over and slice through the horizon. Let it!

At the half way point, our airspeed is about 10 knots above stall speed, 80 knots in the Lance. Our pitch attitude is level since we've already let the nose start falling. Our bank momentarily reaches maximum bank of 30 degrees. We're thinking about \emph{letting} the nose down and releasing the controls. We've gained about 200 more feet.

In the third quarter, we're rolling out from 30 degrees of bank and pitching down. Close to the three quarter point, we find that the Lance needs some nose-down help from the yoke, so we provide that. The airplane is accelerating and descending at this point. We need to manage that. At the three quarter point, we're back at 15 degrees of bank and whatever pitch is appropriate for our other goals - probably 5 or so degrees down.

In the fourth quarter, we're taking our time to make sure that we roll out on airspeed, and on heading.

We finish the maneuver at the same airspeed and altitude as our entry, but in the opposite direction, over the same reference line. Once we're there, we start a turn in the opposite direction. We can keep doing this indefinitely.

The hypothetical perfect VFR pilot can conduct this maneuver exclusively by looking at the airplane, so they say.

Simple enough, huh?

Well, the author didn't think so. Here are a few more tricks that the author found helpful.

First, watch this video: \url{https://www.youtube.com/watch?v=3Oxbr1PuoSQ}.

The video raises some important points.

First is that the lazy eight is an exercise in managing the overbanking tendencies of the aircraft. Past a few degrees of input, the airplane wants to keep turning. Or, as I see it, we tickle an unstable mode of the aircraft with this maneuver.

Second is the incredible importance of coordination. Sure, skidding or slipping is disqualifying on a checkride, but that's not the important or interesting part. The airplane has a left turning tendency, which we'll need to fight with right rudder. The Lance has a decent amount of adverse yaw that ends up helping us some here.

Third is an important reminder that this is a visual reference maneuver. In the video, the instructor has the panel covered with a sheet of paper. All he does is take a couple of quick peeks at the instruments to confirm altitude and airspeed.

Last is that we can complete the maneuver with fairly minimul control inputs. Well, at least, on a Cessna. But that's not true on the Lance. As my Navy test pilot friend points out, the Lance has \emph{extreme} roll stability, so we have to help it along some.

In addition to the video, I'll add my own observations.

Thinking about the timing of the maneuver helped me to quantify what it means to ``slow down''. As someone who regularly deals with events that happen in the span of about 5 \emph{milli}seconds at work, fifteen seconds feels like an eternity! But quantifying that helped tremendously. Each quarter should take about 20-30 seconds, and a complete turn should take about 90 seconds.

In particular, the first quarter will take a really long time. We use this quarter to try to gain some separation from our starting point. It helps us to fly a wider turn which gives us more options in the later segments. The second quarter sees the most control inputs. In the third quarter, it's important to get out of that 30 degrees of bank quickly or else the turn will be over too soon.

We sometimes had trouble with the aircraft not losing altitude fast enough in the third and fourth quarters. One instructor pointed out a ``half ground, half sky'' sight picture that served as a good reference.

I also thought more carefully about what the aircraft was doing in each of the yaw, pitch, and roll axes individually, and what the aircraft's stability was doing for us.

In yaw: the airplane is turning 180 degrees. The rate of yaw increases through the first half and decreases through the second half. It's probably close to a perfect cosine. The Lance is lightly stable in yaw and doesn't need much help here.

In pitch: We're nosing up, than descending. Yoke back, let the nose fall over, yoke forward. Since we didn't touch power or trim, the airplane \emph{wants} to find its original airspeed and altitude. Zoom up, zoom down.

In roll: We start the plane rolling. We give it a bank angle and maybe a touch of momentum. The airplane wants to keep increasing the bank to some extent. But these things are happening at time constants that end up being faster than our maneuver. So we baby the airplane into, and out of, a gentle bank, not exceeding 30 degrees.

We haven't talked about crosswinds yet. Strictly speaking, this is not a ground reference maneuver. But if we don't correct for winds, we don't stay on our line. Instead, we start drifting. So, through the maneuver, just like in a traffic pattern, we can think about gaming the turn. If we're getting blown away from our reference line, maybe we want to hurry up the first quarter and start our turn sooner. If we're getting blown toward it, maybe we want to slow down through the first quarter and slow down the last quarter.

This maneuver comes back on the CFI checkride, where we might need to teach it. So approaching it with rigor up front pays dividends later.

For anyone who picked up this maneuver more easily, without all this rigor: I envy you.

If you're still stuck, have a look at \url{https://www.av8n.com/how/htm/maneuver.html#sec-lazy-eight}. I don't degree with teaching the attitude-centric or nose-centric view of that explanation, but it provides an interesting cross-check for what I've described above.

Anyway. All the quantification aside, this is supposed to be a \emph{smooth} maneuver. We're not fighting the airplane on the controls or micro-managing it. We're making small, subtle control inputs.

\subsection{Eights on Pylons}

The commercial pilot maneuver ``eights on pylons'' has us turning about two ground reference points, about three-quarters of a mile apart. The idea is to fly around one, then head to the other and turn in the opposite direction. We enter the maneuver on the downwind, at our highest ground speed, and fly it at constant power.

When in the turn, the goal is to keep the wing pointed precisely at the pylon, so that it appears to be frozen in one spot looking out the window.

We recall that this is a constant power maneuver. So how can we keep the pylon in exactly one place? The answer is with altitude control.

As it turns out, to keep the wing pointed at a single reference, altitude and ground speed are closely coupled. I'll attempt to build some intuition for this before diving into a more formal proof.

If we are in a car driving along - at zero altitude - it is impossible to do this maneuver. There is no way to control where anything is in the window whizzing past us. If we were to climb (imagine an overpass), the ground would "slow down". Things tend to get behind us really quickly.

Conversely, if we were to find ourselves up in space, the ground would effectively appear to stand still at the sorts of speeds a single engined airplane could fly (assuming there is any air up there to hold us up). If we tried to turn, we would have to cover an incredibly long distance to keep a point in one place.

The intuition is as follows:

\begin{itemize}
    \item Reference point moving behind the wing. We are too low! Climb!
    \item Reference point moving in front of the wing. We are too high! Descend!
\end{itemize}

So, there is a ``sweet spot'' of altitudes, and it happens to be around 1000 AGL.

Using some trigonometry, vector decomposition, and basic physics, we can prove that, for every ground speed, there is an altitude that allows us to turn with the wing pointed at a point on the ground. Curiously enough, radius and bank angle - through related to one another - have nothing to do with the pivotal altitude.

A really nice discussion of the proof is here: \url{https://www.youtube.com/watch?v=oc9mqDadv\_M}.

The FAA's training materials focus exclusively on ground speed. But, the proof I just shared, also talks about air speed.

Using a spreadsheet, I was able to calculate this pivotal altitude chart, rounded to the nearest 5 feet for simplicity.

\begin{figure}
\begin{center}
\begin{tabular}{ |c|c| }
    \hline
    Ground Speed & Absolute Altitude \\
    (knots) & (feet) \\
    \hline
     60 &  320 \\
     65 &  375 \\
    \hline
     70 &  435 \\
     75 &  500 \\
    \hline
     80 &  565 \\
     85 &  640 \\
    \hline
     90 &  715 \\
     95 &  800 \\
    \hline
    100 &  885 \\
    105 &  975 \\
    \hline
    110 & 1070 \\
    115 & 1170 \\
    \hline
    120 & 1275 \\
    125 & 1385 \\
    \hline
    130 & 1495 \\
    135 & 1615 \\
    \hline
\end{tabular}
\end{center}
\caption{Pivotal Altitude Chart}
\end{figure}

Have a good look at those altitudes. Remember we never want to be lower than 500 AGL if we can help it, especially on a checkride, which means we'll be flying this maneuver ideally at 75 knots at the slowest. Remember, we're turning through the maneuver, and going from a tailwind to a headwind, which means we'll be scrubbing a good deal of airspeed and altitude.

In the Lance, I like to enter this maneuver at about 120 knots, 2400 RPM, 20'' of pressure, so that the speed doesn't get too low.

The commercial pilot ACS wants us to not exceed a bank angle of 40 degrees. Recalling that bank angle is solely a function of horizontal distance to the pylon, we'll want to make sure we keep enough horizontal space from the pylon to enable this.


\subsection{Maneuvering During Slow Flight and Stalls}

Nothing too special about these except the very tight commercial pilot ACS limits: $\pm50$ feet for altitude, $\pm10^\circ$ heading, $+5/-0$ for airspeed (which in the Lance is usually 70), and $\pm5^\circ$ angle of bank. The examiner may want to see these in the clean configuration, or an approach configuration, which could be one notch of flaps, two notches of flaps and gear down, full flaps and gear down. The may ask you to fly straight and level, to turn (usually 15 degrees is about right), to climb or descend, or to do more than one at the same time. If we're doing these power off, of course we cannot maintain altitude, so allow plenty of margin for descending just above stall speed. If we're doing these power on, correct rudder application is paramount.

Remember: pitch (and quite a lot of nose up trim in the Lance) for airspeed, power for altitude.

Give plenty of space above the ground for these. KMDD airport sits at 2805.4 feet MSL. We could be entering these maneuvers at 5500-6000 feet MSL.

For the commercial pilot, stalls may be to first indication (BEEEP!) or a full stall (BEEEP!, heavy buffet, and a very tiny nose drop and excessive sink rate in the Lance).

Disqualifications include: getting below 1,500 AGL. Make that 2000 AGL. Stall recovery can be up to 550 feet in the Lance. Call that 3000 AGL. Forgetting to make clearing turns is disqualifying. Forgetting to stay coordinated could be as well.

\subsection{Accelerated Stalls and Spin Awareness}

Fortunately, these are harder to enter than they are to exit. Also fortunately, we don't have an altitude requirement for this one. Well, except for one: DO NOT get below 3000 feet AGL. If we're especially ham fisted, we'll enter a stall, which - shocker! - is disqualifying.

Set up the airplane for $V_A$, configure as requested, set power appropriately, and enter a coordinated 45 degree banking turn. Don't worry about altitude as long as we're high enough.

Now: PULL. It will take a HEAVY two handed pull in the Lance to get a stall warning.

Hear the BEEEP? Wings level and go around. Done.

\subsection{Emergency Descent}

\subsection{Emergency Approach and Landing}

\section{Maneuver Checklists}

These checklists are specifically NOT presented in ACS order. Instead, they are presented in the order that makes sense for a typical training sortie.

GUMPFSS Landing Check used on maneuvers in this section:
\begin{itemize}
    \item G - Gas: On the most full tank. Switch on fuel pump before changing tanks.
    \item U - Undercarraige: Extend if below 129 KIAS on the Lance. Leave up for upper air maneuvers or power off 180s.
    \item M - Mixture: Full rich. Lean the mixture if full rich causes a marked reduction in power. Leave alone for constant power upper air maneuvers.
    \item P - Propeller. Leave alone for upper air work, push forward for landings.
    \item F - Flaps. As needed.
    \item S - Seatbelts and Shoulder Harnesses. Must be fastened.
    \item S - Switches. Landing light as needed. Fuel pump: switch ON.

\end{itemize}

STEEP TURNS
\begin{itemize}
    \item CLEAR THE AREA + GUMPFSS
    \item Pick reference point on horizon.
    \item Set heading indicator.
    \item Roll right to 50
    \item Power comes in
    \item Substantial back pressure – both hands.
    \item Keep reference point on horizon. Reference point is oil port on top cowling left of nose.
    \item In left turns the nose is above the horizon.
    \item In right turns nose is below horizon.
    \item Lead roll out of turn by 25 degrees.
    \item Our standard is +-50 feet altitude.
\end{itemize}

CHANDELLE
\begin{itemize}
    \item CLEAR THE AREA + GUMPFSS
    \item 18”/2400 RPM or whatever speed is needed for 120 KIAS
    \item Pick a reference point and two side reference points.
    \item Roll to 30 degrees, power comes in.
    \item Pitch right up to 12-15 degrees nose up.
    \item Substantial back pressure – both hands.
    \item At the 45: Call out: Max Bank
    \item 45-90: Back pressure continues to increase.
    \item 90: Call out: Max Pitch. Reduce roll to 20. Keep the back pressure. Maneuver is about a third done.
    \item 90-135: make sure turn rate and loss of speed are not excessive.
    \item 135-180: game it so you’re right at 70 KIAS at 180 degrees.
    \item 180 degrees: BEEEEP on heading. Level off.
    \item Watch the nose. Don’t balloon or sink. Forward pressure as we accelerate.
\end{itemize}

ENGINE FAILURE
\begin{itemize}
    \item A – Airfield. Probably under you or close to it.
    \item B – Best Glide.
    \item C – Checklist. Touch anything related to fuel, oil, mixture, or outside air. GUMPFSS check.
    \item D – Declare emergency 121.5 MAYDAY x 3
    \item E - Execute.
\end{itemize}

POWER OFF SPIRAL aka STEEP SPIRAL (not an ``Emergency Descent'').
\begin{itemize}
    \item CLEAR THE AREA if able + GUMPFSS
    \item Pick a point below. Pick a heading reference.
    \item Heading bug.
    \item Straight to 92 KIAS clean. NOT an emergency descent. Trim for it.
    \item Where is wind blowing? Clear engine into it.
    \item Vary bank to keep point where it needs to be. No more than 60 degrees.
    \item Start high up enough for three turns.
    \item 1500 AGL plus 1200/turn is 5100 AGL MINIMUM.
    \item Smooth recovery after three turns.
\end{itemize}

LAZY EIGHTS
\begin{itemize}
    \item CLEAR THE AREA + GUMPFSS
    \item 18”/2400 RPM or whatever speed is needed for 120 KIAS (sometimes 19 works better")
    \item Pick a road.
    \item Slow slow slow.
    \item Roll to 15 degrees.
    \item Pitch up.
    \item Don’t touch the power.
    \item 0-45: pitching up. Slow slow slow.
    \item 45: Call out. ``Maximum pitch''.
    \item 45-80 work up to max pitch and max bank. You’ll gain 400-600 feet.
    \item 80-100 nose slices through horizon.
    \item 90: Call out. ``Maximum bank''.
    \item 100-135 baby the nose down, back to 120 KIAS.
    \item 135-180 keep pushing that nose down.
    \item Finish on altitude and on heading.
    \item Do it again.
\end{itemize}

EIGHTS ON PYLONS
\begin{itemize}
    \item CLEAR THE AREA + GUMPFSS
    \item 18”/2400 RPM or whatever speed is needed for 120 KIAS
    \item Pivotal altitude is about 800-1200 AGL.
    \item Pick points a mile apart. Wind blowing between them. Enter on a downwind and turn left first.
    \item Think about wind correction. Xs should be symmetrical.
    \item Don’t touch power.
    \item Spot ahead of the wing? DIVE.
    \item Spot is usually ahead.
    \item Spot behind the wing? CLIMB.
    \item You’re slowing up in the turn.
\end{itemize}

POWER OFF 180
\begin{itemize}
    \item Traffic pattern, runway 1/3 up wing.
    \item Cut power abeam numbers.
    \item 92 KIAS - GUMPFSS
    \item Rounded short approach
    \item GEAR DOWN when field is made
    \item HAND ON THROTTLE
    \item Touch down on 1000 foot markers
    \item GO AROUND if it’s not assured.
    \item FLAPS if needed.
\end{itemize}

