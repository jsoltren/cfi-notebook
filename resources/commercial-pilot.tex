\chapter{The Commercial Pilot}

So you want to become a commercial pilot? What are the privileges and limitations for such a pilot? What does that even mean?

One thing is for certain: the private pilot checkride is NOT simply a ``glorified commercial pilot checkride''. The expectations - and risks - for a commercial pilot are much higher.

Think of it this way. The private pilot checkride represents your first flight with a passenger (albeit a fairly picky one at that). The commercial pilot checkride is meant to simulate your first \emph{job interview} as a pilot. It's all about professionalism, polish, and positive control.

The private pilot checkride is your change to demonstrate competence. The commercial pilot checkride demonstrates fluency, professionalism, experience, and finesse. Of course, the ATP checkride takes this to another level, but we'll talk about that another day.

The PHAK \cite{PHAK} is a great reference. So is the AFH \cite{AFH}. It's good for commercial pilots to be familiar with hazardous attitudes \cite{adm} as well.

\section{Training}

The Commercial Pilot is held to a higher standard than the Private Pilot. Many of the maneuvers on the private pilot checkride make an appearance on the commercial pilot checkride as well. Quantitatively, the commercial pilot has less margin: tighter limits for airspeed, altitde, heading, bank, landing distance, etc. Qualitatively, the checkride examiner wants to see that the commercial pilot candidate is the clear master of the aircraft. They are looking for positive control, smoothness, rudder coordination (and cross-coordination when appropriate!), and appropriate use of trim or checklists.

With the private pilot we ask, would I trust the candidate to take my spouse or child flying? With the commercial pilot we ask, would I trust the candidate to take my entire family flying? Are they the clear master of the aircraft? Can they explain the aircraft's, as Doug De Muro would say, quirks and features? Do they make the airplane do what they want it to do, when they want it, for reasons they can clearly explain?

For the author, the process of commercial pilot training was humbling. I already had about 100 hours in my airplane, was already instrument rated, and felt like a safe, competent pilot. How hard could it possibly be? I thought I could knock out the training in about five hours of dual instruction (in addition to all the other requirements in 14 CFR 61.129 of course). I thought wrong.

Commercial pilot training forced me to be a student pilot again. It was frustrating. It was humbling. I was practicing many of the same things that I had already demonstrated as a private pilot, but in a larger, faster airplane that was not a Cessna 172. (The author feels that there is an implicit bias toward the Cessna 172 in pilot training materials.) I was being forced to look outside - after learning to fly as an instrument pilot. I was being asked to put G forces on my body and airplane that were not necessarily comfortable, and would be extremely concerning in an instrument environment. Not only was I having to re-learn things that I had previously learned in a different airplane with a different instructor, but I was also being held to a higher standard. That standard, of course, is the Commercial Pilot ACS \cite{acs-commercial}.

The ``good news'', at least for me, was that I was not immediately dependent on passing my checkride to gain any new privileges. I didn't have an airline job lined up. I could still fly myself and my family around if I wanted to. After some initial frustration, I learned to re-embrace the learning process, and that it's okay to postpone a checkride once, twice, or even more times. Cancellations are free, but disqualifications will cost you.

If I were to go back in time, and talk to myself when beginning, I would offer the following pieces of advice:

\begin{itemize}
    \item This is going to take longer than you expect.
    \item This is VFR training with an instructor. Schedules, airplane maintenance, and weather all need to align. Often times they won't.
    \item For every hour in the air, spend three hours on the ground, split evenly between debriefing the previous flight and preparing for the next flight.
    \item Try to remember to have fun with this! Flight training is an incredible privilege that not everyone is able to do.
    \item Just because examiners book a month out doesn't mean you need to book immediately. A concrete date can be a good thing (motivation) and a bad thing (an unnecessary stressor).
    \item The difficulty of commercial pilot training is a direct function of competency at the end of the private pilot checkride. If soft and short field operations, steep turns, and emergency descents remained regular operations, things will go easier. If, however, it has been two years since the words ``steep turns'' came out of your mouth, this might take a little while.
\end{itemize}

Ultimately, my own goals include being a flight instructor (for which the commercial rating is a prerequisite) and being competent in my airplane (commercial training certainly explores the flight envelope).

\section{Ground Operations}

TODO. Spend some time talking here about what makes a commercial pilot more senior than a private pilot on the ground. Do so in ACS order. Talk about airworthiness requirements, weather information, cross country flight planning, the National Airspace System, Performance and Limitations, Operation of Systems, Human Factors, Preflight Assessment, Flight Deck Management. This ends up reading like a checkride cheatsheet.

\subsection{Engine Starting}

At the commercial pilot we kick everything up a notch, even something as humdrum as starting the engine. Surely we've started an ending before?! But now, we should be able to talk a little more about what's going on.

TODO: talk about how in starting the engine, the prime makes the fuel mixture rich, and cranking makes the mixture progressively leaner until we catch.

TODO: talk about gear driven starters and how to not damage them.

TODO: discuss whether or not to leave the alternator on while starting. Depends on POH. Belt vs gear driven alternators.

TODO: discuss importance of ground lean.

\subsection{Taxiing}

On the checkride, make sure to use an appropriately slow taxi speed. Rushing on the ground will only make the checkride end faster by getting you disqualified. 10 knots is a good ground speed. Feet off the brakes and minimal RPM for this. Don't forget crosswind controls.

\subsection{Before Takeoff Check}

Follow the checklist.

What would you do if the engine ran rough?

\section{Flight Operations}

The Commercial Pilot ACS \cite{acs-commercial} is the bible for the Commercial Pilot checkride. We refer to it frequently in this section.

The author has, as of this writing, spent the most time in the P32T type, a T-tailed Piper Lance II. The detailed procedures here implicitly refer to the Lance, and its airspeeds and quirks. Some details will differ based upon the particular aircraft.

For consistency, I will explore these topics in the same order in which they appear in the Commercial Pilot ACS in the subsections that follow.

\subsection{Traffic Patterns}

By now, traffic patterns should be old hat. But there are still some improvements I made along the way.

I sometimes had a tendency to fly tight traffic patterns. While this is fine for a long succession of take offs and landings for currency, it's not ideal and reinforces bad habits. The traffic pattern should not be a racetrack. (Well, it should for the power off 180, but that's a topic for later.) It should have a clear crosswind, downwind, base, and final leg. The crosswind leg should be long enough so that the runway is off the Lance's wing tip, and so that the four legs of the traffic pattern hane a distinct ground track (since it's so easy to check with ForeFlight or FlightAware now).

A wider traffic pattern gives us more time to spot other aircraft and more time to get set up for short and soft field landings. The Lance has more energy than a 172 and doesn't want to slow down right away so we need to plan that out some.

The turn to base should be with the runway threshold 45 degrees behind.

Pay close attention to the wind. In the pattern, is the wind blowing you toward or away from the runway? Crosswind and base should have the same ground track distance. But, one will have a longer duration on account of winds.

Pay close attention to an RP on the sectional chart for a right hand traffic pattern. For the checkride, scope out potential right patterns ahead of time. Especially if much time has been spent doing left handed takeoffs, landings, and power off 180s, a right pattern could really cause a problem if they are unfamiliar!

\subsection{Normal Takeoff and Climb}

\subsection{Lazy Eights}

A lazy eight is a combination of a 180 degree turn, done at the same time as a shallow climb and descent. We turn one way, then turn the other. Its nearest living relative is the private pilot maneuver ``S-turns across a road''. The aerobatic maneuver ``wingover'' is a cousin.

The standard literature describes this as a ``graceful'' maneuver. It is supposed to be ``beautiful''. It is supposed to be ``ballet in the air''. It is supposed to demonstrate ``mastery of the aircraft'' or somesuch.

Such subjective descriptions are of zero use to the poor pilot who attempts this maneuver. We need some more rigor to derive and describe this maneuver.

The lazy eight is NOT a chandelle. It is a slow maneuver. The standard literature describes it as ``lazy''. Again, that's subjective. How slow? About 40 seconds for a Cessna 152, and about 60 seconds for the Lance. This largely comes down to how many knots of airspeed separate $V_A$ and $V_S0$ on your aircraft. I suspect it has to do with the square of that since we're talking about energy that needs to be dissipated. I'll get back to that.

Before we go too much further, let's at least talk through the conventional way of teaching this maneuver. It's not a bad start, and for many student pilots more skilled at the controls than the author, it's often enough.

First, let's consider a skateboarder shredding a half pipe. If not familiar, have a look at this video: \url{https://www.youtube.com/watch?v=FJfqcpnH6Ys}. The half-pipe in this case is a U shaped channel, about twice as tall as the skateboarder and about 30 times as long as the skateboard. The skateboarder starts at the top of the pipe. They skate down the ramp, raching maximum speed at the bottom. They climb up the other side, going about as high as the top. Once they reach the top, they turn and go back down. As they go up the pipe, they convert kinetic for potential energy, slowing down as they climb. The minimum speed is at the top of the pipe, where they turn around and go back down.

Usually, to stay on the pipe, the skateboarder makes all their turns in the same direction - to the left or to the right. But, if the pipe were sufficiently long, the skateboarder could do one to the left, then one to the right, continuing on indefinitely down the half-pipe.

One thing. The half-pipe happens pretty quickly. What if the pipe were more shallow? Well, then we could slow it down, and take our time turning around at the top.

A lazy eight is similar to shredding half-pipes (in alternating directions) in the sky with the airplane, except we make it take longer to demonstrate that we can control the aircraft.

In the airplane, we start out by flying straight and level. We pitch up and start turning. The plane climbs. As the plane climbs, it slows down, and our rate of turn naturally increases. At some point about halfway through, we've reached our minimum speed. The plane noses over and starts descending. We end up flying straight and level in the opposite direction.

Seems easy enough. What's so tricky about it? A couple of things. If we do the maneuver too quickly, we risk stalling or spinning the aircraft. If we do it too slowly, it's not really a maneuver, so much as just flying the airplane. So we pick a time scale somewhere in the middle.

Another tricky thing is that it's easy to get disoriented. For this we need good visual references. Not terribly surprising since the commercial pilot certificate is a visual certificate. I'll come back to this.

Another challenge is consistency. We need to impose some order on the maneuver so we can reproduce it consistently. I group being on the correct airspeed and heading under consistency.

The final challenge for now is airspeed management and airframe protection. Too fast and we could rip the wings off the airplane. Too slow and we could stall and spin.

Once you put all those constraints together, you arrive at the following procedure, which is usually how lazy eights are taught.

The pilot selects a clear visual reference line on the ground. A major highway, river, railroad, right of way for a power line, or other sufficiently long, straight feature is suitable for this maneuver. The keys are that the reference line be visible from all angles, and be distinct enough from other things on the ground to avoid confusion, disorientation, and ultimately, maneuver failure. A short road is not great. A small road among many other small roads is not great. The edge of a field is not great. If necessary, scope this out ahead of time using a sectional chart, Google Earth, or any other available resource.

The pilot plans to enter the maneuver flying perpendicular to the ground reference line. To protect the airframe, we want to be flying no faster than $V_A$, the design maneuvering speed for the aircraft. We likely want to be flying slower since we're probably not at gross weight. In the Lance, we enter at 120 knots indicated.

While flying straight at the reference line, but before crossing it, the pilot selects some key reference points. One is at 45 degrees off the nose in the direction of the turn. This point will represent the one-quarter point through the turn. One is at 90 degrees. More than likely this will simply be the reference line itself. Alignment with the reference line will represent the halfway point through the turn. The last reference point is at 135 degrees, or 45 degrees behind the reference line. That point represents the three-quarter point through the turn. Since the turn in one direction will immediately be followed by a turn in the other direction, it would be a good idea to get 45 and 135 degree references on both sides. Try to pick something obvious: a quarry, a silo, a castle, a lake, a river bend. Clouds aren't the worst choice but they do move. A particular field surrounded by other similar fields is probably not a great choice.

We break the maneuver down into quarters. We begin execution once we cross the reference line. This is a constant power maneuver, so don't touch the throttle - 18" of manifold pressure and 2400 RPM in the Lance on a standard day at 4500' MSL is about perfect. We shouldn't need to touch the trim either. Let's assume calm winds for now. Here we go!

In the first quarter, we are, at the same time, entering a shallow turn of about 5 degrees, and beginning to increase our pitch. Control inputs smoothly increase. At the 45 degree point, we are at 10 degrees of pitch, representing maximum pitch, and at about 15 degrees of bank, representing half of our maximum bank. We've gained about 200 feet.

In the second quarter, we are starting to let the nose down while increasing bank to the maximum. Just before the 90 degree point - let's say ten degrees prior - an interesting thing starts to happen with the nose. We are approaching maximum turn rate and minimum speed. Our lift vector is no longer pointing straight up. The nose wants to nose over and slice through the horizon. Let it!

At the half way point, our airspeed is about 10 knots above stall speed, 80 knots in the Lance. Our pitch attitude is level since we've already let the nose start falling. Our bank momentarily reaches maximum bank of 30 degrees. We're thinking about \emph{letting} the nose down and releasing the controls. We've gained about 200 more feet.

In the third quarter, we're rolling out from 30 degrees of bank and pitching down. Close to the three quarter point, we find that the Lance needs some nose-down help from the yoke, so we provide that. The airplane is accelerating and descending at this point. We need to manage that. At the three quarter point, we're back at 15 degrees of bank and whatever pitch is appropriate for our other goals - probably 5 or so degrees down.

In the fourth quarter, we're taking our time to make sure that we roll out on airspeed, and on heading.

We finish the maneuver at the same airspeed and altitude as our entry, but in the opposite direction, over the same reference line. Once we're there, we start a turn in the opposite direction. We can keep doing this indefinitely.

The hypothetical perfect VFR pilot can conduct this maneuver exclusively by looking at the airplane, so they say.

Simple enough, huh?

Well, the author didn't think so. Here are a few more tricks that the author found helpful.

First, watch this video: \url{https://www.youtube.com/watch?v=3Oxbr1PuoSQ}.

The video raises some important points.

First is that the lazy eight is an exercise in managing the overbanking tendencies of the aircraft. Past a few degrees of input, the airplane wants to keep turning. Or, as I see it, we tickle an unstable mode of the aircraft with this maneuver.

Second is the incredible importance of coordination. Sure, skidding or slipping is disqualifying on a checkride, but that's not the important or interesting part. The airplane has a left turning tendency, which we'll need to fight with right rudder. The Lance has a decent amount of adverse yaw that ends up helping us some here.

Third is an important reminder that this is a visual reference maneuver. In the video, the instructor has the panel covered with a sheet of paper. All he does is take a couple of quick peeks at the instruments to confirm altitude and airspeed.

Last is that we can complete the maneuver with fairly minimul control inputs. Well, at least, on a Cessna. But that's not true on the Lance. As my Navy test pilot friend points out, the Lance has \emph{extreme} roll stability, so we have to help it along some.

In addition to the video, I'll add my own observations.

Thinking about the timing of the maneuver helped me to quantify what it means to ``slow down''. As someone who regularly deals with events that happen in the span of about 5 \emph{milli}seconds at work, fifteen seconds feels like an eternity! But quantifying that helped tremendously. Each quarter should take about 20-30 seconds, and a complete turn should take about 90 seconds.

In particular, the first quarter will take a really long time. We use this quarter to try to gain some separation from our starting point. It helps us to fly a wider turn which gives us more options in the later segments. The second quarter sees the most control inputs. In the third quarter, it's important to get out of that 30 degrees of bank quickly or else the turn will be over too soon.

I also thought more carefully about what the aircraft was doing in each of the yaw, pitch, and roll axes individually, and what the aircraft's stability was doing for us.

In yaw: the airplane is turning 180 degrees. The rate of yaw increases through the first half and decreases through the second half. It's probably close to a perfect cosine. The Lance is lightly stable in yaw and doesn't need much help here.

In pitch: We're nosing up, than descending. Yoke back, let the nose fall over, yoke forward. Since we didn't touch power or trim, the airplane \emph{wants} to find its original airspeed and altitude. Zoom up, zoom down.

In roll: We start the plane rolling. We give it a bank angle and maybe a touch of momentum. The airplane wants to keep increasing the bank to some extent. But these things are happening at time constants that end up being faster than our maneuver. So we baby the airplane into, and out of, a gentle bank, not exceeding 30 degrees.

We haven't talked about crosswinds yet. Strictly speaking, this is not a ground reference maneuver. But if we don't correct for winds, we don't stay on our line. Instead, we start drifting. So, through the maneuver, just like in a traffic pattern, we can think about gaming the turn. If we're getting blown away from our reference line, maybe we want to hurry up the first quarter and start our turn sooner. If we're getting blown toward it, maybe we want to slow down through the first quarter and slow down the last quarter.

This maneuver comes back on the CFI checkride, where we might need to teach it. So approaching it with rigor up front pays dividends later.

For anyone who picked up this maneuver more easily, without all this rigor: I envy you.

If you're still stuck, have a look at \url{https://www.av8n.com/how/htm/maneuver.html#sec-lazy-eight}. I don't degree with teaching the attitude-centric or nose-centric view of that explanation, but it provides an interesting cross-check for what I've described above.

Anyway. All the quantification aside, this is supposed to be a \emph{smooth} maneuver. We're not fighting the airplane on the controls or micro-managing it. We're making small, subtle control inputs.

\subsection{Soft-Field Takeoff and Climb}

The soft field takeoff and climb demonstrates that the pilot is able to operate the aircraft safely on a dirt, grass, or other similar off-pavement airstrip. The constant theme is unweighting: keeping weight off of the tires, particularly the nosewheel. A secondary theme is smoothness.

We configure the aircraft with two notches of flaps and one or two turns of nose-up trim past the neutral point. Once we begin moving the aircraft, we continue to move until we are airborne or abort the landing. Back pressure on the yoke will transfer as much weight as possible from the nose wheel to the mains. The Lance has very little elevator authority at low speeds so this effect will be minimized.

Taxi smoothly and continuously onto the runway. Apply takeoff power, at perhaps half the rate as usual (taking 6-8 seconds to advance the throttles instead of 3-4) to help prevent kicking up debris. Holding firm backpressure, accelerate down the runway.

The Lance's $V_x$ speed with gear down is only 68 knots. The Lance usually doesn't climb until 70 knots. So, in this aircraft, it may not be necessary to accelerate in ground effect, as we might with many other aircraft. The pilot must be mindful of this.

Once we are airborne at rotation speed above $V_x$, we hold that to clear a 50- or 100-foot obstacle. Then, once positive rate is confirmed and the obstacle is clear, we retract the gear, lower the flaps one notch at a time, and continue accelerate to our gear up $V_x$ of 87 knots, which just so happens to be $V_y$ with gear down and flaps up. After that, we continue to accelerate to a clean $V_y$ of 92 knots, and continue to climb from there.

Disqualifications might include: porpoising on the ground due to incorrect transition to ground effect flight (which we shouldn't need in the Lance), stopping once we begin the ground roll, inappropriate use of controls, incorrect airspeeds.

\subsection{Soft-Field Approach and Landing}

The soft field landing demonstrates that the pilot is able to manage the aircraft on a surface other than a paved runway. Much like the soft field take off, the goal of the maneuver is to keep as much weight off of the tires, particularly the nose tire, for as long as possible. Note that, by default, the soft field approach is NOT also a short field approach. However, the Lance's POH does not differentiate between them.

The landing begins as a typical short landing: wheels down at 75 knots, full flaps. In the Lance, a blip of throttle is needed to allow the main tires to gently set down on the ground as opposed to slamming down and potentially digging in. Then, we continue to hold back pressure and leave the flaps down to gently bring the nose down, doing so as late as able. We can lower the flaps to two notches as soon as able, and have the option of leaving them down as we taxi. We must not stop until we are clear of the active.

In the Lance, we typically do these maneuvers with half a tank of fuel and two front seat occupants pushing the station weight limit of 440 lbs. This leads to a fairly fore CG condition where the aircraft is nose heavy. When the aircraft is aft loaded, the nose becomes MUCH lighter, and it is possible to wheelie all the way down the runway unless we are very careful.

Note that the soft field landing has no aiming point specified. This is good, since a true grass strip won't have markings.

Disqualifications might include: bouncing, slamming the nose wheel, stopping on the runway.

\subsection{Short-Field Takeoff and Maximum Performance Climb}

In the short field takeoff, we are looking to use a minimum of runway, lifting off at the earliest spot possible, and to clear a 50 foot obstacle. To use the minimum runway, we position the aircraft as close to the very end of the runway as possible, using a displaced threshold to our advantage. In the Lance we use two notches of flaps. We hold the brakes, apply full power, lean if appropriate, and release the brakes. Seriously - feet off the brakes!

We recall that $V_x$ in the Lance is a mere 68 knots in the dirty configuration. We're lucky if we are airborne at 70-75. Maintain that airspeed until clear. Then it's gear up and flaps up as we accelerate through $V_x$ of 87 knots to $V_y$ of 92 knots.

Disqualifications might include: forgetting to position the airplane as far as possible on the end of the runway, incorrect flap settings, forgetting to run up with brakes held, incorrect airspeeds.

\subsection{Short-Field Approach and Landing}

The Lance doesn't like to be slow as this maneuver reminds us.

The short field approach is an important practical maneuver. On a checkride, we usually have the examiner give us an aiming point, and it's usually the 1000 foot markers. In real life, that aiming point is the numbers as we're looking to truly use minimal runway.

In the Lance, we set up for a full flaps stabilized approach diung 75 knots over the numbers. The plane will want to sink quickly when we cut power so a small dose of throttle will set us up for success. However, it's imporant to be within 100 feet of the required spot, so a hard landing is better than a failed checkride. Recalling that the 1000 foot markers are 200 feet ling (source?), the goal is to be ON the markers.

Once down, retract flaps (easy in the Lance, lower the lever to the ground), apply full back pressure, and call out ``maximum braking''. It's not necessary to actually brake hard, we don't need to leave tire marks on the runway.

Disqualifications - and there are many for this one - might include: improper configuration, missing the aiming point, bouncing so hard that we porpoise and have to go around, forget to call out ``maximum braking''.

\subsection{Power-Off 180$^\circ$ Accuracy Approach and Landing}

This maneuver is probably responsible for more failed commercial pilot and CFI checkrides than all the others combined. It is unforgiving and requires a very high degree of precision, particularly to correct for winds. The candidate gets one shot at this one. If they land short, or land long, or have to go around, that's a disqualification. It still makes sense to continue the checkride, but if this happens to you, expect to lick your wounds, spend an hour with a CFI for an additional signoff, and to take another shot in the coming week.

For this maneuver, we fly a traffic pattern with a closer than usual downwind leg ($\frac1 3$ of the way up the Lance's wing instead of just off the tip) and cut the power abeam our touchdown point, the 1000 foot markers. We immediately pitch for our best glide speed of 92 knots, turn a fairly tight pattern, put the gear down once the field is made, and execute a normal, possibly no flaps landing. This ends up being a fiarly aggressive, tight, rounded approach, with no real base leg, as downwind transitions smoothly to base and final.

DO NOT FORGET TO PUT THE GEAR DOWN. If you're less than 50' AGL and don't see three greens, GO AROUND. This isn't worth a new engine.

With respect to the aiming point, there are three possible outcomes:

\begin{itemize}
\item{Too short.} We've lost too much energy and have no hope of making the aiming point. Perhaps our pattern was too wide, we did not correct for wind, or we otherwise failed to manage energy. Identify and call out the situation, and immediately GO AROUND. Whether or not this is disqualifying, this is the correct course of action.
\item{On point.} This is ideal. But getting it magically right by chance means we're not really in control of the aircraft. So, even though this will pass the test, it's not the best place to be.
\item{Too far.} We have not lost enough energy. As long as we don't have \emph{such} an excess of energy that we cannot manage the airplane and the landing, this is fine. We have some tricks to get lower: cut the airspeed (down to 75 instead of 87-92), lower the flaps, slip with flaps extended (approved in the Lance!).
\end{itemize}

You've got 200 feet from the aiming point for this one. If the aiming point is the near side of the 1000 foot markers, the limit will be the end of them. You'll know immediately when you've done this one correctly, and so will your examiner.

Assuming you do this correctly, this is arguably the hardest part of the checkride. So spend plenty of time practicing.

Setup is absolutely critical for this one. If the setup isn't right, it's technically not a go around if you do a 360 in the pattern for spacing. The examiner may frown on this. Make sure to allow plenty of time for a complete, stabilized downwind leg. Also, we don't want to be \emph{too} fast downwind. Best glide is 92. Being at about 105 is right. If we're at 130, that's all that much more extra energy that we need to manage.

Even the Lance will float a tiny bit in ground effect, so we've started aiming for two lines prior, flaring and floating to the marks.

In the Lance, the bank angles and sink rates may seem excessive. I've had at least one safety pilot / right seat CFI get nervous. So, with an examiner, make certain to brief this.

Disqualifications include: failing to make the aiming point (obviously), failing to go around when appropriate.

Slamming the plane on the ground on the marks isn't great. Being off center line can be disqualifying. Failing to make a crosswind correction can be a huge problem. If the wind is blowing you toward the runway, consider a wider pattern. If there is a strong wind blowing away from the runway, consider an even tighter pattern. We've practiced these almost exclusively to the left, which is preferred with the airplane's left turning tendency (partially caused by even a windmilling propeller).

Instructors and examiners get really nervous if you take your hand off the throttle for this one. Keep the hand on the throttle, ready to go around.

\subsection{Slow Flight and Stalls}

Nothing too special about these except the very tight commercial pilot ACS limits: $\pm50$ feet for altitude, $\pm10^\circ$ heading, $+5/-0$ for airspeed (which in the Lance is usually 70), and $\pm5^\circ$ angle of bank. The examiner may want to see these in the clean configuration, or an approach configuration, which could be one notch of flaps, two notches of flaps and gear down, full flaps and gear down. The may ask you to fly straight and level, to turn (usually 15 degrees is about right), to climb or descend, or to do more than one at the same time. If we're doing these power off, of course we cannot maintain altitude, so allow plenty of margin for descending just above stall speed. If we're doing these power on, correct rudder application is paramount.

Remember: pitch (and quite a lot of nose up trim in the Lance) for airspeed, power for altitude.

Give plenty of space above the ground for these. KMDD airport sits at 2805.4 feet MSL. We could be entering these maneuvers at 5500-6000 feet MSL.

For the commercial pilot, stalls may be to first indication (BEEEP!) or a full stall (BEEEP!, heavy buffet, and a very tiny nose drop and excessive sink rate in the Lance).

Disqualifications include: getting below 1,500 AGL. Make that 2000 AGL. Stall recovery can be up to 550 feet in the Lance. Call that 3000 AGL. Forgetting to make clearing turns is disqualifying. Forgetting to stay coordinated could be as well.

\subsection{Accelerated Stalls}

Fortunately, these are harder to enter than they are to exit. Also fortunately, we don't have an altitude requirement for this one. Well, except for one: DO NOT get below 3000 feet AGL. If we're especially ham fisted, we'll enter a stall, which - shocker! - is disqualifying.

Set up the airplane for $V_A$, configure as requested, set power appropriately, and enter a coordinated 45 degree banking turn. Don't worry about altitude as long as we're high enough.

Now: PULL. It will take a HEAVY two handed pull in the Lance to get a stall warning.

Hear the BEEEP? Wings level and go around. Done.

\section{Maneuver Checklists}

GUMPFSS Landing Check used on maneuvers in this section:
\begin{itemize}
    \item G - Gas: On the most full tank. Switch on fuel pump before changing tanks.
    \item U - Undercarraige: Extend if below 129 KIAS on the Lance. Leave up for upper air maneuvers or power off 180s.
    \item M - Mixture: Full rich. Lean the mixture if full rich causes a marked reduction in power. Leave alone for constant power upper air maneuvers.
    \item P - Propeller. Leave alone for upper air work, push forward for landings.
    \item F - Flaps. As needed.
    \item S - Seatbelts and Shoulder Harnesses. Must be fastened.
    \item S - Switches. Landing light as needed. Fuel pump: switch ON.

\end{itemize}

STEEP TURNS
\begin{itemize}
    \item CLEAR THE AREA + GUMPFSS
    \item Pick reference point on horizon.
    \item Set heading indicator.
    \item Roll right to 50
    \item Power comes in
    \item Substantial back pressure – both hands.
    \item Keep reference point on horizon. Reference point is oil port on top cowling left of nose.
    \item In left turns the nose is above the horizon.
    \item In right turns nose is below horizon.
    \item Lead roll out of turn by 25 degrees.
    \item Our standard is +-50 feet altitude.
\end{itemize}

CHANDELLE
\begin{itemize}
    \item CLEAR THE AREA + GUMPFSS
    \item 18”/2400 RPM or whatever speed is needed for 120 KIAS
    \item Pick a reference point and two side reference points.
    \item Roll to 30 degrees, power comes in.
    \item Pitch right up to 12-15 degrees nose up.
    \item Substantial back pressure – both hands.
    \item At the 45: Call out: On Pitch – On Bank
    \item 45-90: Back pressure continues to increase.
    \item 90: Reduce roll to 20. Keep the back pressure. Maneuver is about a third done.
    \item 90-135: make sure turn rate and loss of speed are not excessive.
    \item 135-180: game it so you’re right at 70 KIAS at 180 degrees.
    \item 180 degrees: BEEEEP on heading. Level off.
    \item Watch the nose. Don’t balloon or sink. Forward pressure as we accelerate.
\end{itemize}

ENGINE FAILURE
\begin{itemize}
    \item A – Airfield. Probably under you or close to it.
    \item B – Best Glide.
    \item C – Checklist. Touch anything related to fuel, oil, mixture, or outside air. GUMPFSS check.
    \item D – Declare emergency 121.5 MAYDAY x 3
    \item E - Execute.
\end{itemize}

POWER OFF SPIRAL aka STEEP SPIRAL (not an ``Emergency Descent'').
\begin{itemize}
    \item CLEAR THE AREA if able + GUMPFSS
    \item Pick a point below. Pick a heading reference.
    \item Heading bug.
    \item Straight to 92 KIAS clean. NOT an emergency descent. Trim for it.
    \item Where is wind blowing? Clear engine into it.
    \item Vary bank to keep point where it needs to be. No more than 60 degrees.
    \item Start high up enough for three turns.
    \item 1500 AGL plus 1200/turn is 5100 AGL MINIMUM.
    \item Smooth recovery after three turns.
\end{itemize}

LAZY EIGHTS
\begin{itemize}
    \item CLEAR THE AREA + GUMPFSS
    \item 18”/2400 RPM or whatever speed is needed for 120 KIAS (sometimes 19 works better")
    \item Pick a road.
    \item Slow slow slow.
    \item Roll to 15 degrees.
    \item Pitch up.
    \item Don’t touch the power.
    \item 0-45: pitching up. Slow slow slow.
    \item 45: Call out. ``Maximum pitch''.
    \item 45-80 work up to max pitch and max bank. You’ll gain 400-600 feet.
    \item 80-100 nose slices through horizon.
    \item 90: Call out. ``Maximum bank''.
    \item 100-135 baby the nose down, back to 120 KIAS.
    \item 135-180 keep pushing that nose down.
    \item Finish on altitude and on heading.
    \item Do it again.
\end{itemize}

EIGHTS ON PYLONS
\begin{itemize}
    \item CLEAR THE AREA + GUMPFSS
    \item 18”/2400 RPM or whatever speed is needed for 120 KIAS
    \item Pivotal altitude is about 800-1200 AGL.
    \item Pick points a mile apart. Wind blowing between them. Enter on a downwind and turn left first.
    \item Think about wind correction. Xs should be symmetrical.
    \item Don’t touch power.
    \item Spot ahead of the wing? DIVE.
    \item Spot is usually ahead.
    \item Spot behind the wing? CLIMB.
    \item You’re slowing up in the turn.
\end{itemize}

POWER OFF 180
\begin{itemize}
    \item Traffic pattern, runway 1/3 up wing.
    \item Cut power abeam numbers.
    \item 92 KIAS - GUMPFSS
    \item Rounded short approach
    \item GEAR DOWN when field is made
    \item HAND ON THROTTLE
    \item Touch down on 1000 foot markers
    \item GO AROUND if it’s not assured.
    \item FLAPS if needed.
\end{itemize}

