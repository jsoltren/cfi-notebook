\chapter{The Commercial Pilot}

So you want to become a commercial pilot? What are the privileges and limitations for such a pilot? What does that even mean?

One thing is for certain: the private pilot checkride is NOT simply a ``glorified commercial pilot checkride''. The expectations - and risks - for a commercial pilot are much higher.

Think of it this way. The private pilot checkride represents your first flight with a passenger (albeit a fairly picky one at that). The commercial pilot checkride is meant to simulate your first \emph{job interview} as a pilot. It's all about professionalism, polish, and positive control.

The private pilot checkride is your change to demonstrate competence. The commercial pilot checkride demonstrates fluency, professionalism, experience, and finesse. Of course, the ATP checkride takes this to another level, but we'll talk about that another day.

The PHAK \cite{PHAK} is a great reference. So is the AFH \cite{AFH}. It's good for commercial pilots to be familiar with hazardous attitudes \cite{adm} as well.

\section{Maneuvers}

The Commercial Pilot is held to a higher standard than the Private Pilot. Many of the maneuvers on the private pilot checkride make an appearance on the commercial pilot checkride as well. Quantitatively, the commercial pilot has less margin: tighter limits for airspeed, altitde, heading, bank, landing distance, etc. Qualitatively, the checkride examiner wants to see that the commercial pilot candidate is the clear master of the aircraft. They are looking for positive control, smoothness, rudder coordination (and cross-coordination when appropriate!), and appropriate use of trim or checklists.

With the private pilot we ask, would I trust the candidate to take my spouse or child flying? With the commercial pilot we ask, would I trust the candidate to take my entire family flying?

The Commercial Pilot ACS \cite{acs-commercial} is the bible for the Commercial Pilot checkride. We refer to it frequently in this section.

The author has, as of this writing, spent the most time in the P32T type, a T-tailed Piper Lance II. The detailed procedures here implicitly refer to the Lance, and its airspeeds and quirks. Some details will differ based upon the particular aircraft.

\subsection{Soft-Field Takeoff and Climb}

The soft field takeoff and climb demonstrates that the pilot is able to operate the aircraft safely on a dirt, grass, or other similar off-pavement airstrip. The constant theme is unweighting: keeping weight off of the tires, particularly the nosewheel. A secondary theme is smoothness.

We configure the aircraft with two notches of flaps and one or two turns of nose-up trim past the neutral point. Once we begin moving the aircraft, we continue to move until we are airborne or abort the landing. Back pressure on the yoke will transfer as much weight as possible from the nose wheel to the mains. The Lance has very little elevator authority at low speeds so this effect will be minimized.

Taxi smoothly and continuously onto the runway. Apply takeoff power, at perhaps half the rate as usual (taking 6-8 seconds to advance the throttles instead of 3-4) to help prevent kicking up debris. Holding firm backpressure, accelerate down the runway.

The Lance's $V_x$ speed with gear down is only 68 knots. The Lance usually doesn't climb until 70 knots. So, in this aircraft, it may not be necessary to accelerate in ground effect, as we might with many other aircraft. The pilot must be mindful of this.

Once we are airborne at rotation speed above $V_x$, we hold that to clear a 50- or 100-foot obstacle. Then, once positive rate is confirmed and the obstacle is clear, we retract the gear, lower the flaps one notch at a time, and continue accelerate to our gear up $V_x$ of 87 knots, which just so happens to be $V_y$ with gear down and flaps up. After that, we continue to accelerate to a clean $V_y$ of 92 knots, and continue to climb from there.

Disqualifications might include: porpoising on the ground due to incorrect transition to ground effect flight (which we shouldn't need in the Lance), stopping once we begin the ground roll, inappropriate use of controls, incorrect airspeeds.

\subsection{Soft-Field Approach and Landing}

The soft field landing demonstrates that the pilot is able to manage the aircraft on a surface other than a paved runway. Much like the soft field take off, the goal of the maneuver is to keep as much weight off of the tires, particularly the nose tire, for as long as possible. Note that, by default, the soft field approach is NOT also a short field approach. However, the Lance's POH does not differentiate between them.

The landing begins as a typical short landing: wheels down at 75 knots, full flaps. In the Lance, a blip of throttle is needed to allow the main tires to gently set down on the ground as opposed to slamming down and potentially digging in. Then, we continue to hold back pressure and leave the flaps down to gently bring the nose down, doing so as late as able. We can lower the flaps to two notches as soon as able, and have the option of leaving them down as we taxi. We must not stop until we are clear of the active.

In the Lance, we typically do these maneuvers with half a tank of fuel and two front seat occupants pushing the station weight limit of 440 lbs. This leads to a fairly fore CG condition where the aircraft is nose heavy. When the aircraft is aft loaded, the nose becomes MUCH lighter, and it is possible to wheelie all the way down the runway unless we are very careful.

Note that the soft field landing has no aiming point specified. This is good, since a true grass strip won't have markings.

Disqualifications might include: bouncing, slamming the nose wheel, stopping on the runway.

\subsection{Short-Field Takeoff and Maximum Performance Climb}

In the short field takeoff, we are looking to use a minimum of runway, lifting off at the earliest spot possible, and to clear a 50 foot obstacle. To use the minimum runway, we position the aircraft as close to the very end of the runway as possible, using a displaced threshold to our advantage. In the Lance we use two notches of flaps. We hold the brakes, apply full power, lean if appropriate, and release the brakes. Seriously - feet off the brakes!

We recall that $V_x$ in the Lance is a mere 68 knots in the dirty configuration. We're lucky if we are airborne at 70-75. Maintain that airspeed until clear. Then it's gear up and flaps up as we accelerate through $V_x$ of 87 knots to $V_y$ of 92 knots.

Disqualifications might include: forgetting to position the airplane as far as possible on the end of the runway, incorrect flap settings, forgetting to run up with brakes held, incorrect airspeeds.

\subsection{Short-Field Approach and Landing}

The Lance doesn't like to be slow as this maneuver reminds us.

The short field approach is an important practical maneuver. On a checkride, we usually have the examiner give us an aiming point, and it's usually the 1000 foot markers. In real life, that aiming point is the numbers as we're looking to truly use minimal runway.

In the Lance, we set up for a full flaps stabilized approach diung 75 knots over the numbers. The plane will want to sink quickly when we cut power so a small dose of throttle will set us up for success. However, it's imporant to be within 100 feet of the required spot, so a hard landing is better than a failed checkride. Recalling that the 1000 foot markers are 200 feet ling (source?), the goal is to be ON the markers.

Once down, retract flaps (easy in the Lance, lower the lever to the ground), apply full back pressure, and call out ``maximum braking''. It's not necessary to actually brake hard, we don't need to leave tire marks on the runway.

Disqualifications - and there are many for this one - might include: improper configuration, missing the aiming point, bouncing so hard that we porpoise and have to go around, forget to call out ``maximum braking''.

\subsection{Power-Off $180^\circ$ Accuracy Approach and Landing}

This maneuver is probably responsible for more failed commercial pilot and CFI checkrides than all the others combined. It is unforgiving and requires a very high degree of precision, particularly to correct for winds. The candidate gets one shot at this one. If they land short, or land long, or have to go around, that's a disqualification. It still makes sense to continue the checkride, but if this happens to you, expect to lick your wounds, spend an hour with a CFI for an additional signoff, and to take another shot in the coming week.

For this maneuver, we fly a traffic pattern with a closer than usual downwind leg ($\frac1 3$ of the way up the Lance's wing instead of just off the tip) and cut the power abeam our touchdown point, the 1000 foot markers. We immediately pitch for our best glide speed of 92 knots, turn a fairly tight pattern, put the gear down once the field is made, and execute a normal, possibly no flaps landing. This ends up being a fiarly aggressive, tight, rounded approach, with no real base leg, as downwind transitions smoothly to base and final.

DO NOT FORGET TO PUT THE GEAR DOWN. If you're less than 50' AGL and don't see three greens, GO AROUND. This isn't worth a new engine.

With respect to the aiming point, there are three possible outcomes:

\begin{itemize}
\item{Too short.} We've lost too much energy and have no hope of making the aiming point. Perhaps our pattern was too wide, we did not correct for wind, or we otherwise failed to manage energy. Identify and call out the situation, and immediately GO AROUND. Whether or not this is disqualifying, this is the correct course of action.
\item{On point.} This is ideal. But getting it magically right by chance means we're not really in control of the aircraft. So, even though this will pass the test, it's not the best place to be.
\item{Too far.} We have not lost enough energy. As long as we don't have \emph{such} an excess of energy that we cannot manage the airplane and the landing, this is fine. We have some tricks to get lower: cut the airspeed (down to 75 instead of 87-92), lower the flaps, slip with flaps extended (approved in the Lance!).
\end{itemize}

You've got 200 feet from the aiming point for this one. If the aiming point is the near side of the 1000 foot markers, the limit will be the end of them. You'll know immediately when you've done this one correctly, and so will your examiner.

Assuming you do this correctly, this is arguably the hardest part of the checkride. So spend plenty of time practicing.

Disqualifications include: failing to make the aiming point (obviously), failing to go around when appropriate.

\subsection{Slow Flight and Stalls}

Nothing too special about these except the very tight commercial pilot ACS limits: $\pm50$ feet for altitude, $\pm10^\circ$ heading, $+5/-0$ for airspeed (which in the Lance is usually 70), and $\pm5^\circ$ angle of bank. The examiner may want to see these in the clean configuration, or an approach configuration, which could be one notch of flaps, two notches of flaps and gear down, full flaps and gear down. The may ask you to fly straight and level, to turn (usually 15 degrees is about right), to climb or descend, or to do more than one at the same time. If we're doing these power off, of course we cannot maintain altitude, so allow plenty of margin for descending just above stall speed. If we're doing these power on, correct rudder application is paramount.

Remember: pitch (and quite a lot of nose up trim in the Lance) for airspeed, power for altitude.

Give plenty of space above the ground for these. KMDD airport sits at 2805.4 feet MSL. We could be entering these maneuvers at 5500-6000 feet MSL.

For the commercial pilot, stalls may be to first indication (BEEEP!) or a full stall (BEEEP!, heavy buffet, and a very tiny nose drop and excessive sink rate in the Lance).

Disqualifications include: getting below 1,500 AGL. Make that 2000 AGL. Stall recovery can be up to 550 feet in the Lance. Call that 3000 AGL. Forgetting to make clearing turns is disqualifying. Forgetting to stay coordinated could be as well.

\subsection{Accelerated Stalls}

Fortunately, these are harder to enter than they are to exit. Also fortunately, we don't have an altitude requirement for this one. Well, except for one: DO NOT get below 3000 feet AGL. If we're especially ham fisted, we'll enter a stall, which - shocker! - is disqualifying.

Set up the airplane for $V_A$, configure as requested, set power appropriately, and enter a coordinated 45 degree banking turn. Don't worry about altitude as long as we're high enough.

Now: PULL. It will take a HEAVY two handed pull in the Lance to get a stall warning.

Hear the BEEEP? Wings level and go around. Done.


