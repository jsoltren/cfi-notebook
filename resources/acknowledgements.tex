\section*{Acknowledgements}

It takes a village to keep a small airplane in the air, and to teach a pilot.

Most immediately, I must call out John Grieger and Brent Kelley, my partners (as of this writing) in our shared ownership of N36262, our 1978 Piper Lance II. With their help I have learned more about what it means to be the owner and operator of an aircraft than I could have hoped to otherwise. I appreciate Martyn Atterton for selling me his share of the airplane when he decided it was time to hang up his wings.

My thanks go to the crew of phenomenal flight instructors with whom I have had the pleasure of flying throughout my aviation career. In reverse chronological order of last flight, these are: Eric Lenk, Steve Jennings, Lawrence Spinetta, Mark Kobelin, Stephen Pitts, and Tim Stingle. Each has brought their own unique perspective of flight and instruction to the table. All have helped me to shape and refine my own instructional style. With them together, I feel like I've gotten the best combination of military flight instruction, airline flight instruction, aeronautical university instruction, and home-grown Part 61 instruction that I could have possibly gotten as a low hour pilot.

I would also like to thank the fantastic crew of safety pilots who have flown with me through the years, regardless of who was in which seat. These would be Leland Freeman, David Kindley, Geoffrey Blake, Robert Curtis, David Loia, and Jeff Chern. It is somewhat refreshing to know that all pilots, regardless of their seniority, are still students and are still humans.

Due credit goes to Joe Locasto, the old man at KSQL airport who gave me the final push to get into flying in 2011. When I lamented that flying was too expensive, he responded: ``well, it's never going to be cheaper, so what are you waiting for?"

My eternal thanks go to my wife Jane, for supporting (or at least tolerating) many late nights spent putting the airplane away. When I propose a new adventure, this wonder woman never says ``no'', only, ``how?''.


