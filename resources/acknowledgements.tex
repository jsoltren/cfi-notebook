\section*{Acknowledgements}

It takes a village to keep a small airplane in the air, and to make a pilot. At some point - more quickly than you realize - that list becomes
too long to fit on a page. My apologies for anyone who feels they should be on this page and are not.

I'll start with Cole Turner. Cole had the patience to get me through my commercial checkride, and has stayed on as
a trusted mentor and friend. Like any good mentor, he continues to push me.

I must call out Don Grenier, John Grieger, Tim Stahla, and Brent Kelley, my current and former partners in our shared ownership of N36262, our 1978 Piper Lance II. With their help I have learned more about what it means to be the owner and operator of an aircraft than I could have hoped to otherwise. Our hangar flying sessions have made us all better pilots. I appreciate Martyn Atterton for selling me his share of the airplane when he decided it was time to hang up his wings.

Due credit also goes to the superstar mechanic who keeps the Lance in the air: Bob Gloris. We are also grateful for the efforts of many other mechanics and technicians who help with all manner of aircraft issues.

My thanks go to the crew of phenomenal flight instructors with whom I have had the pleasure of flying throughout my aviation career. Thanks, too, to my checkride examiners. Each has brought their own unique perspective of flight and instruction to the table. All have helped me to shape and refine my own instructional style. With them together, I feel like I've gotten the best combination of military flight instruction, airline flight instruction, aeronautical university instruction, and home-grown Part 61 instruction that I could have possibly gotten as a low hour pilot.

I would also like to thank the fantastic crew of safety pilots who have flown with me through the years, regardless of who was in which seat. It is refreshing to know that all pilots, regardless of their seniority, are still students and are still humans.

Due credit goes to Joe Locasto, the old man at KSQL airport who gave me the final push to get into flying in 2011. When I lamented that flying was too expensive, he responded: ``well, it's never going to be cheaper, so what are you waiting for?" And here we are.

Flying is more like karate than I realized. So I must also thank Hoke Nunan, my Tang Soo Do instructor, for teaching me
how to operate effectively in a public facing environment with regular evaluations. I use his parable of
``emptying the cup'' with my students in our early lessons.

My eternal thanks go to my wife Jane, for supporting (or at least tolerating) many late nights spent putting the airplane away. When I propose a new adventure, this wonder woman never says ``no'', only, ``how?''


