\chapter{Aeromedical Factors}

High altitude operations involve their own aeromedical and equipment factors and are covered in their own section.

Much of the information from this section comes from the \href{https://www.faa.gov/air_traffic/publications/atpubs/aim_html/index.html}{Aeronautical Information Manual (AIM)}, dated 2023-10-05 with Change 1. The AIM changes frequently. The AIM is not a FAR but it assists with compliance with the FARs so it is wise to follow it.

In particular, \href{https://www.faa.gov/air_traffic/publications/atpubs/aim_html/chap8_section_1.html}{AIM 8-1-1 Fitness for Flight} provides a wealth of information.

The IMSAFE checklist (see the Alphabet Soup chapter) comes straight from the AIM.

An extensive discussion of medical certificate rules is in \href{https://www.ecfr.gov/current/title-14/chapter-I/subchapter-D/part-67}{14 CFR Part 67}. The biggest differentiators are:

\begin{itemize}
\item The expiration dates, which we covered in the Federal Aviation Regulations chapter.
\item Distant Vision, needs to be 20/20 corrected for first and second class, 20/40 for third.
\item An EKG for the first class medical starting at age 35, and yearly at age 40.
\end{itemize}

The FAA maintains a safe medication list, located at \url{https://www.faa.gov/pilots/medical_certification/medications}. In particular, many over the counter medications are considered safe, but diphenhydramine, dextromethorphan, doxylamine, and any sedating antihistamine is not. Anything that induces drowsiness is a no-go.

Hypoxia is covered in the High Altitude chapter of this book.

\section{Spatial Disorientation}

Spatial disorientation is the inability to determine position or relative motion, commonly occurring during periods of challenging visibility, since vision is the dominant sense for orientation. The inner ear and its semicurcular canals are essential for spatial orientation, so any inner ear issues such as congestion, allergies, etc. will be problematic.

AIM 8-1-1 gives a detailed treatment of all of the various types of spatial disorientation. Here are nine of them.

\begin{enumerate}

\item The Leans. The pilot aligns to a horizontal reference that is not the horizon, and will lean either themselves or the aircraft to attempt to correct. This happens if entering a turn too slowly.
\item Coriolis Illusion. Abrupt head motion during a constant rate turn can create the illision of rotation about a different axis entirely. The attempt to correct this can easily put the airplane into an unusual attitude. The way to prevent this is to avoid abrupt head movements!
\item Graveyard Spin. Recovery from a spin can cause the pilot to believe that they have entered a spin in the opposite direction! Recovery will cause re-entering the spin.
\item Graveyard Spiral. A prolonged (over 20 seconds) constant rate turn can give the pilot the illusion of being level. Now, if the airplane loses altitude, the pilot will think they are in a stright dive instead of a turning dive. The pilot will instinctively pull back on the yoke. This just tightens the turn and increases the sink rate.
\item Somatogravic Illusion. A rapid acceleration (high performance aircraft) makes the pilot think they are in a nose up attitude. They push the yoke forward. If this happens on takeoff, this results in the plane hitting the ground. A rapid deceleration can do the opposite: lead the pilot to believe they are in a nose down attitude, pull back on the yoke, and enter a stall.
\item Inversion Illusion. An abrupt change from a climb to straight and level can create the sensation of falling backwards. The instinctive correction is to push the nose down, which only makes it worse.
\item Elevator Illusion. An abrupt updraft creates the sensation of being in a climb and causes the pilot to push the nose down. An abrupt downdraft causes the sensation of being in a descent and causes the pilot to pull the nose up.
\item False Horizon. At night, in IMC, or in other conditions, the pilot can align with a horizontal reference that is not the true horizon, putting the aircraft into an unusual attitude.
\item Autokinesis. If staring at a light for an extended period of time (10-12 seconds), the light will appear to move. Avoid this by keeping the scan up.
\end{enumerate}

The acronym ICE FLAGGS in the Alphabet Soup chapter covers this topic.

\section{Landing Errors}

A narrow runway makes the pilot think they are high and fly a lower approach. A wide runway makes the pilot think they are low and flare too high.

An upsloping runway or terrain creates the illusion of the plane being too high, and the pilot flies a lower approach. If downsloping, the pilot thinks they are low and flies a higher approach.

Featureless terrain can create the illusion of a ``black hole'' and cause the pilot to fly too low.

Rain on the windscreen creates the illusion of being too high. Haze creates the illusion of being too far. Sudden entry into fog causes an illusion of pitching up and causes the pilot to lower the nose and steepen the approach.

Lights in a line can be mistaken for runways.


