\chapter{Commercial Pilot Checkride}

This chapter contains a hodge-podge of material that I prepared for checkride day.

\section{Commercial Operations}

There are more formal definitions of these terms. These are my informal working definitions of them in my own words.

Commercial Operator - the person or entity that makes the airplane available for a commercial operation. Being a commercial operator is a Big Deal and often involves a bunch of legal entanglement. An individual cannot legally serve as a commercial operator without a whole mess of paperwork (think Part 121 or Part 135 operations).

Holding Out - advertising! Posting to social media, putting up a flyer at the airport, or even word of mouth can serve as advertising. Don't do it!

Common Carriage - the transport of a member of the public by airplane. Generally not someone you know personally. Can't do this with my basic commercial certificate.

Private Carriage - the transport of a familiar person, such as a friend, family member, coworker, etc. It can also include serving as someone's personal pilot for some period of time. There is a spectrum here: if you're engaging in a different ``private carriage'' operating several times a week with people you barely know, that could be construed as common carriage.

So what can you do with a commercial pilot certificate? You can be paid for flying an airplane. But you don't get to be a commercial operator or engage in common carriage unless you work for a commercial operator.

Things you may do: flight instruction, banner towing, aerial surveillance and photography, crop dusting, ferry flights, sightseeing flights (is it 50 or 25 nmi?), glider towing.

14 CFR 61.33, 119.1, AC 120-12.

\section{Aeronautical Knowledge}

Talk about hydroplaning.

The examiner may ask about the different kinds of hydroplaning. They are:

Dynamic hydroplaning. The airplane landing on a water soaked surface will build a wedge of water in front of the wheel. The wedge does not compress and lifts the wheel off the ground, causing a marked reduction in traction.

Viscous hydroplaning. The airplane landing on a water coated surface will compress a thin film of water which will lift the contact patch of the tire. A very small amount of water is necessary for this phenomenon. The key is that the surface needs to be extremely smooth with nowhere for water to go.

Reverted rubber hydroplaning. The airplane's tire locks trapping some water underneath the contact patch. The water absorbs energy and turns to superheated steam, lifting the tire and de-vulcanizing or ``reverting'' the rubber on the tire.

The NASA report Phenomena of Pneumatic Tire Hydroplaning \cite{hydroplaning} is the primary source for this material from the late 1960s. Fairly early on that source introduces the hydroplaning equation, $V_P = 9 \sqrt{p}$, where $V_P$ is the hydroplaning speed in knots, $p$ is the tire pressure in PSI, and $9$ is a unitless conversion factor for knots. To get $V_P$ in miles per hour this conversion factor is instead $10.35$.

\section{Weight Shift Formula}

\begin{equation}
    \frac{weight\ of\ object}{Weight\ of\ aircraft} = \frac{Distance\ CG\ moves}{distance\ between\ stations}
\end{equation}

\section{NTSB}

Accident - death, serious injury, substantial damage, from first passenger boarding to last passenger deboarding

Incident - not an accident but affects safety

Immediate notification: accident, flight control system, unable to perform duties, turblne failure, in flight fire, release of propeller blade, mid air collision, 25k of damage, complete loss of information more than fifty percent of EFIS, overdue airplane believed to be in accident

Serious injury - fracture, hemmhorage, internal organ, 2nd or 3rd degree burn, hospitalization 48 hours within 7 days

Serious damage - affects safety, requires repair or replacement

\section{Checkride Flight Planning}

For my checkride I was posed with this problem:

\begin{enumerate}

\item Plan a VFR cross-country as follows: HOME AIRPORT-KAEX (GPS navigation will be unavailable to you in the airplane). (We clarified that KACT, the checkride airport, is the ``home airport'' here.)

\item Have your cross-country navigation logs (winds, estimated times en-route, etc.) and flight plan forms completed prior to your arrival for the practical test. Also bring with you a weather briefing packet for the trip. You can either call the F.S.S. or use FAA approved on-line resources. You don't have to print out everything, only the information that is pertinent to the trip.

\item When you select check points for flight, try to find points that are not too far apart, preferably no more than 10-20 nm. apart. Doing so will shorten the cross-country portion of the flight.

\item I weigh 210 lbs. We will take 5 lbs. of luggage as well. Please compute weight and balance, take-off and landing distance data for our flight. Determine your aircraft's maximum range using the cruise altitude, and other conditions for the first leg.

\item Be able to locate your aircraft's current Weight and Balance and Equipment List, supplements, and the other required aircraft documents (AROW).

\item Review the current Commercial Pilot ACS in its entirety (available on www.faa.gov). Also take time to review the Appendices in the ACS, including Stall/Spin Awareness/Avoidance, and Runway Incursion Avoidance (Hot Spots), etc.

\item \emph{Information on fees and payment intentionally omitted.}

\end{enumerate}

Lots to talk about here.

\subsection{Cross Country Planning}

We're going form KACT to KAEX. This is an interesting routing for a few reasons. There is no obvious direct route. There is no single highway, railroad, or power line we can follow. There are a bunch of airports and VORs in the area. There really aren't any useful radio aids to navigation on the direct route. There are also some MOAs along the way.

With passengers on board, I would avoid MOAs, assuming they are generally active. I could check the sectional chart for more information on those.

If I had a GPS and were going VFR I'd go direct to 3R4 or to the Natchitoches (OOC) NDB and then direct to KAEX. But we're no GPS.

My first thought is to go IFR without GPS. Off airways we could go KACT ACT LOA LFK AEX KAEX. This is 261.3 nmi where direct is 240.8. If we had to stick to airways we could go KACT ACT V15 CLL V565 LFK V212 AEX KAEX. Radar and VOR services should be good in this flat land area so those would be reasonably safe.

But we should expect to do this VFR. This is the VFR route that I've picked. It comes to a hair under 250 nmi. I would do this with flight following for sure.

\begin{enumerate}
    \item Take off from KACT having announced ``VFR to the southeast''. Fly a heading of 116 for about 7 nmi.
    \item Fly over or south of the Baylor stadium. Watch for game related TFRs! Flying south of the stadium avoids KCNW's airspace.
    \item Fly heading 086 for 8nmi to a small lake as a visual reference point.
    \item Fly heading 075 for 23 nmi to overfly Mexia KLXY airport. Top of climb should be in here somewhere as well.
    \item Fly heading 072 for 19 nmi to overfly the town of Fairfield, passing two small airfields and crossing a railroad line and a highway along the way.
    \item Fly heading 079 for 23 nmi to overfly Palestine KPSN airport.
    \item Fly heading 100 for 33 nmi to overfly the town of Alto, which is on Highway 69.
    \item Fly heading 102 for 19 nmi to overfly Nacogdoches KOCH airport.
    \item Fly heading 093 for 28 nmi to overfly San Augustine 78R airport, just before a highway, a power line, a railroad line, and the town of San Augustine.
    \item Fly heading 088 for 35 nmi to overfly Hart 3R4 airport. Along the way overfly Ammons private airport, which is on a prominent peninsula.
    \item Fly heading 079 for 23 nmi to intercept Highway 49 and the rail line well south of the Natchitoches town, airport, and NDB.
    \item Following Highway 49 south, fly a general heading of 130 for 21 nmi to overfly Lake Rodemacher. Establish two way radio communication with KAEX no later than the Lake.
    \item From the lake, I would expect to be vectored in, but flying a heading of 108 for 10 nmi would take me straight to the airport.
\end{enumerate}

I chose this route for a few reasons. Flying down to LFK and overflying the MOA would not be great flying VFR. There are plenty of airports along the way which provide opportune places to divert if needed. Early on, the waypoints are close together. Since it's a checkride I'd expect a diversion pretty early on.

If I were routed to fly through KCNW's airspace to avoid town, I could just follow Highway 84 to the west. Instead of the stadium and the lake, I would fly over KCNW, then proceed on a heading of 087 for 15 nmi until crossing an abandoned military airfield. Another 14 nmi on that same heading would get me to KLXY where I could resume the previous plan. But a transition through KCNW is not guaranteed, and I couldn't get to 3000' to overfly it in time.

\subsection{Navigation Logs, Flight Plan Forms, Weather Briefing}

I would calculate these with ForeFlight and print backups the day before.

\subsection{Cross Country Planning Cross-Check: Waypoint Distances}

The points I selected are generally about 10-20 nmi apart. Again, I'm expecting an early diversion to KPWG, which is an examiner favorite, but KCNW and KLXY are possibilities as well.

Pilotage, dead reckoning, fuel, emergency equipment, briefings, etc.

\subsection{Weight and Balance}

The examiner is 210 lbs and I am 230 lbs. Together that puts us at the station limit of 440 lbs for the front seats. The examiner's 5 lb bag can go on the second row seat. I plan on my own 15 lbs bag, 20 lbs of nose cargo and 50 lbs of aft cargo. We carry about 50 gallons of fuel because I like the way the airplane handles. We burn about 14 an hour at 145 knots so 50 gallons gives 3.5 hours of endurance or about 507 nmi of range. That's plenty for this flight of 250 nmi with IFR reserves of cruise to alternate (let's say KMKV) plus 45 minutes.

I would do all of this on ForeFlight.

\subsection{Aircraft Documents}

On the Lance, the airworthiness certificate, registration certificate, and radio certificate are on the wing spar bulkhead immediately aft of the pilot's seat. The owner's manual, supplements, and weight and balance are beind the copilot's seat. I'll take photos of all of these on my phone for ease of demonstration.

\subsection{Commercial Pilot ACS Review}

...still reviewing...

I didn't see a notice in the current Commercial Pilot ACS (DATE) for Stall/Spin Awareness/Avoidance, Runway Incursion Avoidance, Hot Spots, etc. I think he meant the ``Special Emphasis Areas'' from the PTS of days of old. The FAA seems to think that, instead of calling them out separately, they should just be inline with certain ACS items. However I find the combined presentation useful and still relevant. I was able to find a copy at \url{http://sethlake.aero/FAA_SpecialEmphasisAreas.pdf}.


