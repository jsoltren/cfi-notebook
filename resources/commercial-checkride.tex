\chapter{Commercial Pilot Checkride}

This chapter contains a hodge-podge of material that I prepared for checkride day.

\section{Commercial Operations}

There are more formal definitions of these terms. These are my informal working definitions of them in my own words.

Commercial Operator - the person or entity that makes the airplane available for a commercial operation. Being a commercial operator is a Big Deal and often involves a bunch of legal entanglement. An individual cannot legally serve as a commercial operator without a whole mess of paperwork (think Part 121 or Part 135 operations).

Holding Out - advertising! Posting to social media, putting up a flyer at the airport, or even word of mouth can serve as advertising. Don't do it!

Common Carriage - the transport of a member of the public by airplane. Generally not someone you know personally. Can't do this with my basic commercial certificate.

Private Carriage - the transport of a familiar person, such as a friend, family member, coworker, etc. It can also include serving as someone's personal pilot for some period of time. There is a spectrum here: if you're engaging in a different ``private carriage'' operating several times a week with people you barely know, that could be construed as common carriage.

So what can you do with a commercial pilot certificate? You can be paid for flying an airplane. But you don't get to be a commercial operator or engage in common carriage unless you work for a commercial operator.

Things you may do: flight instruction, banner towing, aerial surveillance and photography, crop dusting, ferry flights, sightseeing flights (is it 50 or 25 nmi?), glider towing.

\section{Aeronautical Knowledge}

Talk about hydroplaning.

The examiner may ask about the different kinds of hydroplaning. They are:

Dynamic hydroplaning. The airplane landing on a water soaked surface will build a wedge of water in front of the wheel. The wedge does not compress and lifts the wheel off the ground, causing a marked reduction in traction.

Viscous hydroplaning. The airplane landing on a water coated surface will compress a thin film of water which will lift the contact patch of the tire. A very small amount of water is necessary for this phenomenon. The key is that the surface needs to be extremely smooth with nowhere for water to go.

Reverted rubber hydroplaning. The airplane's tire locks trapping some water underneath the contact patch. The water absorbs energy and turns to superheated steam, lifting the tire and de-vulcanizing or ``reverting'' the rubber on the tire.

The NASA report Phenomena of Pneumatic Tire Hydroplaning \cite{hydroplaning} is the primary source for this material from the late 1960s. Fairly early on that source introduces the hydroplaning equation, $V_P = 9 \sqrt{p}$, where $V_P$ is the hydroplaning speed in knots, $p$ is the tire pressure in PSI, and $9$ is a unitless conversion factor for knots. To get $V_P$ in miles per hour this conversion factor is instead $10.35$.

\section{Cross Country Flight Planning}

For my checkride I was posed with this problem:

\begin{enumerate}

\item Plan a VFR cross-country as follows: HOME AIRPORT-KAEX (GPS navigation will be unavailable to you in the airplane). (We clarified that KACT, the checkride airport, is the ``home airport'' here.)

\item Have your cross-country navigation logs (winds, estimated times en-route, etc.) and flight plan forms completed prior to your arrival for the practical test. Also bring with you a weather briefing packet for the trip. You can either call the F.S.S. or use FAA approved on-line resources. You don't have to print out everything, only the information that is pertinent to the trip.

\item When you select check points for flight, try to find points that are not too far apart, preferably no more than 10-20 nm. apart. Doing so will shorten the cross-country portion of the flight.

\item I weigh 210 lbs. We will take 5 lbs. of luggage as well. Please compute weight and balance, take-off and landing distance data for our flight. Determine your aircraft's maximum range using the cruise altitude, and other conditions for the first leg.

\item Be able to locate your aircraft's current Weight and Balance and Equipment List, supplements, and the other required aircraft documents (AROW).

\item Review the current Commercial Pilot ACS in its entirety (available on www.faa.gov). Also take time to review the Appendices in the ACS, including Stall/Spin Awareness/Avoidance, and Runway Incursion Avoidance (Hot Spots), etc.

\item \emph{Information on fees and payment intentionally omitted.}

\end{enumerate}

Lots to talk about here.

Pilotage, dead reckoning, fuel, emergency equipment, briefings, etc.


