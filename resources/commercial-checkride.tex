\chapter{Commercial Pilot Checkride}

This chapter contains a hodge-podge of material that I prepared for checkride day.

\section{Requirements}

The applicability for a commercial pilot candidate comes from \href{https://www.ecfr.gov/current/title-14/chapter-I/subchapter-D/part-61/subpart-F/section-61.121}{14 CFR 61.121} and thereafter.

\begin{itemize}
    \item 18 years of age
    \item English language
    \item Knowledge test
    \item Instructor endorsement
    \item Aeronautical experience
    \item Pass the checkride
    \item Private pilot or better
    \item Category and class requirements
\end{itemize}

You also need a medical. 3rd Class (or Basic Med) just to take the checkride, 2nd Class to exercise the privileges.

\subsection{Aeronautical Experience}

For further details please see \href{https://www.ecfr.gov/current/title-14/chapter-I/subchapter-D/part-61/subpart-F/section-61.129}{14 CFR 61.129}. Since this is written with the Piper Lance II in mind, this subsection focuses on Airplane Single Engine Land (ASEL) requirements.

\begin{itemize}
\item 250 hours total flight time
\item 100 hours powered, of which 50 in airplanes
\item 100 hours PIC time
      \begin{itemize}
      \item 50 hours in airplanes
      \item 50 hours cross country (greater than 50 nmi), of which 10 hours in airplanes
      \end{itemize}
\item 20 hours of training
      \begin{itemize}
      \item 10 hours instrument training using view limiting device - not IMC! 5 must be in a single.
      \item 10 hours in a complex (retract, constant speed prop, flaps), turbine, or TAA (moving map, 2 axis autopilot)
      \item 2 hour cross country, daytime, with a 100 nmi leg
      \item 2 hour cross country, night, with a 100 nmi leg
      \item 3 hours checkride prep with CFI in preceding two calendar months.
      \end{itemize}
\item 10 hours of solo time to include
      \begin{itemize}
      \item Long cross country: at least 300nmi, one 250nmi leg, at least three landings.
      \item 5 hours VFR night with 10 takeoffs, traffic patterns, and landings at a towered airport
      \end{itemize}
  \item The 10 hours may be flown with an instructor (e.g. ``supervised solo'') but you cannot mix and match. See the \href{https://www.faa.gov/about/office_org/headquarters_offices/agc/practice_areas/regulations/interpretations/Data/interps/2016/Grannis_2016_Legal_Interpretation.pdf}{Grannis Interpretation}.
\end{itemize}


\section{Commercial Operations}

There are more formal definitions of these terms. These are my informal working definitions of them in my own words.

Commercial Operator - the person or entity that makes the airplane available for a commercial operation. Being a commercial operator is a Big Deal and often involves a bunch of legal entanglement. An individual cannot legally serve as a commercial operator without a whole mess of paperwork (think Part 121 or Part 135 operations).

Holding Out - advertising! Posting to social media, putting up a flyer at the airport, or even word of mouth can serve as advertising. Don't do it!

Common Carriage - the transport of a member of the public by airplane. Generally not someone you know personally. Can't do this with my basic commercial certificate.

Private Carriage - the transport of a familiar person, such as a friend, family member, coworker, etc. It can also include serving as someone's personal pilot for some period of time. There is a spectrum here: if you're engaging in a different ``private carriage'' operating several times a week with people you barely know, that could be construed as common carriage.

So what can you do with a commercial pilot certificate? You can be paid for flying an airplane. But you don't get to be a commercial operator or engage in common carriage unless you work for a commercial operator. An instrument rating in the same category and class is required for operations in excess of 50 miles from the home base, or at night.

Things you may do: flight instruction, banner towing, aerial surveillance and photography, crop dusting, ferry flights, sightseeing flights within 25 nmi, glider towing.

\href{https://www.ecfr.gov/current/title-14/chapter-I/subchapter-D/part-61/subpart-F/section-61.133}{14 CFR 61.133}, \href{https://www.ecfr.gov/current/title-14/chapter-I/subchapter-G/part-119}{14 CFR 119.1}, \href{https://www.faa.gov/regulations_policies/advisory_circulars/index.cfm/go/document.information/documentID/22647}{Advisory Circular 120-12A - Private Carriage Versus Common Carriage of Persons or Property}.

\section{Aeronautical Knowledge}

\subsection{Hydroplaning}

The New Oxford English Dictionary offers this definition of the verb hydroplane: ``(of a vehicle) slide uncontrollably on the wet surface of a road''.

The examiner may ask about the different kinds of hydroplaning. They are:

Dynamic hydroplaning. The airplane landing on a water soaked surface will build a wedge of water in front of the wheel. The wedge does not compress and lifts the wheel off the ground, causing a marked reduction in traction.

Viscous hydroplaning. The airplane landing on a water coated surface will compress a thin film of water which will lift the contact patch of the tire. A very small amount of water is necessary for this phenomenon. The key is that the surface needs to be extremely smooth with nowhere for water to go.

Reverted rubber hydroplaning. The airplane's tire locks trapping some water underneath the contact patch. The water absorbs energy and turns to superheated steam, lifting the tire and de-vulcanizing or ``reverting'' the rubber on the tire.

The NASA report Phenomena of Pneumatic Tire Hydroplaning \cite{hydroplane} is the primary source for this material from the late 1960s. Fairly early on that source introduces the hydroplaning equation, $V_P = 9 \sqrt{p}$, where $V_P$ is the hydroplaning speed in knots, $p$ is the tire pressure in PSI, and $9$ is a unitless conversion factor for knots. To get $V_P$ in miles per hour this conversion factor is instead $10.35$.

\section{Weight Shift Formula}

\begin{equation}
    \frac{weight\ of\ object}{Weight\ of\ aircraft} = \frac{Distance\ CG\ moves}{distance\ between\ stations}
\end{equation}

\section{NTSB}

Accident - death, serious injury, substantial damage, from first passenger boarding to last passenger deboarding

Incident - not an accident but affects safety

Immediate notification: accident, flight control system, unable to perform duties, turblne failure, in flight fire, release of propeller blade, mid air collision, 25k of damage, complete loss of information more than fifty percent of EFIS, overdue airplane believed to be in accident

Serious injury - fracture, hemmhorage, internal organ, 2nd or 3rd degree burn, hospitalization 48 hours within 7 days

Serious damage - affects safety, requires repair or replacement

\section{Checkride Flight Planning - Take One}

For my checkride I was posed with this problem:

\begin{enumerate}

\item Plan a VFR cross-country as follows: HOME AIRPORT-KAEX (GPS navigation will be unavailable to you in the airplane). (We clarified that KACT, the checkride airport, is the ``home airport'' here.)

\item Have your cross-country navigation logs (winds, estimated times en-route, etc.) and flight plan forms completed prior to your arrival for the practical test. Also bring with you a weather briefing packet for the trip. You can either call the F.S.S. or use FAA approved on-line resources. You don't have to print out everything, only the information that is pertinent to the trip.

\item When you select check points for flight, try to find points that are not too far apart, preferably no more than 10-20 nm. apart. Doing so will shorten the cross-country portion of the flight.

\item I weigh 210 lbs. We will take 5 lbs. of luggage as well. Please compute weight and balance, take-off and landing distance data for our flight. Determine your aircraft's maximum range using the cruise altitude, and other conditions for the first leg.

\item Be able to locate your aircraft's current Weight and Balance and Equipment List, supplements, and the other required aircraft documents (AROW).

\item Review the current Commercial Pilot ACS in its entirety (available on www.faa.gov). Also take time to review the Appendices in the ACS, including Stall/Spin Awareness/Avoidance, and Runway Incursion Avoidance (Hot Spots), etc.

\item \emph{Information on fees and payment intentionally omitted.}

\end{enumerate}

Lots to talk about here.

\subsection{Cross Country Planning}

We're going form KACT to KAEX. This is an interesting routing for a few reasons. There is no obvious direct route. There is no single highway, railroad, or power line we can follow. There are a bunch of airports and VORs in the area. There really aren't any useful radio aids to navigation on the direct route. There are also some MOAs along the way.

With passengers on board, I would avoid MOAs, assuming they are generally active. I could check the sectional chart for more information on those.

If I had a GPS and were going VFR I'd go direct to 3R4 or to the Natchitoches (OOC) NDB and then direct to KAEX. But we're no GPS.

My first thought is to go IFR without GPS. Off airways we could go KACT ACT LOA LFK AEX KAEX. This is 261.3 nmi where direct is 240.8. If we had to stick to airways we could go KACT ACT V15 CLL V565 LFK V212 AEX KAEX. Radar and VOR services should be good in this flat land area so those would be reasonably safe. Who knows, ATC might just give me vectors the entire way.

But we should expect to do this VFR. This is the VFR route that I've picked. It comes to a hair under 250 nmi. I would do this with flight following for sure.

\begin{enumerate}
    \item Take off from KACT having announced ``VFR to the southeast''. Fly a heading of 116 for about 7 nmi.
    \item Fly over or south of the Baylor stadium. Watch for game related TFRs! Flying south of the stadium avoids KCNW's airspace.
    \item Fly heading 086 for 8nmi to a small lake, Tradinghouse Creek Reservoir, as a visual reference point.
    \item Fly heading 075 for 23 nmi to overfly Mexia KLXY airport. Top of climb should be in here somewhere as well.
    \item Fly heading 072 for 19 nmi to overfly the town of Fairfield, passing two small airfields and crossing a railroad line and a highway along the way.
    \item Fly heading 079 for 23 nmi to overfly Palestine KPSN airport.
    \item Fly heading 100 for 33 nmi to overfly the town of Alto, which is on Highway 69.
    \item Fly heading 102 for 19 nmi to overfly Nacogdoches KOCH airport.
    \item Fly heading 093 for 28 nmi to overfly San Augustine 78R airport, just before a highway, a power line, a railroad line, and the town of San Augustine.
    \item Fly heading 088 for 35 nmi to overfly Hart 3R4 airport. Along the way overfly Ammons private airport, which is on a prominent peninsula.
    \item Fly heading 079 for 23 nmi to intercept Highway 49 and the rail line well south of the Natchitoches town, airport, and NDB.
    \item Following Highway 49 south, fly a general heading of 130 for 21 nmi to overfly Lake Rodemacher. Establish two way radio communication with KAEX no later than the Lake.
    \item From the lake, I would expect to be vectored in, but flying a heading of 108 for 10 nmi would take me straight to the airport.
\end{enumerate}

I chose this route for a few reasons. Flying down to LFK and overflying the MOA would not be great flying VFR. There are plenty of airports along the way which provide opportune places to divert if needed. Early on, the waypoints are close together. Since it's a checkride I'd expect a diversion pretty early on.

If I were routed to fly through KCNW's airspace to avoid town, I could just follow Highway 84 to the west. Instead of the stadium and the lake, I would fly over KCNW, then proceed on a heading of 087 for 15 nmi until crossing an abandoned military airfield. Another 14 nmi on that same heading would get me to KLXY where I could resume the previous plan. But a transition through KCNW is not guaranteed, and I couldn't get to 3000' to overfly it in time.

Altitude selection is based on winds aloft and the hemispherical altitude rule. Being instrument rated, if I were flying at night, I would want to climb higher, 9000' IFR or 9500' VFR, to have options in case of an engine failure. With the Lance's 7.8:1 glide ratio, at 9500 feet, I would have, in theory, up to 14 statuate miles or 12 nmi of glide distance. Chances are pretty good that I would be able to make an airport along my route. Otherwise, during the day, I would probably select 5500' VFR to get high enough to give my passengers cool air, but keep them low enough that there is plenty of oxygen to breathe. 7500' and 9500' do offer better cruise performance, I might select those altitudes if I wanted to make better time. Terrain is pretty much no factor on this lowland route.

\subsection{Navigation Logs, Flight Plan Forms, Weather Briefing}

I would calculate these with ForeFlight and print backups the day before.

Make sure to look at winds aloft for several altitudes since these will come into play for the maneuvers section of the checkride.

\subsection{Cross Country Planning Cross-Check: Waypoint Distances}

The points I selected are generally about 10-20 nmi apart. Again, I'm expecting an early diversion to KPWG, which is an examiner favorite, but KCNW and KLXY are possibilities as well.

Pilotage, dead reckoning, fuel, emergency equipment, briefings, etc.

\subsection{Weight and Balance}

The examiner is 210 lbs and I am 230 lbs. Together that puts us at the station limit of 440 lbs for the front seats. The examiner's 5 lb bag can go on the second row seat. I plan on my own 15 lbs bag, 20 lbs of nose cargo and 50 lbs of aft cargo. We carry about 50 gallons of fuel because I like the way the airplane handles. We burn about 14 an hour at 145 knots so 50 gallons gives 3.5 hours of endurance or about 507 nmi of range. That's plenty for this flight of 250 nmi with IFR reserves of cruise to alternate (let's say KMKV) plus 45 minutes.

I would do all of this on ForeFlight.

\subsection{Aircraft Documents}

On the Lance, the airworthiness certificate, registration certificate, and radio certificate are on the wing spar bulkhead immediately aft of the pilot's seat. The owner's manual, supplements, and weight and balance are beind the copilot's seat. I'll take photos of all of these on my phone for ease of demonstration.

\subsection{Commercial Pilot ACS Review}

...still reviewing...

I didn't see a notice in the current Commercial Pilot ACS (DATE) for Stall/Spin Awareness/Avoidance, Runway Incursion Avoidance, Hot Spots, etc. I think he meant the ``Special Emphasis Areas'' from the PTS of days of old. The FAA seems to think that, instead of calling them out separately, they should just be inline with certain ACS items. However I find the combined presentation useful and still relevant. I was able to find a copy at \url{http://sethlake.aero/FAA_SpecialEmphasisAreas.pdf}.

\subsection{Compensation for the DPE}

In the past I've had DPEs accept electronic payment - Venmo or Zelle - for checkrides. Venmo was fine. Zelle had a transaction limit that left me scrambling for the ATM at the last minute.

Cash works.

DPEs sometimes complain about big wads of cash. Do them - and yourself - a favor and get big bills from the bank. Yes it's old school but it works. Don't forget cash for a re-examination fee. Better to have it and not need it than need it and not have it.

A sealed security envelope works nicely for the examiner's fee.

\section{Checkride Flight Planning - Take Two}

My examiner finally texted me at 9:45am that he would not be able to make a 9am scheduled checkride. That's unacceptable. So I fired him.

I changed to a different examiner who gave me a different problem. I was asked to plan a cross-country from KHYI to KVCT, with a 225 pound passenger. This is arguably an easier problem than the one I had earlier. I used a combination of SkyVector.com and ForeFlight to perform my route planning. I obtained airspace information from a digital edition of the San Antonio sectional chart.

\subsection{The Route}

KHYI to KVCT is a fairly short cross-country at 81.9 miles. It is above 50 miles in length, so to fly it legally as a commercial pilot with passengers I would need to be instrument rated.

My first inclination is always to fly it IFR. Without looking further I might expect a routing of radar vectors, direct to the VCT VOR, then direct. Indeed, a quick peek at ForeFlight shows that to be a popular route. I would select an altitude of 5000 feet (IFR) or 5500 feet (VFR). KVCT even has two VOR approaches (and an ILS into 13) that would serve us well if we lost GPS. (We could figure our position using the CWK, SAT, and VCT VORs if we did get lost.)

Of course, I expect that I won't have GPS at all, and I'll fly it VFR. This is the route I would select. Distances in nautical miles.

\begin{enumerate}
    \item Depart KHYI from the current runway and expect radar vectors to the south.
    \item Fly a nominal heading of 145 for 9 miles to pass over Fentress Airpark.
    \item Fly heading 126 for 9 miles to pass over the Town of Luling.
    \item Fly heading 129 for 13 miles to pass over Dreyer Memorial (T20) airport. We've been mostly following the San Marcos River until now.
    \item Fly heading 155 for 29 miles to pass over Cuero Muni (T71) and the Town of Cuero, where a number of roads and railroads converge near a prison.
    \item Fly heading 120 for 13 miles to pass a closed airfield. From here we should be looking straight down Runway 13 at KVCT. This is a good place to start our descent if we have not already, and to establish radio contact with KVCT tower.
    \item Fly heading 128 for 10 miles to KVCT.
\end{enumerate}

\subsection{Airspace Analysis}

Departure from KHYI, a Class D towered airport with ceilings up to 2700 feet. It reverts to Class E when the tower is closed. Traffic pattern for me in the Lance is 1600 AGL. We exit the airport airspace into Class E airspace.

Near Fentress Airpark we transition from Class E airspace with a 700 foot base to Class E with a 1200 foot base. We also have warnings of parachute activity in the vicinity.

Near Luling we cross Victor aitrways V198. That is Class E airspace too, but we are already in it.

Near T20 we cross two military training route: IR148 and VR1120. IR148 will see high speed military IFR traffic above 1500 feet AGL (three digits). VR1120 will see military traffic operating at VFR below 1500 feet AGL (four digits).

We pass underneath the Randolph 1A MOA (8000 to 17999 feet, SR-SS Mon-Fri), Alert Area A-632E (6000 to but not including 9000 feet, SR-2400 Mon-Fri, 1400-2400 Sun, or by DOD NOTAM), and the Kingsville 5 MOA (9000 to 17999 feet, SR-2400 Mon-Fri, 1400-2400 Sun). Those are listed in ForeFlight as well as in the margin of the VFR Sectional Chart.

KVCT is a towered Class D airport with ceilings up to 2600 feet. It reverts to Class E when the tower is closed. Pattern altitude for me in a light aircraft is 1100 AGL.

An altitude of 4000-4500 feet northbound and 5000-5500 feet southbound is appropriate for cruise without needing to deal with the MOAs or Alert Area.

It should be no factor, but I would stay far away from R-6312, which is designated for ``high altitude release bomb training''. Yikes. \url{https://www.federalregister.gov/documents/2001/10/29/01-27159/modification-of-restricted-area-r-6312-cotulla-tx}

\subsection{Weather Considerations}

Victoria is far enough inland that we should be shielded from advection fog and other coastal weather phenomena. I would expect winds from the southeast.

For an IFR alternate, I would be looking for something a bit further inland and towered. There isn't much, so we might be going back where we came. I'd select KSAT to give my passengers the greatest number of options possible, or offer to take them back to KHYI. Depending on their mission and the weather, we could select another nearby field.

\subsection{Weight and Balance}

Much of the previous discussion applies. 50 gallons of fuel is plenty to fly all the way to VCT, go missed, fly back to SAT and have ample reserves.






