\section*{Introduction}

This book represents Jos\'e's draft CFI notebook. One day, I hope to use this collection of notes to help me remember everything I learned, and how I learned it, so that I might share this wisdom with my students. Perhaps one day, it will cease to be a draft and become a book.

The private pilot certificate is attainable by a motivated 17 year old learner. The challenge we face as instructors is
how to tailor the material to the learner.

There are some training materials that are extremely thorough, Denker \cite{denker} comes to mind. They are fantastic. They also assume a pretty rigorous grounding in physics. I love these training materials - as \emph{references}.

So, where does this work lie? My goal is to use this notebook to simplify some truly advanced resources, and make them attainable
and usable for my students.

As an engineer, I perceive the world - including aviation - in a rigorous, quantitative fashion. Sure, when flying, I make subjective, sometimes spontaneous decisions, based on experience. But, when learning, I find that the basics are enough. I need rigor. I need quantification. The added information is most immediately useful as a memory aid, providing another mental structure (in addition to time in the aircraft) to scaffold everything we learn.

I'll take lazy eights as an example here. The author's command on this commercial pilot maneuver took a quantum leap once some key observations were made. We're exploring a stable mode of the airplane: the desire to gently nose over as we lose lift.

Join me as we take to the skies.

