\section*{Introduction}

DISCLAIMER: As of this writing, the author is NOT an FAA-certified ground or flight instructor. This book must be used for reference purposes, only. For the purpose of ground or flight instruction, the reader is directed to consult a ground instructor, certificated flight instructor (CFI) or certificated flight instrument instructor (CFII).

This book represents Jos\'e's draft CFI notebook. One day, I hope to use this collection of notes to help me remember everything I learned, and how I learned it, so that I might share this wisdom with my students.

There are some training materials that are directed at a motivated 17 year old. They gloss over details and speak in platitudes. They are helpful, but I would argue they are a good start. I would argue they do not assume enough of their audience.

There are some training materials that are extremely thorough, Denker \cite{denker} comes to mind. They are fantastic. They also assume a pretty rigorous grounding in physics. I love these training materials - as \emph{references}.

So, where does this work lie? It's somewhere in between.

As an engineer, I perceive the world - including aviation - in a rigorous, quantitative fashion. Sure, when flying, I make subjective, sometimes spontaneous decisions, based on experience. But, when learning, I find that the basics are enough. I need rigor. I need quantification. The added information is most immediately useful as a memory aid, providing another mental structure (in addition to time in the aircraft) to scaffold everything we learn.

I'll take lazy eights as an example here. The author's command on this commercial pilot maneuver took a quantum leap once some key observations were made.

Join me as we take to the skies.

