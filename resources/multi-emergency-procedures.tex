
\chapter{Emergency Procedures}

\textbf{KNOW PROCEDURES BY MEMORY!}

\section{Engine Failure in Flight (above 3000' AGL)}

\subsection{Procedures}
\textbf{Airspeed: PITCH FOR \textcolor{blue}{BLUE LINE} - \vyse (85 KIAS),\\OR maintain altitude at a higher airspeed if able.}

\textbf{Directional Control: MAINTAIN, verify with AI/DG.}

\textbf{ALL AVAILABLE POWER:}
\begin{enumerate}
    \item \textbf{\textcolor{red}{MIXTURES} SET (max available power)}
    \item \textbf{\textcolor{blue}{PROPELLERS} Full Forward}
    \item \textbf{THROTTLES Full Power}
\end{enumerate}

\textbf{CLEAN UP:}
\begin{enumerate}
    \item \textbf{FLAPS UP (or as required)}
    \item \textbf{GEAR UP (or as required)}
    \item \textbf{RAISE THE DEAD: Trim and bank into good engine.}
\end{enumerate}

\textbf{IDENTIFY/VERIFY:}
\begin{enumerate}
    \item \textbf{IDENTIFY Dead foot = dead engine.}
    \item \textbf{VERIFY Power idle on dead engine. No change in performance - verified.}
\end{enumerate}

\textbf{FIX OR FEATHER:}
\begin{itemize}
    \item \textbf{DECISION Based upon situation/altitude (Restart or Feather?)}
\end{itemize}

If immediately after takeoff, assess the situation. Are we maintaining altitude? Great, circle back to land.
Otherwise, we are pitching for minimum sink rate and bracing for what is ahead.

Rarely do engines fail suddenly and completely (fuel starvation is the exception). If an engine is running poorly, but
developing some power, \textbf{you are better off letting it run} (above 3000 AGL) until you sort out the problem. The
decision to feather should be made with some deliberation. A catastrophic engine failure would require feathering
the engine (avoiding a potential fire), however, a rough running engine should not be feathered as any horsepower it
is producing potentially helps.

\textbf{The exception is during a critical phase of flight}, such as initial climb-out,
approach or landing. During these phases of flight the propeller on the problem engine should be feathered
immediately, as there is not enough time to safely perform the fix procedures. Maintaining aircraft control is
\emph{the priority} and you should land as soon and as safely as possible.

\textbf{FIX:}

\textbf{CHECK (\textcolor{red}{Red Items}) - only on affected engine.}
\begin{enumerate}
    \item \textbf{Fuel - ON}
    \item \textbf{Carburetor Heat - ON}
    \item \textbf{Mixture - SET (max available power)}
    \item \textbf{Boost Pumps - ON}
    \item \textbf{Magnetos - CYCLE Left - Right - Both}
\end{enumerate}

\textbf{\textcolor{red}{WARNING} Feathering the wrong engine is incredibly dangerous!
Work methodically to make certain the correct engine is feathered.}

\textbf{FEATHER:}
\begin{enumerate}
    \item \textbf{Feather Propeller on inoperative engine.}
    \item \textbf{Mixture Idle Cutoff on inoperative engine.}
    \item \textbf{Power as needed on good engine to maintain altitude/airspeed.\\Minimum speed: \textcolor{blue}{BLUE LINE 85 KIAS}.}
\end{enumerate}

\emph{Memory aid: 3\degree{} bank for slip.}

\textbf{SHUTDOWN AND SECURE ENGINE:}
\begin{itemize}
    \item \textbf{Mixture - VERIFY Idle Cutoff}
    \item \textbf{Fuel Selector - OFF}
    \item \textbf{Cowl Flaps - Open on operative engine, closed on inoperative engine.}
    \item \textbf{Fuel Pump - OFF}
    \item \textbf{Magnetos - OFF}
    \item \textbf{Alternator Switch - OFF}
    \item \textbf{Notify ATC.}
    \item \textbf{Land as soon as practical.}
\end{itemize}

\textbf{\textcolor{red}{** NOW REFER TO CHECKLIST. **}}


\section{Engine Failure During or After Takeoff}

\subsection{Procedures}

{\centering
\textbf{DURING INITIAL CLIMB OUT THE NOSE NEEDS TO BE LOWERED\\
5 DEGREES OR MORE TO MAINTAIN 85 KIAS!!!}
\par }

Bank approximately \textbf{3 degrees} toward the good engine with the rudder ball half out toward the good engine. This
will provide maximum climb performance. \textbf{Each degree of bank back toward the inoperative engine increases \vmc
by 3 knots per degree. Therefore, with only a 2 degree bank toward the operative engine, \vmc might be 3 knots higher
than published.}

If the pilot inadvertently or instinctively tries to hold wings level in an engine out situation, \vmc
\textbf{CAN INCREASE AS MUCH AS 15 KNOTS. THE AIRCRAFT COULD BE UNCONTROLLABLE AT A
SPEED AS HIGH AS \vyse!} This situation \textbf{WILL EXIST} if the pilot flies wings level and tries to maintain heading
with the ball centered.

\emph{Memory aid: Raise the dead (engine).}

\subsection{Power-Loss Briefing (Before Takeoff)}

A power loss briefing is to be given before takeoff to remind the pilot of the actions to be taken in the event of a
power loss during or after the takeoff roll. Time is critical so actions must be immediate but deliberate.

\bfseries{

Loss of directional control on the ground:
\begin{itemize}
    \item Throttles - IDLE
    \item Regain Control (mostly rudder)
    \item Brake straight ahead
\end{itemize}

Airborne loss of directional control: usable runway remaining and gear down:
\begin{itemize}
    \item Throttles – IDLE
    \item Land
    \item Brake straight ahead
\end{itemize}

Airborne loss of directional control: no usable runway remaining or gear up:
\begin{itemize}
    \item Blue Line, Maintain Heading, All Available Power
    \item Clean Up
        \begin{itemize}
            \item[\ding{226}] Flaps – UP
            \item[\ding{226}] Gear – UP
        \end{itemize}
    \item Identify dead engine and Raise The Dead
    \item Verify dead engine
    \item Feather dead engine
    \item Return for landing
\end{itemize}
%} % \bfseries
\mdseries

\section{Airman Certification Standards}

\emph{Source: FAA-S-ACS-7B, Commercial Pilot for Airplane Category Airman Certification Standards, November 2023} 

\subsection{X.A Maneuvering with One Engine Inoperative (AMEL, AMES)}

% From https://www.tablesgenerator.com/ with modifications for [H] and raggedleft/right.

\begin{table}[H]
\centering
\begin{tabular}%
  {>{\raggedleft\arraybackslash}p{0.15\linewidth}%
   >{\raggedright\arraybackslash}p{0.8\linewidth}%
  }
%\textbf{Task}                                                      & \textbf{A. Maneuvering with One Engine Inoperative (AMEL, AMES)}                                                                                                        \\ \hline
\textit{References:}                                                & \textit{FAA-H-8083-2, FAA-H-8083-3, FAA-H-8083-25; FAA-P-8740-66; POH/AFM}                                                                                              \\
\textbf{Objective:}                                                 & To determine the applicant exhibits satisfactory knowledge, risk management, and skills associated with maneuvering with one engine inoperative.                        \\
\textit{\textbf{Note:}}                                             & \textit{See Appendix 2: Safety of Flight and Appendix 3: Aircraft, Equipment, and Operational Requirements \& Limitations for information related to this Task.}        \\ \hline
\textbf{Knowledge:}                                                 & The applicant demonstrates understanding of:                                                                                                                            \\
\textit{CA.X.A.K1}                                                 & Factors affecting minimum controllable speed (\vmc)                                                                                                                      \\
\textit{CA.X.A.K2}                                                 & \vmc (red line) and best single-engine rate of climb airspeed (\vyse) (blue line).                                                                                        \\
\textit{CA.X.A.K3}                                                 & How to identify, verify, feather, and secure an inoperative engine.                                                                                                     \\
\textit{CA.X.A.K4}                                                 & Importance of drag reduction, including propeller feathering, gear and flap retraction, the manufacturer's recommended control input and its relation to zero sideslip. \\
\textit{CA.X.A.K5}                                                 & Feathering, securing, unfeathering, and restarting.                                                                                                                     \\ \hline
\textbf{\begin{tabular}[c]{@{}l@{}}Risk\\ Management:\end{tabular}} & The applicant is able to identify, assess, and mitigate risk associated with:                                                                                           \\
\textit{CA.X.A.R1}                                                 & Potential engine failure during flight.                                                                                                                                 \\
\textit{CA.X.A.R2}                                                 & Collision hazards.                                                                                                                                                      \\
\textit{CA.X.A.R3}                                                 & Configuring the airplane.                                                                                                                                               \\
\textit{CA.X.A.R4}                                                 & Low altitude maneuvering, including stall, spin, or controlled flight into terrain (CFIT).                                                                              \\
\textit{CA.X.A.R5}                                                 & Distractions, task prioritization, loss of situational awareness, or disorientation.                                                                                    \\ \hline
\textbf{Skills:}                                                    & The applicant exhibits the skill to:                                                                                                                                    \\
    \textit{CA.X.A.S1}                                                          & Recognize an engine failure, maintain control, use manufacturer’s memory item procedures, and use appropriate emergency procedures.                                     \\
    \textit{CA.X.A.S2}                                                          & Set the engine controls, identify and verify the inoperative engine, and feather the appropriate propeller.                                                             \\
    \textit{CA.X.A.S3}                                                          & Use flight controls in the proper combination as recommended by the manufacturer, or as required to maintain best performance, and trim as required.                    \\
    \textit{CA.X.A.S4}                                                          & Attempt to determine and resolve the reason for the engine failure.                                                                                                     \\
    \textit{CA.X.A.S5}                                                          & Secure the inoperative engine and monitor the operating engine and make necessary adjustments.                                                                          \\
    \textit{CA.X.A.S6}                                                          & Restart the inoperative engine using manufacturer’s restart procedures.                                                                                                 \\
    \textit{CA.X.A.S7}                                                          & Maintain altitude ±100 feet or minimum sink rate if applicable, airspeed ±10 knots, and selected headings ±10°.                                                         \\
    \textit{CA.X.A.S8}                                                          & Complete the appropriate checklist(s).                                                                                                                                 
\end{tabular}
\end{table}
  
% Engine Failure During Takeoff Before \vmc (Simulated) (AMEL, AMES)

\newpage

\subsection{IX.E Engine Failure During Takeoff Before \vmc (Simulated) (AMEL, AMES)}

\begin{table}[h]
\centering
\begin{tabular}%
  {>{\raggedleft\arraybackslash}p{0.15\linewidth}%
   >{\raggedright\arraybackslash}p{0.8\linewidth}%
  }
\textit{References:}       & \textit{FAA-H-8083-2, FAA-H-8083-3, FAA-H-8083-25; FAA-P-8740-66; POH/AFM}                                                                                                        \\
\textbf{Objective:}        & To determine the applicant exhibits satisfactory knowledge, risk management, and skills associated with engine failure during takeoff before minimum controllable airspeed (\vmc). \\
\textit{\textbf{Note:}}    & \textit{See Appendix 2: Safety of Flight and Appendix 3: Aircraft, Equipment, and Operational Requirements \& Limitations for information related to this Task.}                  \\ \hline
\textbf{Knowledge:}        & The applicant demonstrates understanding of:                                                                                                                                      \\
\textit{CA.IX.E.K1}        & Factors affecting minimum controllable speed (\vmc).                                                                                                                               \\
\textit{CA.IX.E.K2}        & \vmc (red line) and best single-engine rate of climb airspeed (\vyse) (blue line).                                                                                                  \\
\textit{CA.IX.E.K3}        & Accelerate/stop distance.                                                                                                                                                         \\ \hline
\textbf{\begin{tabular}[c]{@{}l@{}}Risk\\ Management:\end{tabular}}  & The applicant is able to identify, assess, and mitigate risk associated with:                                                                                                     \\
\textit{CA.IX.E.R1}        & Potential engine failure during takeoff.                                                                                                                                          \\
\textit{CA.IX.E.R2}        & Configuring the airplane.                                                                                                                                                         \\
\textit{CA.IX.E.R3}        & Distractions, task prioritization, loss of situational awareness, or disorientation.                                                                                              \\ \hline
\textbf{Skills:}           & The applicant exhibits the skill to:                                                                                                                                              \\
\textit{CA.IX.E.S1}        & Close the throttles smoothly and promptly when a simulated engine failure occurs.                                                                                                 \\
\textit{CA.IX.E.S2}        & Maintain directional control and apply brakes (AMEL), or flight controls (AMES), as necessary.                                                                                   
\end{tabular}
\end{table}

\newpage

\subsection{IX.F Engine Failure After Liftoff (Simulated) (AMEL, AMES)}

\begin{table}[H]
\centering
%\begin{tabular}{rl}
\begin{tabular}%
  {>{\raggedleft\arraybackslash}p{0.15\linewidth}%
   >{\raggedright\arraybackslash}p{0.8\linewidth}%
  }
\textit{References:}                                                                    & \textit{FAA-H-8083-2, FAA-H-8083-3, FAA-H-8083-25; FAA-P-8740-66; POH/AFM}                                                                                                                                                 \\
\textbf{Objective:}                                                                     & To determine the applicant exhibits satisfactory knowledge, risk management, and skills associated with engine failure after liftoff.                                                                                      \\
\textit{\textbf{Note:}}                                                                 & \textit{See Appendix 2: Safety of Flight and Appendix 3: Aircraft, Equipment, and Operational Requirements \& Limitations for information related to this Task.}                                                           \\ \hline
\textbf{Knowledge:}                                                                     & The applicant demonstrates understanding of:                                                                                                                                                                               \\
\textit{CA.IX.F.K1}                                                                     & Factors affecting minimum controllable speed (\vmc).                                                                                                                                                                        \\
\textit{CA.IX.F.K2}                                                                     & \vmc (red line), \vyse (blue line), and safe single-engine speed ($V_{SSE}$).                                                                                                                                                     \\
\textit{CA.IX.F.K3}                                                                     & Accelerate/stop and accelerate/go distances.                                                                                                                                                                               \\
\textit{CA.IX.F.K4}                                                                     & How to identify, verify, feather, and secure an inoperative engine.                                                                                                                                                        \\
\textit{CA.IX.F.K5}                                                                     & Importance of drag reduction, including propeller feathering, gear and flap retraction, the manufacturer’s recommended control input and its relation to zero sideslip.                                                    \\
\textit{CA.IX.F.K6}                                                                     & Simulated propeller feathering and the evaluator’s zero-thrust procedures and responsibilities.                                                                                                                            \\ \hline
\multicolumn{1}{l}{\textbf{\begin{tabular}[c]{@{}l@{}}Risk\\ Management:\end{tabular}}} & The applicant is able to identify, assess, and mitigate risk associated with:                                                                                                                                              \\
\textit{CA.IX.F.R1}                                                                     & Potential engine failure after lift-off.                                                                                                                                                                                   \\
\textit{CA.IX.F.R2}                                                                     & Collision hazards.                                                                                                                                                                                                         \\
\textit{CA.IX.F.R3}                                                                     & Configuring the airplane.                                                                                                                                                                                                  \\
\textit{CA.IX.F.R4}                                                                     & Low altitude maneuvering, including stall, spin, or controlled flight into terrain (CFIT).                                                                                                                                 \\
\textit{CA.IX.F.R5}                                                                     & Distractions, task prioritization, loss of situational awareness, or disorientation.                                                                                                                                       \\ \hline
\textbf{Skills:}                                                                        & The applicant exhibits the skill to:                                                                                                                                                                                       \\
\textit{CA.IX.F.S1}                                                                     & Promptly recognize an engine failure, maintain control, and use appropriate emergency procedures.                                                                                                                          \\
\textit{CA.IX.F.S2}                                                                     & Establish \vyse; if obstructions are present, establish best single-engine angle of climb speed (\vxse) or \vmc +5 knots, whichever is greater, until obstructions are cleared. Then transition to \vyse.                      \\
\textit{CA.IX.F.S3}                                                                     & Reduce drag by retracting landing gear and flaps in accordance with the manufacturer’s guidance.                                                                                                                           \\
\textit{CA.IX.F.S4}                                                                     & Simulate feathering the propeller on the inoperative engine (evaluator should then establish zero thrust on the inoperative engine).                                                                                       \\
\textit{CA.IX.F.S5}                                                                     & Use flight controls in the proper combination as recommended by the manufacturer, or as required to maintain best performance, and trim as required.                                                                       \\
\textit{CA.IX.F.S6}                                                                     & Monitor the operating engine and aircraft systems and make adjustments as necessary.                                                                                                                                       \\
\textit{CA.IX.F.S7}                                                                     & Recognize the airplane’s performance capabilities. If a climb is not possible at \vyse, maintain \vyse and return to the departure airport for landing, or initiate an approach to the most suitable landing area available. \\
\textit{CA.IX.F.S8}                                                                     & Simulate securing the inoperative engine.                                                                                                                                                                                  \\
\textit{CA.IX.F.S9}                                                                     & Maintain heading ±10° and airspeed ±5 knots.                                                                                                                                                                               \\
\textit{CA.IX.F.S10}                                                                    & Complete the appropriate checklist(s).
\end{tabular}
\end{table}

