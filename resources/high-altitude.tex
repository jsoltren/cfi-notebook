\chapter{High Altitude Operations}

In this chapter, we introduce high altitude operations.

Before we dive in, we must answer a couple of questions.

What are high altitude operations? Simply put, these are operations far outside the ``normal'' environment you might expect to find on a Standard Day at sea level. We're talking about altitudes that are not terribly comfortable for humans, and possibly altitudes that are so high above MSL that we won't find any terrain there at all. There are enough aeromedical factors and enough changes to aircraft performance that we need to consider these operations carefully.

Why do we care about high altitude operations? We might need to get high enough to cross some mountains and stay above terrain. But there is also the desire for speed. I'm not just talking about the speed limit of 250 knots below 10,000 feet - if we wish to go faster, we need to go higher. I'm talking about the fact that \emph{thinner air provides less drag}, which permits us to fly through the air more efficiently.

Mother Nature, of course, exists outside the realm of aviation. The challenges of high altitudes were present well before the first powered flights - just ask anyone who tried to climb Mount Everest in the 1800s. So, first, we must have a look at the atmosphere itself.

\section{The Atmosphere and Its Layers}

Recalling the basics from aviation weather, we know that the atmosphere is divided into several layers. Starting from the ground and going on up, we have the troposphere, the stratosphere, the mesosphere, the thermosphere, and the exosphere. All weather - and most of the earth's air! - reside in the troposphere. Flight operations rarely venture into the stratosphere (even more rarely now that Concorde is no longer flying).

So, when we say ``high altitude'', we mean ``higher than a short cross country flight but lower than Concorde''. We're talking altitudes below about 45,000 feet. If you ever manage to get above that, please let me know. (I'm looking at you, astronaut friends... you know who you are!)

\section{Density and Altitude}

In order to better talk about the atmosphere, can we all agree on a make believe, consistent, common, daresay, \emph{standard} atmosphere?

Well, apparently, we can. Aviators are all too familiar with the International Standard Atmosphere. This aviator only recently learned that when we say standard, we mean Standard, namely, International Standard (ISO) 2533 from 1975, which is identical to the ICAO Standard Atmosphere from -2 to 32 km, or up to about 100,000 feet.

Assuming you've paid the 198 CHF fee, or can otherwise access the standard, you would see an interesting thing happening. You would see the value of p, the pressure, and $\rho$, the density of the atmosphere, decrease as altitude increases - and rapidly at that. At sea level (zero elevation), pressure is the all too familiar 1013.25 hectopascals. At 3,000 meters or about 10,000 feet, we're down to 701.21 hectopascals - only 69\% of the pressure at sea level! That means 69\% as much air, 69\% as much oxygen available for breathing or combustion, 69\% as much drag.

Up at 10,000 meters or about 33,000 feet - typical airliner altitudes - we're down to 262 hectopascals. That's just over 25\% of the pressure at sea level.

This is a huge change! It's enough that we need to pay special attention to how we are powering our airplanes, how we are providing our bodies with oxygen, and how we are constructing our airplanes. It's enough that we need to take into account certain operational considerations.

\section{Propulsion}

One of the problems with high altitude flight is that the air is so thin, there may not be enough oxygen for the (internal combustion) engine to run properly.

Recall that a normally aspirated aircraft with a piston engine has a service ceiling. That ceiling is defined as the altitude at which the airplane's climb rate slows to 100 feet per minute \cite{marchman}. That exact altitude will vary based on aircraft loading and how the conditions compare to a standard day. Famously, Concorde would continue to climb while underway, going higher as it got lighter. But the fact remains: for a piston airplane, we can only go so high.

Since the limitation is oxygen, and not fuel, there are a few ways that we could gain power at a higher altitude.

What if we had an electric aircraft? That would be great - it doesn't need oxygen at all. Today, in 2022, battery technology isn't quite there, nor are solar cells, outside from a very few research aircraft.

What if we simply had supplemental oxygen on board? Certainly, if we carried oxygen tanks, we could supplement the oxygen in ambient air to gain more power. Operationally, this is not a great idea, since we now have to deal with the added weight, complexity, and risk of an oxygen system, and it will not last very long. But air-breathing rockets do exactly this, so it's not a bad idea per se, simply not the right one.

What if we could grab more air from the atmosphere, possibly by compressing the air before it gets to the engine intake manifold? Now we're on the right track.

The first approach we might take is simply inserting a compressor between the outside air intake and the engine's intake manifold. That compressor could be a piston pump, similar to the vacuum pum in the airplane. But that's not terribly efficient: it generates a fair amount of waste heat. For this application, a turbine is a better idea. We could power this compressor with the aircraft's engine, most likely with a belt or a gear drive off the engine crankshaft. Such a contraption has a name: supercharger. It requires more power from the engine, and there are some losses due to the belt or gear drive mechanism, but it is a new positive up to a point: we can shove more oxygen into the engine, and raise our service ceiling.

Some clever person (who?) looked elsewhere on the engine and found another possible source of energy for driving this compressor. They realized that the exhaust gasses coming out of the engine were quite warm and under a decent amount of pressure. What if this energy could be captured? We can add a \emph{second} turbine in line with the exhaust pipe, and use the torque generated by that turbine to drive our intake compressor! As it turns out, this works better than a supercharger. We call this a ``turbocharger''. Sometimes we have more than one.

(some diagrams should go here)

Operating a supercharged or turbocharged engine has some key differences from operating a normally aspirated engine. One, we need to re-calibrate our manifold pressures. By this I mean, in normal operation, a piston engine's intake manifold would achieve a pressure no higher than ambient pressure, likely no more than 32 inches of mercury. But, a turbocharged (or supercharged) engine may be able to go above this. I say ``may'' because some turbos simply give sea-level pressure at higher altitudes (turbonormalized), whereas others can go well above sea level pressure. Regardless, the turbo lets the engine run harder than it could otherwise, which means increased heat and risk of engine damaged. There are often limitations on how long an engine may run on a particular power setting as a result.

The turbocharger (or supercharger) turbines are themselves metal parts that are subject to fatigue and thermal stresses. After flights, we want to give the turbines some time to cool down, lest we shut them off entirely and subject them to thermal shock.

Recall that, when we compress air, it gets quite warm. This is unfortunate, since we recall from the application of carburetor heat that warm air will lean our mixture. What if there were a way to compress the air, and then cool it to get it as dense as possible? There is: it is called an ``intercooler''. But, as this requires even more heat to be dissipated from the aircraft, the intercooler must be carefully placed so that it can dissipate heat effectively.

With a turbocharged engine in particular, the concept of ``exhaust gas temperature'' takes on a new meeting. Where are we measuring: before the exhaust turbine, or after? It makes more sense to measure before, and we call this the ``turbine inlet temperature'' or TIT. We need to manage this temperature carefully: we want the TIT to be as high as possible for efficient leaning of the engine, but if it is too high, it could melt the exhaust turbine.

These turbines need to spin quite fast - think 100,000 RPM - in order to be effective. It is challenging and expensive to create machinery that can operate at this speed.

All of this seems awfully complicated for a little bit of extra power. What if there was a better way?

There is. It's called the jet engine.

The jet engine is, at is simplest, a single turbine. It ingests air and compresses it. We inject fuel into the compressed air and ignite it, causing it to heat and expand. The exhaust heats and expands so much that it exerts a force on a turbine, which serves as thrust for the airplane. Much like the turbocharger, the jet engine grabs some energy from the exhaust system to power the input compression phase.

More specifically, this single arrangement is called a "turbojet". It is simple, powerful, and reliable. So long as fuel usage and noise are of no concern, this is the best powerplant. The military often doesn't care about fuel or noise so we see turbojets on plenty of military aircraft. If we care more about efficiency and noise than (potentially supersonic) speed, we can also attach a really fast-spinning propeller to this, and capture the propeller's thrust in a duct for maximum efficiency. This combination of a turbojet with a ducted fan is called a "turbofan" and is the most popular way of powering large aircraft.

Of course, the propulsion system allows the aircraft to operate at a higher altitude. But what about the pilot, flight crew, and passengers? Do we need to do anything special for them if we are flying at high altitudes?

\section{Aeromedical Factors}

Much like the airplane's engine, the human body needs a certain amount of oxygen to perform.

\subsection{Impairment}

\subsection{Hypoxia}

\subsubsection{Hypoxic Hypoxia}

If we want to get really specific about this, the real problem is that the ``partial pressure" of the oxygen in the air is not enough to cross into the blood stream. So the problem we need to solve, one way or another, is making sure that there is enough oxygen in the air, that it has enough pressure to cross into the blood stream.

\subsubsection{Hypemic Hypoxia}

\subsubsection{Stagnant Hypoxia}

\subsubsection{Histotoxic Hypoxia}

\subsection{Time of Useful Consciousness}

\subsection{Diving and Flight}

\subsection{Humidity}

\section{Oxygen Systems}

At this point we are well convinced that we need more oxygen. It will keep the crew able to do their job and the passengers happy. The next natural question is, where should we get it? We have two options: do a better job of pulling it out of the air, or, carry it with us.

\subsection{Pressurization}

One way of getting more oxygen into the cabin is to grab more oxygen from the air around the airplane. At typical airliner altitudes, there is not enough oxygen to sustain human life. But, there is enough that, if we had some way of compressing air and keeping it in the airplane, we could manage.

So, to be pedantic (a theme in this book, in case you have not noticed), the problems of ``getting more oxygen" and ``keeping it around" are separate, but must work together to provide more oxygen to the humans and other living creatures aboard the aircraft.

We've already talked about some solutions to the ``getting more oxygen" problem. They include superchargers, turbochargers, and turbine compressor sections. These are already pressurizing the air for the sake of the engine. Is getting more oxygen really as simple as siphoning off a bit of pressurized air from these compression systems, maybe filtering it, and breathing it in? In short, yes.

What if we don't have a turbo or a turbine? Do we have other options? Sure, but they are not terribly common. One is to simply use a pump. Using a combination of ram air pressure and a pump, we could compress air in much the same way a turbocharger would. To power this pump, we could use an electrical source, a belt or gear off the engine, or maybe even a ram air turbine. The author is not aware of such a system ever being built or used - prove me wrong!

We might also use an oxygen concentrator. We're just beginning to see these applied to the world of aviation. Oxygen concentrators are effectively reverse-osmosis machines for air. They grab air from the atmosphere and push it through an extremely fine filter (a porous solid). The filter is sized to allow oxygen moleculed through but nothing much larger. On the other side, we find a higher concentration of oxygen (and nitrogen!) than we might otherwise find.

But, now that we have the extra oxygen, where do we keep it? One option is to pressurize the entire aircraft, blowing it up ever so slightly like a giant aluminum (or carbon fiber) balloon. One option is to keep it in a tank. One option is to just put it in a pipe and send it directly to a human, discarding any excess.

The most popular method is pressurization. We inflate the entire aircraft. In designing the aircraft, we take great care to pressurize only as much of the airframe as necessary, and select materials, openings, and structures that will permit the airframe to withstand not only an overpressure compared to the outside world, but also repeated cycles. (The early de Havilland DH.106 Comet airliner is a famous example of a plane designed without understanding of the stresses of pressurization cycles.)

With a pressurized aircraft, there are some essential design and operational considerations.

First off, let's say we pressurize the aircraft once. Are we done? Can we simply seal off the aircraft once this is done? The answer is no: in doing so, the occupants of the aircraft would slowly, but surely, convert all the oxygen in the cabin to carbon dioxide, leading to the eventual suffocation of all on board. (Plus, depending on what everyone had for dinner, it would smell pretty awful in the cabin, if you take my meaning. I can hear my children giggling at the prospect.)

So, we need not only pressurization, but also a constant flow of fresh air through the cabin. We have a source of air, likely bleed air from a turbine or a jet. So what about exhaust air? We can lose air through a controlled leak in the aircraft. We're familiar with controlled leaks: the vertical speed indicator famously uses a controlled leak to operate.

How much should we pressurize the airplane? This is a complicated choice. We can refer to our Standard Atmosphere table as a reference. Keeping the pressurization level as close as possible to sea level will keep our passengers and crew as comfy as can be. But, this results in the highest overpressure and the most stress on the airframe. We know from the FARs that above 12,500 feet, the crew need oxygen (after 30 minutes), and above 15,000 feet, we must offer oxygen to our passengers. Presumably this is because the pilots need to be alert, whereas the passengers mostly need to stay seated and are free to doze off. So, if I had to venture a guess, I'd interpolate: take the best case (sea level) and the worst case (15,000 feet) and take the average. Pressurize to 7,500 feet? As it turns out, yes, airliners often pressurize to about 8,000 feet, so this isn't a bad guess after all.

Airframe designers aren't thinking in terms of altitude, they're thinking in terms of pressure, and specifically, differential pressure: the difference between the pressure inside the airplane and the pressure outside. They're designing for a particular pressure differential, taking the airplane's operating altitudes into consideration. A hypothetical airliner with a cabin pressure altitude of 10,000 feet flying at a ceiling of 20,000 feet, has to deal with a much lower pressure differential than another airliner with a cabin pressure altitude of 6,000 feet (hello, Dreamliner!) operating at 40,000 feet.

Now that we know what pressure we need, and how to maintain it (with a controlled leak), how do we get there? Can we just start pumping air into the cabin? Not exactly. Humans (and other animals...) have fairly sensitive respiratory systems. Pressurizing too fast can wreak havoc on ear drums and sinuses - especially if one is congested. So, pressurization systems need to have a rate control mechanism. This can be manual or automated.

With the rate considered, we can have three kinds of pressure: too much, not enough, and just right. (Terminology is my own.)

What if we have too much pressure? Well, the aircraft has a structural limitation. Too much pressure will start blowing out windows, blowing rivets, etc. If the calibrated drain of which we spoke earlier were to become clogged, this could be a possibility. This does not make for a great passenger experience. So, we need some sort of emergency pressure dump.

What if we have not enough pressure? Those FARs we talked about are regulatory. So if we don't have enough pressure - maybe the aircraft's intake was cracked, or we lost a window, or some other horrible catastrophe - we'd need to get the plane down to a lower altitude to be legal. The pilots and the passengers have supplemental oxygen in most cases (we'll talk about that in a moment...) but they only last for so long. In this case, an emergency descent down to an appropriate altitude is in order.

Those design considerations aside, pressurizing the cabin works great. Airliners do it all the time. Even Concorde did it up at 60,000 feet. Once we start getting much higher than that, it is difficult to impossible to get enough air to pressurize a large volume. Pressure suits are smaller than an entire cabin, which is why SR-71, U-2, and other high altitude pilots utilized them. True spacecraft don't have the opportunity to scavenge oxygen from the air, since there is no air!

With that said, it's high time to talk about ways that we can make our own luck, so to say, and carry oxygen with us.

\subsection{Supplemental Oxygen Sources}

\subsection{Oxygen Delivery}



\section{Operational Considerations}

