\chapter{High Altitude Operations}

In this chapter, we introduce high altitude operations.

Before we dive in, we must answer a couple of questions.

What are high altitude operations? Simply put, these are operations far outside the ``normal'' environment you might expect to find on a Standard Day at sea level. We're talking about altitudes that are not terribly comfortable for humans, and possibly altitudes that are so high above MSL that we won't find any terrain there at all. There are enough aeromedical factors and enough changes to aircraft performance that we need to consider these operations carefully.

Why do we care about high altitude operations? We might need to get high enough to cross some mountains and stay above terrain. But there is also the desire for speed. I'm not just talking about the speed limit of 250 knots below 10,000 feet - if we wish to go faster, we need to go higher. I'm talking about the fact that \emph{thinner air provides less drag}, which permits us to fly through the air more efficiently.

Mother Nature, of course, exists outside the realm of aviation. The challenges of high altitudes were present well before the first powered flights - just ask anyone who tried to climb Mount Everest in the 1800s. So, first, we must have a look at the atmosphere itself.

\section{The Atmosphere and Its Layers}

Recalling the basics from aviation weather, we know that the atmosphere is divided into several layers. Starting from the ground and going on up, we have the troposphere, the stratosphere, the mesosphere, the thermosphere, and the exosphere. All weather - and most of the earth's air! - reside in the troposphere. Flight operations rarely venture into the stratosphere (even more rarely now that Concorde is no longer flying).

So, when we say ``high altitude'', we mean ``higher than a short cross country flight but lower than Concorde''. We're talking altitudes below about 45,000 feet. If you ever manage to get above that, please let me know. (I'm looking at you, astronaut friends... you know who you are!)

\section{Density and Altitude}

In order to better talk about the atmosphere, can we all agree on a make believe, consistent, common, daresay, \emph{standard} atmosphere?

Well, apparently, we can. Aviators are all too familiar with the International Standard Atmosphere. This aviator only recently learned that when we say standard, we mean Standard, namely, International Standard (ISO) 2533 from 1975, which is identical to the ICAO Standard Atmosphere from -2 to 32 km, or up to about 100,000 feet.

Assuming you've paid the 198 CHF fee, or can otherwise access the standard, you would see an interesting thing happening. You would see the value of p, the pressure, and $\rho$, the density of the atmosphere, decrease as altitude increases - and rapidly at that. At sea level (zero elevation), pressure is the all too familiar 1013.25 hectopascals. At 3,000 meters or about 10,000 feet, we're down to 701.21 hectopascals - only 69\% of the pressure at sea level! That means 69\% as much air, 69\% as much oxygen available for breathing or combustion, 69\% as much drag.

Up at 10,000 meters or about 33,000 feet - typical airliner altitudes - we're down to 262 hectopascals. That's just over 25\% of the pressure at sea level.

This is a huge change! It's enough that we need to pay special attention to how we are powering our airplanes, how we are providing our bodies with oxygen, and how we are constructing our airplanes. It's enough that we need to take into account certain operational considerations.

\section{Propulsion}

\section{Aeromedical Factors}

\section{Operational Considerations}

