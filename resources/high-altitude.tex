\chapter{High Altitude Operations}

In this chapter, we introduce high altitude operations.

Before we dive in, we must answer a couple of questions.

What are high altitude operations? Simply put, these are operations far outside the ``normal'' environment you might expect to find on a Standard Day at sea level. We're talking about altitudes that are not terribly comfortable for humans, and possibly altitudes that are so high above MSL that we won't find any terrain there at all. There are enough aeromedical factors and enough changes to aircraft performance that we need to consider these operations carefully.

Why do we care about high altitude operations? We might need to get high enough to cross some mountains and stay above terrain. But there is also the desire for speed. I'm not just talking about the speed limit of 250 knots below 10,000 feet - if we wish to go faster, we need to go higher. I'm talking about the fact that \emph{thinner air provides less drag}, which permits us to fly through the air more efficiently.

Mother Nature, of course, exists outside the realm of aviation. The challenges of high altitudes were present well before the first powered flights - just ask anyone who tried to climb Mount Everest in the 1800s. So, first, we must have a look at the atmosphere itself.

\section{The Atmosphere and Its Layers}

Recalling the basics from aviation weather, we know that the atmosphere is divided into several layers. Starting from the ground and going on up, we have the troposphere, the stratosphere, the mesosphere, the thermosphere, and the exosphere. All weather - and most of the earth's air! - reside in the troposphere. Flight operations rarely venture into the stratosphere (even more rarely now that Concorde is no longer flying).

So, when we say ``high altitude'', we mean ``higher than a short cross country flight but lower than Concorde''. We're talking altitudes below about 45,000 feet. If you ever manage to get above that, please let me know. (I'm looking at you, astronaut friends... you know who you are!)

\section{Density and Altitude}

In order to better talk about the atmosphere, can we all agree on a make believe, consistent, common, daresay, \emph{standard} atmosphere?

Well, apparently, we can. Aviators are all too familiar with the International Standard Atmosphere. This aviator only recently learned that when we say standard, we mean Standard, namely, International Standard (ISO) 2533 from 1975, which is identical to the ICAO Standard Atmosphere from -2 to 32 km, or up to about 100,000 feet.

Assuming you've paid the 198 CHF fee, or can otherwise access the standard, you would see an interesting thing happening. You would see the value of p, the pressure, and $\rho$, the density of the atmosphere, decrease as altitude increases - and rapidly at that. At sea level (zero elevation), pressure is the all too familiar 1013.25 hectopascals. At 3,000 meters or about 10,000 feet, we're down to 701.21 hectopascals - only 69\% of the pressure at sea level! That means 69\% as much air, 69\% as much oxygen available for breathing or combustion, 69\% as much drag.

Up at 10,000 meters or about 33,000 feet - typical airliner altitudes - we're down to 262 hectopascals. That's just over 25\% of the pressure at sea level.

This is a huge change! It's enough that we need to pay special attention to how we are powering our airplanes, how we are providing our bodies with oxygen, and how we are constructing our airplanes. It's enough that we need to take into account certain operational considerations.

\section{Propulsion}

One of the problems with high altitude flight is that the air is so thin, there may not be enough oxygen for the (internal combustion) engine to run properly.

Recall that a normally aspirated aircraft with a piston engine has a service ceiling. That ceiling is defined as the altitude at which the airplane's climb rate slows to 100 feet per minute \cite{marchman}. That exact altitude will vary based on aircraft loading and how the conditions compare to a standard day. Famously, Concorde would continue to climb while underway, going higher as it got lighter. But the fact remains: for a piston airplane, we can only go so high.

Since the limitation is oxygen, and not fuel, there are a few ways that we could gain power at a higher altitude.

What if we had an electric aircraft? That would be great - it doesn't need oxygen at all. Today, in 2022, battery technology isn't quite there, nor are solar cells, outside from a very few research aircraft.

What if we simply had supplemental oxygen on board? Certainly, if we carried oxygen tanks, we could supplement the oxygen in ambient air to gain more power. Operationally, this is not a great idea, since we now have to deal with the added weight, complexity, and risk of an oxygen system, and it will not last very long. But air-breathing rockets do exactly this, so it's not a bad idea per se, simply not the right one.

What if we could grab more air from the atmosphere, possibly by compressing the air before it gets to the engine intake manifold? Now we're on the right track.

The first approach we might take is simply inserting a compressor between the outside air intake and the engine's intake manifold. That compressor could be a piston pump, similar to the vacuum pum in the airplane. But that's not terribly efficient: it generates a fair amount of waste heat. For this application, a turbine is a better idea. We could power this compressor with the aircraft's engine, most likely with a belt or a gear drive off the engine crankshaft. Such a contraption has a name: supercharger. It requires more power from the engine, and there are some losses due to the belt or gear drive mechanism, but it is a new positive up to a point: we can shove more oxygen into the engine, and raise our service ceiling.

Some clever person (who?) looked elsewhere on the engine and found another possible source of energy for driving this compressor. They realized that the exhaust gasses coming out of the engine were quite warm and under a decent amount of pressure. What if this energy could be captured? We can add a \emph{second} turbine in line with the exhaust pipe, and use the torque generated by that turbine to drive our intake compressor! As it turns out, this works better than a supercharger. We call this a ``turbocharger''. Sometimes we have more than one.

(some diagrams should go here)

Operating a supercharged or turbocharged engine has some key differences from operating a normally aspirated engine. One, we need to re-calibrate our manifold pressures. By this I mean, in normal operation, a piston engine's intake manifold would achieve a pressure no higher than ambient pressure, likely no more than 32 inches of mercury. But, a turbocharged (or supercharged) engine may be able to go above this. I say ``may'' because some turbos simply give sea-level pressure at higher altitudes (turbonormalized), whereas others can go well above sea level pressure. Regardless, the turbo lets the engine run harder than it could otherwise, which means increased heat and risk of engine damaged. There are often limitations on how long an engine may run on a particular power setting as a result.

The turbocharger (or supercharger) turbines are themselves metal parts that are subject to fatigue and thermal stresses. After flights, we want to give the turbines some time to cool down, lest we shut them off entirely and subject them to thermal shock.

Recall that, when we compress air, it gets quite warm. This is unfortunate, since we recall from the application of carburetor heat that warm air will lean our mixture. What if there were a way to compress the air, and then cool it to get it as dense as possible? There is: it is called an ``intercooler''. But, as this requires even more heat to be dissipated from the aircraft, the intercooler must be carefully placed so that it can dissipate heat effectively.

With a turbocharged engine in particular, the concept of ``exhaust gas temperature'' takes on a new meeting. Where are we measuring: before the exhaust turbine, or after? It makes more sense to measure before, and we call this the ``turbine inlet temperature'' or TIT. We need to manage this temperature carefully: we want the TIT to be as high as possible for efficient leaning of the engine, but if it is too high, it could melt the exhaust turbine.

These turbines need to spin quite fast - think 100,000 RPM - in order to be effective. It is challenging and expensive to create machinery that can operate at this speed.

All of this seems awfully complicated for a little bit of extra power. What if there was a better way?

There is. It's called the jet engine.

The jet engine is, at is simplest, a single turbine. It ingests air and compresses it. We inject fuel into the compressed air and ignite it, causing it to heat and expand. The exhaust heats and expands so much that it exerts a force on a turbine, which serves as thrust for the airplane. Much like the turbocharger, the jet engine grabs some energy from the exhaust system to power the input compression phase.

More specifically, this single arrangement is called a "turbojet". It is simple, powerful, and reliable. So long as fuel usage and noise are of no concern, this is the best powerplant. The military often doesn't care about fuel or noise so we see turbojets on plenty of military aircraft. If we care more about efficiency and noise than (potentially supersonic) speed, we can also attach a really fast-spinning propeller to this, and capture the propeller's thrust in a duct for maximum efficiency. This combination of a turbojet with a ducted fan is called a "turbofan" and is the most popular way of powering large aircraft.

Of course, the propulsion system allows the aircraft to operate at a higher altitude. But what about the pilot, flight crew, and passengers? Do we need to do anything special for them if we are flying at high altitudes?

\section{Aeromedical Factors}

Much like the airplane's engine, the human body needs a certain amount of oxygen to perform.

\subsection{Hypoxia}

\subsection{Diving and Flight}

\subsection{Humidity}

\section{Supplemental Oxygen Systems}

\section{Operational Considerations}

