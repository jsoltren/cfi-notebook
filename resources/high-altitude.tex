\chapter{High Altitude Operations}

In this chapter, we introduce high altitude operations.

Before we dive in, we must answer a couple of questions.

What are high altitude operations? Simply put, these are operations far outside the ``normal'' environment you might expect to find on a Standard Day at sea level. We're talking about altitudes that are not terribly comfortable for humans, and possibly altitudes that are so high above MSL that we won't find any terrain there at all. There are enough aeromedical factors and enough changes to aircraft performance that we need to consider these operations carefully.

Why do we care about high altitude operations? We might need to get high enough to cross some mountains and stay above terrain. But there is also the desire for speed. I'm not just talking about the speed limit of 250 knots below 10,000 feet - if we wish to go faster, we need to go higher. I'm talking about the fact that \emph{thinner air provides less drag}, which permits us to go faster more efficiently.

Mother Nature, of course, exists outside the realm of aviation. The challenges of high altitudes were present well before the first powered flights - just ask anyone who tried to climb Mount Everest in the 1800s. So, first, we must have a look at the atmosphere itself.

\section{The Atmosphere and its layers}

\section{Density and altitude}

\section{Propulsion}

\section{Aeromedical Factors}

\section{Operational Considerations}

