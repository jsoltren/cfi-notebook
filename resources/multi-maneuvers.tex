
\chapter{Maneuvers}

\section{Introduction to Maneuvers}

All maneuvers begin and end the same way. Begin by clearing the area, followed by a power change (if required),
and a GUMP-CS check. The maneuvers end by re-establishing slow cruise (20''/2400), and doing a final GUMP-CS
check for cruise configuration.

\begin{table}[H]
\centering
\begin{tabular}{|c|l|c|c|}
\hline
                    &                                                 & \textbf{Maneuver Entry} & \textbf{Resume Cruise} \\ \hline
                    & \textcolor{blue}{\textbf{Power}}                & As Required             & 20'' MP                \\ \hline
\textcolor{blue}{G} & \textcolor{blue}{\textbf{G}}as (fuel selectors) & On                      & On                     \\
                    & \textcolor{blue}{\textbf{G}}as (boost pumps)    & On                      & Off                    \\ \hline
\textcolor{blue}{U} & \textcolor{blue}{\textbf{U}}ndercarriage        & As Required             & Up                     \\ \hline
\textcolor{blue}{M} & \textcolor{blue}{\textbf{M}}ixtures             & Rich                    & Cruise Lean            \\ \hline
\textcolor{blue}{P} & \textcolor{blue}{\textbf{Propellers}}           & As Required             & 2400 RPM               \\ \hline
\textcolor{blue}{C} & \textcolor{blue}{\textbf{C}}owl Flaps           & As Required             & Closed                 \\ \hline
\textcolor{blue}{S} & \textcolor{blue}{\textbf{S}}eat Belts           & Secure                  & Secure                 \\ \hline
\end{tabular}
\end{table}

Airman Certification Standards tables in this chapter are taken from from FAA-S-ACS-7B, Commercial Pilot for Airplane Category Airman Certification Standards, November 2023, and  FAA-S-ACS-25, Flight Instructor for Airplane Category, Airman Certification Standards, November 2023 as appropriate.

%Private pilot multiengine checkrides are rare, and maneuvers performed to the commercial standard are more than satisfactory for the private checkride's more forgiving tolerances.

\newpage

\section{Slow Flight}
\subsection{Maneuver Checklist}

\textbf{Perform clearing turns prior to maneuver. At least 3000' AGL.}

Set up maneuver:

\begin{table}[H]
\centering
\begin{tabular}{|c|l|c|c|}
\hline
                    &                                                 & \textbf{Maneuver Entry} & \textbf{Resume Cruise} \\ \hline
                    & \textcolor{blue}{\textbf{Power}}                & \textbf{15'' MP}        & 20'' MP                \\ \hline
\textcolor{blue}{G} & \textcolor{blue}{\textbf{G}}as (fuel selectors) & On                      & On                     \\
                    & \textcolor{blue}{\textbf{G}}as (boost pumps)    & On                      & Off                    \\ \hline
\textcolor{blue}{U} & \textcolor{blue}{\textbf{U}}ndercarriage        & Down below 140 KIAS     & Up                     \\ \hline
\textcolor{blue}{M} & \textcolor{blue}{\textbf{M}}ixtures             & Rich                    & Cruise Lean            \\ \hline
\textcolor{blue}{P} & \textcolor{blue}{\textbf{Propellers}}           & 2400 RPM                & 2400 RPM               \\ \hline
\textcolor{blue}{C} & \textcolor{blue}{\textbf{C}}owl Flaps           & Open                    & Closed                 \\ \hline
\textcolor{blue}{S} & \textcolor{blue}{\textbf{S}}eat Belts           & Secure                  & Secure                 \\ \hline
\end{tabular}
\end{table}

Extend \underline{full flaps} when in the white arc of the airspeed indicator.

Maintain heading, altitude and an airspeed leading to the onset of the stall (stall horn intermittent, about 70 kts).
Pitch controls airspeed (trim full aft!) and power controls altitude (approximately \underline{\textbf{17''-18''}}).

Recovery:
\begin{itemize}[label={}]
\item \textbf{23''} Power
\item Flaps up above \textbf{71 KIAS}
\item Gear up above \textbf{85 KIAS}
\item Perform post-maneuver GUMP-CS
\end{itemize}

\newpage
\subsection{Airman Certification Standards: Maneuvering During Slow Flight}
\begin{table}[H]
\begin{tabular}%
  {>{\raggedleft\arraybackslash}p{0.15\linewidth}%
   >{\raggedright\arraybackslash}p{0.8\linewidth}%
  }
\textbf{Task}             & \textbf{A. Maneuvering During Slow Flight}                                                                                                                                                                                                        \\ \hline
\textit{References:}      & \textit{FAA-H-8083-2, FAA-H-8083-3, FAA-H-8083-25; POH/AFM}                                                                                                                                                                                       \\
\textbf{Objective:}       & To determine the applicant exhibits satisfactory knowledge, risk management, and skills associated with maneuvering during slow flight in cruise configuration.                                                                                   \\
\textit{\textbf{Note:}}   & \textit{See Appendix 2: Safety of Flight and Appendix 3: Aircraft, Equipment, and Operational Requirements \& Limitations for information related to this Task.}                                                                                  \\ \hline
\textbf{Knowledge:}       & The applicant demonstrates understanding of:                                                                                                                                                                                                      \\
\textit{CA.VII.A.K1}      & Aerodynamics associated with slow flight in various airplane configurations, including the relationship between angle of attack, airspeed, load factor, power setting, airplane weight and center of gravity, airplane attitude, and yaw effects. \\ \hline
\multicolumn{1}{l}{\textbf{\begin{tabular}[c]{@{}l@{}}Risk\\ Management:\end{tabular}}} & The applicant is able to identify, assess, and mitigate risk associated with:                                                                                                                                                                     \\
\textit{CA.VII.A.R1}      & Inadvertent slow flight and flight with a stall warning, which could lead to loss of control.                                                                                                                                                     \\
\textit{CA.VII.A.R2}      & Range and limitations of stall warning indicators (e.g., aircraft buffet, stall horn, etc.).                                                                                                                                                      \\
\textit{CA.VII.A.R3}      & Uncoordinated flight.                                                                                                                                                                                                                             \\
\textit{CA.VII.A.R4}      & Effect of environmental elements on airplane performance (e.g., turbulence, microbursts, and high-density altitude).                                                                                                                              \\
\textit{CA.VII.A.R5}      & Collision hazards.                                                                                                                                                                                                                                \\
\textit{CA.VII.A.R6}      & Distractions, task prioritization, loss of situational awareness, or disorientation.                                                                                                                                                              \\ \hline
\textbf{Skills:}          & The applicant exhibits the skill to:                                                                                                                                                                                                              \\
\textit{CA.VII.A.S1}      & Clear the area.                                                                                                                                                                                                                                   \\
\textit{CA.VII.A.S2}      & Select an entry altitude that allows the Task to be completed no lower than 1,500 feet above ground level (AGL) (ASEL, ASES) or 3,000 feet AGL (AMEL, AMES).                                                                                      \\
\textit{CA.VII.A.S3}               & Establish and maintain an airspeed at which any further increase in angle of attack, increase in load factor, or reduction in power, would result in a stall warning (e.g., aircraft buffet, stall horn, etc.).                                   \\
\textit{CA.VII.A.S4}               & Accomplish coordinated straight-and-level flight, turns, climbs, and descents with the aircraft configured as specified by the evaluator without a stall warning (e.g., aircraft buffet, stall horn, etc.).                                       \\
\textit{CA.VII.A.S5}               & Maintain the specified altitude, ±50 feet; specified heading, ±10°; airspeed, +5/-0 knots; and specified angle of bank, ±5°.                                                                                                                     
\end{tabular}
\end{table}

\newpage

\section{Power-Off Stall}
\subsection{Maneuver Checklist}

\textbf{Perform clearing turns prior to maneuver. At least 3000' AGL.}

Set up maneuver:

\begin{table}[H]
\centering
\begin{tabular}{|c|l|c|c|}
\hline
                    &                                                 & \textbf{Maneuver Entry} & \textbf{Resume Cruise} \\ \hline
                    & \textcolor{blue}{\textbf{Power}}                & \textbf{15'' MP}        & 20'' MP                \\ \hline
\textcolor{blue}{G} & \textcolor{blue}{\textbf{G}}as (fuel selectors) & On                      & On                     \\
                    & \textcolor{blue}{\textbf{G}}as (boost pumps)    & On                      & Off                    \\ \hline
\textcolor{blue}{U} & \textcolor{blue}{\textbf{U}}ndercarriage        & Down below 140 KIAS     & Up                     \\ \hline
\textcolor{blue}{M} & \textcolor{blue}{\textbf{M}}ixtures             & Rich                    & Cruise Lean            \\ \hline
\textcolor{blue}{P} & \textcolor{blue}{\textbf{Propellers}}           & Full forward below 90 KIAS & 2400 RPM            \\ \hline
\textcolor{blue}{C} & \textcolor{blue}{\textbf{C}}owl Flaps           & Open                    & Closed                 \\ \hline
\textcolor{blue}{S} & \textcolor{blue}{\textbf{S}}eat Belts           & Secure                  & Secure                 \\ \hline
\end{tabular}
\end{table}

\textbf{Landing configuration.} Extend \textbf{full flaps} when in the white arc of the airspeed indicator.

Establish landing descent attitude: about 500 FPM @ 85 KIAS.

Pitch to maintain altitude and heading until stall warning or buffet: glare shield to the horizon. 

Recovery:
\begin{itemize}[label={}]
\item \textbf{Full Power}, at least 65 KIAS (pitch for straight and level flight: VSI = 0)
\item Flaps up above \textbf{71 KIAS}
\item Gear up above \textbf{85 KIAS}
\item Perform post-maneuver GUMP-CS
\end{itemize}

\subsection{Airman Certification Standards: Power-Off Stalls}

See table on following page.

\begin{table}[]
\begin{tabular}%
  {>{\raggedleft\arraybackslash}p{0.15\linewidth}%
   >{\raggedright\arraybackslash}p{0.8\linewidth}%
  }
\textbf{Task}                                                       & \textbf{B. Power-Off Stalls}                                                                                                                                                                                                                 \\ \hline
\textit{References:}                                                & \textit{AC 61-67; FAA-H-8083-2, FAA-H-8083-3, FAA-H-8083-25; POH/AFM}                                                                                                                                                                        \\
\textbf{Objective:}                                                 & To determine the applicant exhibits satisfactory knowledge, risk management, and skills associated with power-off stalls.                                                                                                                    \\
\textit{\textbf{Note:}}                                             & \textit{See Appendix 2: Safety of Flight and Appendix 3: Aircraft, Equipment, and Operational Requirements \& Limitations for information related to this Task.}                                                                             \\ \hline
\textbf{Knowledge:}                                                 & The applicant demonstrates understanding of:                                                                                                                                                                                                 \\
\textit{CA.VII.B.K1}                                                & Aerodynamics associated with stalls in various airplane configurations, including the relationship between angle of attack, airspeed, load factor, power setting, airplane weight and center of gravity, airplane attitude, and yaw effects. \\
\textit{CA.VII.B.K2}                                                & Stall characteristics as they relate to airplane design, and recognition impending stall and full stall indications using sight, sound, or feel.                                                                                             \\
\textit{CA.VII.B.K3}                                                & Factors and situations that can lead to a power-off stall and actions that can be taken to prevent it.                                                                                                                                       \\
\textit{CA.VII.B.K4}                                                & Fundamentals of stall recovery.                                                                                                                                                                                                              \\ \hline
\textbf{Risk Mgmt:} & The applicant is able to identify, assess, and mitigate risk associated with:                                                                                                                                                                \\
\textit{CA.VII.B.R1}                                                & Factors and situations that could lead to an inadvertent power-off stall, spin, and loss of control.                                                                                                                                         \\
\textit{CA.VII.B.R2}                                                & Range and limitations of stall warning indicators (e.g., aircraft buffet, stall horn, etc.).                                                                                                                                                 \\
\textit{CA.VII.B.R3}                                                & Stall warning(s) during normal operations.                                                                                                                                                                                                   \\
\textit{CA.VII.B.R4}                                                & Stall recovery procedure.                                                                                                                                                                                                                    \\
\textit{CA.VII.B.R5}                                                & Secondary stalls, accelerated stalls, and cross-control stalls.                                                                                                                                                                              \\
\textit{CA.VII.B.R6}                                                & Effect of environmental elements on airplane performance related to power-off stalls (e.g., turbulence, microbursts, and high-density altitude).                                                                                             \\
\textit{CA.VII.B.R7}                                                & Collision hazards.                                                                                                                                                                                                                           \\
\textit{CA.VII.B.R8}                                                & Distractions, task prioritization, loss of situational awareness, or disorientation.                                                                                                                                                         \\ \hline
\textbf{Skills:}                                                    & The applicant exhibits the skill to:                                                                                                                                                                                                         \\
\textit{CA.VII.B.S1}                                                & Clear the area.                                                                                                                                                                                                                              \\
\textit{CA.VII.B.S2}                                                & Select an entry altitude that allows the Task to be completed no lower than [...] 3,000 feet AGL (AMEL, AMES).                                                                                 \\
\textit{CA.VII.B.S3}                                                & Configure the airplane in the approach or landing configuration, as specified by the evaluator, and maintain coordinated flight throughout the maneuver.                                                                                     \\
\textit{CA.VII.B.S4}                                                & Establish a stabilized descent.                                                                                                                                                                                                              \\
\textit{CA.VII.B.S5}                                                & Transition smoothly from the approach or landing attitude to a pitch attitude that induces a stall.                                                                                                                                          \\
\textit{CA.VII.B.S6}                                                & Maintain a specified heading, ±10° if in straight flight; maintain a specified angle of bank not to exceed 20°, ±5° if in turning flight, until an impending or full stall occurs, as specified by the evaluator.                            \\
\textit{CA.VII.B.S7}                                                & Acknowledge the cues at the first indication of a stall (e.g., aircraft buffet, stall horn, etc.).                                                                                                                                           \\
\textit{CA.VII.B.S8}                                                & Recover at the first indication of a stall or after a full stall has occurred, as specified by the evaluator.                                                                                                                                \\
\textit{CA.VII.B.S9}                                                & Configure the airplane as recommended by the manufacturer, and accelerate to best angle of climb speed ($V_X$) or best rate of climb speed ($V_Y$).                                                                                                \\
\textit{CA.VII.B.S10}                                               & Return to the altitude, heading, and airspeed specified by the evaluator.                                                                                                                                                                   
\end{tabular}
\end{table}

\newpage

\section{Power-On Stall}
\subsection{Maneuver Checklist}

\textbf{Perform clearing turns prior to maneuver. At least 3000' AGL.}

Set up maneuver:

\begin{table}[H]
\centering
\begin{tabular}{|c|l|c|c|}
\hline
                    &                                                 & \textbf{Maneuver Entry} & \textbf{Resume Cruise} \\ \hline
                    & \textcolor{blue}{\textbf{Power}}                & \textbf{12'' MP}        & 20'' MP                \\ \hline
\textcolor{blue}{G} & \textcolor{blue}{\textbf{G}}as (fuel selectors) & On                      & On                     \\
                    & \textcolor{blue}{\textbf{G}}as (boost pumps)    & On                      & Off                    \\ \hline
\textcolor{blue}{U} & \textcolor{blue}{\textbf{U}}ndercarriage        & Up                      & Up                     \\ \hline
\textcolor{blue}{M} & \textcolor{blue}{\textbf{M}}ixtures             & Rich                    & Cruise Lean            \\ \hline
\textcolor{blue}{P} & \textcolor{blue}{\textbf{Propellers}}           & Full forward below 90 KIAS & 2400 RPM            \\ \hline
\textcolor{blue}{C} & \textcolor{blue}{\textbf{C}}owl Flaps           & Open                    & Closed                 \\ \hline
\textcolor{blue}{S} & \textcolor{blue}{\textbf{S}}eat Belts           & Secure                  & Secure                 \\ \hline
\end{tabular}
\end{table}

\textbf{Take off configuration.} At \textbf{85 kts}, increase power to \textbf{20''}.

Pitch \textbf{20\degree{} UP}.

Recovery:
\begin{itemize}[label={}]
\item \textbf{FULL POWER}
\item Pitch Level
\item \textbf{100 kts} reduce power to \textbf{20''/2400}
\item Perform post-maneuver GUMP-CS
\end{itemize}

\subsection{Airman Certification Standards: Power-On Stalls}

See table on following page.

\begin{table}[]
\begin{tabular}%
  {>{\raggedleft\arraybackslash}p{0.15\linewidth}%
   >{\raggedright\arraybackslash}p{0.8\linewidth}%
  }
\textbf{Task}           & \textbf{C. Power-On Stalls}                                                                                                                                                                                                                  \\ \hline
\textit{References:}    & \textit{AC 61-67; FAA-H-8083-2, FAA-H-8083-3, FAA-H-8083-25; POH/AFM}                                                                                                                                                                        \\
\textbf{Objective:}     & To determine the applicant exhibits satisfactory knowledge, risk management, and skills associated with power-on stalls.                                                                                                                     \\
\textit{\textbf{Note:}} & \textit{See Appendix 2: Safety of Flight and Appendix 3: Aircraft, Equipment, and Operational Requirements \& Limitations for information related to this Task.}                                                                             \\ \hline
\textbf{Knowledge:}     & The applicant demonstrates understanding of:                                                                                                                                                                                                 \\
\textit{CA.VII.C.K1}    & Aerodynamics associated with stalls in various airplane configurations, including the relationship between angle of attack, airspeed, load factor, power setting, airplane weight and center of gravity, airplane attitude, and yaw effects. \\
\textit{CA.VII.C.K2}    & Stall characteristics as they relate to airplane design, and recognition impending stall and full stall indications using sight, sound, or feel.                                                                                             \\
\textit{CA.VII.C.K3}    & Factors and situations that can lead to a power-on stall and actions that can be taken to prevent it.                                                                                                                                        \\
\textit{CA.VII.C.K4}    & Fundamentals of stall recovery.                                                                                                                                                                                                              \\ \hline
\textbf{Risk Mgmt:}     & The applicant is able to identify, assess, and mitigate risk associated with:                                                                                                                                                                \\
\textit{CA.VII.C.R1}    & Factors and situations that could lead to an inadvertent power-on stall, spin, and loss of control.                                                                                                                                          \\
\textit{CA.VII.C.R2}    & Range and limitations of stall warning indicators (e.g., aircraft buffet, stall horn, etc.).                                                                                                                                                 \\
\textit{CA.VII.C.R3}    & Stall warning(s) during normal operations.                                                                                                                                                                                                   \\
\textit{CA.VII.C.R4}    & Stall recovery procedure.                                                                                                                                                                                                                    \\
\textit{CA.VII.C.R5}    & Secondary stalls, accelerated stalls, elevator trim stalls, and cross-control stalls.                                                                                                                                                        \\
\textit{CA.VII.C.R6}    & Effect of environmental elements on airplane performance related to power-on stalls (e.g., turbulence, microbursts, and high-density altitude).                                                                                              \\
\textit{CA.VII.C.R7}    & Collision hazards.                                                                                                                                                                                                                           \\
\textit{CA.VII.C.R8}    & Distractions, task prioritization, loss of situational awareness, or disorientation.                                                                                                                                                         \\ \hline
\textbf{Skills:}        & The applicant exhibits the skill to:                                                                                                                                                                                                         \\
\textit{CA.VII.C.S1}    & Clear the area.                                                                                                                                                                                                                              \\
\textit{CA.VII.C.S2}    & Select an entry altitude that allows the Task to be completed no lower than {[}...{]} 3,000 feet AGL (AMEL, AMES).                                                                                                                           \\
\textit{CA.VII.C.S3}    & Establish the takeoff, departure, or cruise configuration, as specified by the evaluator, and maintain coordinated flight throughout the maneuver.                                                                                           \\
\textit{CA.VII.C.S4}    & Set power to no less than 65 percent power.                                                                                                                                                                                                  \\
\textit{CA.VII.C.S5}    & Transition smoothly from the takeoff or departure attitude to the pitch attitude that induces a stall.                                                                                                                                       \\
\textit{CA.VII.C.S6}    & Maintain a specified heading ±10° if in straight flight; maintain a specified angle of bank not to exceed 20°, ±10° if in turning flight, until an impending or full stall is reached, as specified by the evaluator.                        \\
\textit{CA.VII.C.S7}    & Acknowledge the cues at the first indication of a stall (e.g., aircraft buffet, stall horn, etc.).                                                                                                                                           \\
\textit{CA.VII.C.S8}    & Recover at the first indication of a stall or after a full stall has occurred, as specified by the evaluator.                                                                                                                                \\
\textit{CA.VII.C.S9}    & Configure the airplane as recommended by the manufacturer, and accelerate to best angle of climb speed ($V_X$) or best rate of climb speed ($V_Y$).                                                                                                \\
\textit{CA.VII.C.S10}   & Return to the altitude, heading, and airspeed specified by the evaluator.                                                                                                                                                                   
\end{tabular}
\end{table}

\newpage

\section{Steep Turns}
\subsection{Maneuver Checklist}

\textbf{Perform clearing turns prior to maneuver. At least 3000' AGL.}

\textbf{Line up with a prominent outside landmark.}

Set up maneuver:

\begin{table}[H]
\centering
\begin{tabular}{|c|l|c|c|}
\hline
                    &                                                 & \textbf{Maneuver Entry} & \textbf{Resume Cruise} \\ \hline
                    & \textcolor{blue}{\textbf{Power}}                & \textbf{20'' MP}        & 20'' MP                \\ \hline
\textcolor{blue}{G} & \textcolor{blue}{\textbf{G}}as (fuel selectors) & On                      & On                     \\
                    & \textcolor{blue}{\textbf{G}}as (boost pumps)    & On                      & Off                    \\ \hline
\textcolor{blue}{U} & \textcolor{blue}{\textbf{U}}ndercarriage        & Up                      & Up                     \\ \hline
\textcolor{blue}{M} & \textcolor{blue}{\textbf{M}}ixtures             & Lean                    & Cruise Lean            \\ \hline
\textcolor{blue}{P} & \textcolor{blue}{\textbf{Propellers}}           & 2400 RPM                & 2400 RPM               \\ \hline
\textcolor{blue}{C} & \textcolor{blue}{\textbf{C}}owl Flaps           & Closed                  & Closed                 \\ \hline
\textcolor{blue}{S} & \textcolor{blue}{\textbf{S}}eat Belts           & Secure                  & Secure                 \\ \hline
\end{tabular}
\end{table}

Roll into a \textbf{50\degree{} bank}, +/- 5\degree{}. Maintain altitude and roll out on chosen landmark.
Maneuver consists of one turn to the left followed by one turn to the right.

\emph{Memory aid: Lead roll out by half of bank angle, 25\degree{}.}

Recovery:
\begin{itemize}[label={}]
\item Roll straight and level on landmark
\item Perform post-maneuver GUMP-CS
\end{itemize}

\newpage 

\subsection{Airman Certification Standards: Steep Turns}

\begin{table}[H]
\begin{tabular}%
  {>{\raggedleft\arraybackslash}p{0.15\linewidth}%
   >{\raggedright\arraybackslash}p{0.8\linewidth}%
  }
\textbf{Task}                                                       & \textbf{A. Steep Turns}                                                                                                          \\ \hline
\textit{References:}                                                & \textit{FAA-H-8083-2, FAA-H-8083-3, FAA-H-8083-25; POH/AFM}                                                                      \\
\textbf{Objective:}                                                 & To determine the applicant exhibits satisfactory knowledge, risk management, and skills associated with steep turns.             \\
\textit{\textbf{Note:}}                                             & \textit{See Appendix 3: Aircraft, Equipment, and Operational Requirements \& Limitations for information related to this Task.}  \\ \hline
\textbf{Knowledge:}                                                 & The applicant demonstrates understanding of:                                                                                     \\
\textit{CA.V.A.K1}                                                  & How to conduct a proper steep turn.                                                                                              \\
\textit{CA.V.A.K2}                                                  & Aerodynamics associated with steep turns, including:                                                                             \\
\textit{CA.V.A.K2a}                                                 & a. Maintaining coordinated flight                                                                                                \\
\textit{CA.V.A.K2b}                                                 & b. Overbanking tendencies                                                                                                        \\
\textit{CA.V.A.K2c}                                                 & c. Maneuvering speed, including the impact of weight changes                                                                     \\
\textit{CA.V.A.K2d}                                                 & d. Load factor and accelerated stalls                                                                                            \\
\textit{CA.V.A.K2e}                                                 & e. Rate and radius of turn                                                                                                       \\ \hline
\textbf{\begin{tabular}[c]{@{}r@{}}Risk\\ Management:\end{tabular}} & The applicant is able to identify, assess, and mitigate risk associated with:                                                    \\
\textit{CA.V.A.R1}                                                  & Division of attention between aircraft control and orientation.                                                                  \\
\textit{CA.V.A.R2}                                                  & Collision hazards.                                                                                                               \\
\textit{CA.V.A.R3}                                                  & Low altitude maneuvering, including stall, spin, or controlled flight into terrain (CFIT).                                       \\
\textit{CA.V.A.R4}                                                  & Distractions, task prioritization, loss of situational awareness, or disorientation.                                             \\
\textit{CA.V.A.R5}                                                  & Uncoordinated flight.                                                                                                            \\ \hline
\textbf{Skills:}                                                    & The applicant exhibits the skill to:                                                                                             \\
\textit{CA.V.A.S1}                                                  & Clear the area.                                                                                                                  \\
\textit{CA.V.A.S2}                                                  & Establish the manufacturer's recommended airspeed; or if one is not available, an airspeed not to exceed maneuvering speed ($V_A$). \\
\textit{CA.V.A.S3}                                                  & Roll into a coordinated 360° steep turn with approximately a 50° bank.                                                           \\
\textit{CA.V.A.S4}                                                  & Perform the Task in the opposite direction.                                                                                      \\
\textit{CA.V.A.S5}                                                  & Maintain the entry altitude ±100 feet, airspeed ±10 knots, bank ±5°, and roll out on the entry heading ±10°.                    
\end{tabular}
\end{table}

\newpage

\section{Accelerated Stalls}
\subsection{Maneuver Checklist}

\textbf{Perform clearing turns prior to maneuver. At least 3000' AGL.}

Set up maneuver:

\begin{table}[H]
\centering
\begin{tabular}{|c|l|c|c|}
\hline
                    &                                                 & \textbf{Maneuver Entry} & \textbf{Resume Cruise} \\ \hline
                    & \textcolor{blue}{\textbf{Power}}                & \textbf{15'' MP}        & 20'' MP                \\ \hline
\textcolor{blue}{G} & \textcolor{blue}{\textbf{G}}as (fuel selectors) & On                      & On                     \\
                    & \textcolor{blue}{\textbf{G}}as (boost pumps)    & On                      & Off                    \\ \hline
\textcolor{blue}{U} & \textcolor{blue}{\textbf{U}}ndercarriage        & Up                      & Up                     \\ \hline
\textcolor{blue}{M} & \textcolor{blue}{\textbf{M}}ixtures             & Lean                    & Cruise Lean            \\ \hline
\textcolor{blue}{P} & \textcolor{blue}{\textbf{Propellers}}           & 2400 RPM                & 2400 RPM               \\ \hline
\textcolor{blue}{C} & \textcolor{blue}{\textbf{C}}owl Flaps           & Closed                  & Closed                 \\ \hline
\textcolor{blue}{S} & \textcolor{blue}{\textbf{S}}eat Belts           & Secure                  & Secure                 \\ \hline
\end{tabular}
\end{table}

Roll into 45\degree{} bank and smoothly increase back pressure until stall warning or buffet.

Recovery:
\begin{itemize}[label={}]
\item \textbf{20'' MP}
\item Pitch \& Roll to straight \& level
\item \textbf{100 kts} reduce power to \textbf{20''/2400}
\item Perform post-maneuver GUMP-CS
\end{itemize}
\newpage
\subsection{Airman Certification Standards: Accelerated Stalls}

\begin{table}[H]
\begin{tabular}%
  {>{\raggedleft\arraybackslash}p{0.15\linewidth}%
   >{\raggedright\arraybackslash}p{0.8\linewidth}%
  }
\textbf{Task}                                                       & \textbf{D. Accelerated Stalls}                                                                                                                                                                                                                           \\ \hline
\textit{References:}                                                & \textit{AC 61-67; FAA-H-8083-2, FAA-H-8083-3, FAA-H-8083-25; POH/AFM}                                                                                                                                                                                    \\
\textbf{Objective:}                                                 & To determine the applicant exhibits satisfactory knowledge, risk management, and skills associated with accelerated stalls (power-on or power-off).                                                                                                      \\
\textit{\textbf{Note:}}                                             & \textit{See Appendix 2: Safety of Flight and Appendix 3: Aircraft, Equipment, and Operational Requirements \& Limitations for information related to this Task.}                                                                                         \\ \hline
\textbf{Knowledge:}                                                 & The applicant demonstrates understanding of:                                                                                                                                                                                                             \\
\textit{CA.VII.D.K1}                                                & Aerodynamics associated with accelerated stalls in various airplane configurations, including the relationship between angle of attack, airspeed, load factor, power setting, airplane weight and center of gravity, airplane attitude, and yaw effects. \\
\textit{CA.VII.D.K2}                                                & Stall characteristics as they relate to airplane design, and recognition impending stall and full stall indications using sight, sound, or feel.                                                                                                         \\
\textit{CA.VII.D.K3}                                                & Factors leading to an accelerated stall and preventive actions.                                                                                                                                                                                          \\
\textit{CA.VII.D.K4}                                                & Fundamentals of stall recovery.                                                                                                                                                                                                                          \\ \hline
\textbf{\begin{tabular}[c]{@{}r@{}}Risk\\ Management:\end{tabular}} & The applicant is able to identify, assess, and mitigate risk associated with:                                                                                                                                                                            \\
\textit{CA.VII.D.R1}                                                & Factors and situations that could lead to an inadvertent accelerated stall, spin, and loss of control.                                                                                                                                                   \\
\textit{CA.VII.D.R2}                                                & Range and limitations of stall warning indicators (e.g., aircraft buffet, stall horn, etc.).                                                                                                                                                             \\
\textit{CA.VII.D.R3}                                                & Stall warning(s) during normal operations.                                                                                                                                                                                                               \\
\textit{CA.VII.D.R4}                                                & Stall recovery procedure.                                                                                                                                                                                                                                \\
\textit{CA.VII.D.R5}                                                & Secondary stalls, cross-control stalls, and spins.                                                                                                                                                                                                       \\
\textit{CA.VII.D.R6}                                                & Effect of environmental elements on airplane performance related to accelerated stalls (e.g., turbulence, microbursts, and high-density altitude).                                                                                                       \\
\textit{CA.VII.D.R7}                                                & Collision hazards.                                                                                                                                                                                                                                       \\
\textit{CA.VII.D.R8}                                                & Distractions, task prioritization, loss of situational awareness, or disorientation.                                                                                                                                                                     \\ \hline
\textbf{Skills:}                                                    & The applicant exhibits the skill to:                                                                                                                                                                                                                     \\
\textit{CA.VII.D.S1}                                                & Clear the area.                                                                                                                                                                                                                                          \\
\textit{CA.VII.D.S2}                                                & Select an entry altitude that allows the Task to be completed no lower than 3,000 feet above ground level (AGL).                                                                                                                                         \\
\textit{CA.VII.D.S3}                                                & Establish the configuration as specified by the evaluator.                                                                                                                                                                                               \\
\textit{CA.VII.D.S4}                                                & Set power appropriate for the configuration, such that the airspeed does not exceed the maneuvering speed ($V_A$) or any other applicable Pilot's Operating Handbook (POH)/Airplane Flight Manual (AFM) limitation.                                         \\
\textit{CA.VII.D.S5}                                                & Establish and maintain a coordinated turn in a 45° bank, increasing elevator back pressure smoothly and firmly until an impending stall is reached.                                                                                                      \\
\textit{CA.VII.D.S6}                                                & Acknowledge the cues at the first indication of a stall (e.g., aircraft buffet, stall horn, etc.).                                                                                                                                                       \\
\textit{CA.VII.D.S7}                                                & Execute a stall recovery in accordance with procedures set forth in the Pilot's Operating Handbook (POH)/Flight Manual (FM).                                                                                                                             \\
\textit{CA.VII.D.S8}                                                & Configure the airplane as recommended by the manufacturer, and accelerate to best angle of climb speed ($V_X$) or best rate of climb speed ($V_Y$).                                                                                                            \\
\textit{CA.VII.D.S9}                                                & Return to the altitude, heading, and airspeed specified by the evaluator.                                                                                                                                                                               
\end{tabular}
\end{table}

\newpage

\section{Emergency Descent}
\subsection{Maneuver Checklist}

\textbf{Perform clearing turns prior to maneuver. At least 3000' AGL.}

Set up maneuver:

\begin{table}[H]
\centering
\begin{tabular}{|c|l|c|c|}
\hline
                    &                                                 & \textbf{Maneuver Entry} & \textbf{Resume Cruise} \\ \hline
                    & \textcolor{blue}{\textbf{Power}}                & \textbf{11'' MP}        & 20'' MP                \\ \hline
\textcolor{blue}{G} & \textcolor{blue}{\textbf{G}}as (fuel selectors) & On                      & On                     \\
                    & \textcolor{blue}{\textbf{G}}as (boost pumps)    & On                      & Off                    \\ \hline
\textcolor{blue}{U} & \textcolor{blue}{\textbf{U}}ndercarriage        & Down below 140 KIAS     & Up                     \\ \hline
\textcolor{blue}{M} & \textcolor{blue}{\textbf{M}}ixtures             & Rich                    & Cruise Lean            \\ \hline
\textcolor{blue}{P} & \textcolor{blue}{\textbf{Propellers}}           & 2400 RPM                & 2400 RPM               \\ \hline
\textcolor{blue}{C} & \textcolor{blue}{\textbf{C}}owl Flaps           & Closed                  & Closed                 \\ \hline
\textcolor{blue}{S} & \textcolor{blue}{\textbf{S}}eat Belts           & Secure                  & Secure                 \\ \hline
\end{tabular}
\end{table}

Close throttles.

Smoothly roll into \textbf{30\degree{}} - 45\degree{} bank.

Smoothly pitch nose \textbf{20\degree{}} down. \textbf{Do not exceed 140 KIAS with gear down.}

Recovery:
\begin{itemize}[label={}]
\item Smoothly roll \& pitch to straight \& level (lead level off by \~{}200’).
\item \textbf{100 kts} reduce power to \textbf{20''/2400}
\item Perform post-maneuver GUMP-CS
\end{itemize}

\newpage
\subsection{Airman Certification Standards: Emergency Descent}

\begin{table}[H]
\begin{tabular}%
  {>{\raggedleft\arraybackslash}p{0.15\linewidth}%
   >{\raggedright\arraybackslash}p{0.8\linewidth}%
  }
\textbf{Task}                                                       & \textbf{A. Emergency Descent}                                                                                                                                                                                                       \\ \hline
\textit{References:}                                                & \textit{FAA-H-8083-2, FAA-H-8083-3, FAA-H-8083-25; POH/AFM}                                                                                                                                                                         \\
\textbf{Objective:}                                                 & To determine the applicant exhibits satisfactory knowledge, risk management, and skills associated with emergency descent.                                                                                                          \\
\textit{\textbf{Note:}}                                             & \textit{See Appendix 2: Safety of Flight.}                                                                                                                                                                                          \\ \hline
\textbf{Knowledge:}                                                 & The applicant demonstrates understanding of:                                                                                                                                                                                        \\
\textit{CA.IX.A.K1}                                                 & Situations that would require an emergency descent (e.g., depressurization, smoke, or engine fire).                                                                                                                                 \\
\textit{CA.IX.A.K2}                                                 & Immediate action items and emergency procedures.                                                                                                                                                                                    \\
\textit{CA.IX.A.K3}                                                 & Airspeed, including airspeed limitations.                                                                                                                                                                                           \\
\textit{CA.IX.A.K4}                                                 & Aircraft performance and limitations.                                                                                                                                                                                               \\ \hline
\textbf{\begin{tabular}[c]{@{}r@{}}Risk\\ Management:\end{tabular}} & The applicant is able to identify, assess, and mitigate risk associated with:                                                                                                                                                       \\
\textit{CA.IX.A.R1}                                                 & Altitude, wind, terrain, obstructions, gliding distance, and available landing distance considerations.                                                                                                                             \\
\textit{CA.IX.A.R2}                                                 & Collision hazards.                                                                                                                                                                                                                  \\
\textit{CA.IX.A.R3}                                                 & Configuring the airplane.                                                                                                                                                                                                           \\
\textit{CA.IX.A.R4}                                                 & Distractions, task prioritization, loss of situational awareness, or disorientation.                                                                                                                                                \\ \hline
\textbf{Skills:}                                                    & The applicant exhibits the skill to:                                                                                                                                                                                                \\
\textit{CA.VII.D.S1}                                                & Clear the area.                                                                                                                                                                                                                     \\
\textit{CA.VII.D.S2}                                                & Establish and maintain the appropriate airspeed and configuration appropriate to the scenario specified by the evaluator and as covered in Pilot's Operating Handbook (POH)/Airplane Flight Manual (AFM) for the emergency descent. \\
\textit{CA.VII.D.S3}                                                & Maintain orientation, divide attention appropriately, and plan and execute a smooth recovery.                                                                                                                                       \\
\textit{CA.VII.D.S4}                                                & Use bank angle between 30° and 45° to maintain positive load factors during the descent.                                                                                                                                            \\
\textit{CA.VII.D.S5}                                                & Maintain appropriate airspeed +0/-10 knots, and level off at a specified altitude ±100 feet.                                                                                                                                        \\
\textit{CA.VII.D.S6}                                                & Complete the appropriate checklist(s).                                                                                                                                                                                              \\
\textit{CA.VII.D.S7}                                                & Use single-pilot resource management (SRM) or crew resource management (CRM), as appropriate.                                                                                                                                      
\end{tabular}
\end{table}

\newpage

\section{Loss of Directional Control Demonstration (\vmc Demo)}
\subsection{Maneuver Checklist}

\textbf{Perform clearing turns prior to maneuver. At least 3000' AGL.}

Set up maneuver:

\begin{table}[H]
\centering
\begin{tabular}{|c|l|c|c|}
\hline
                    &                                                 & \textbf{Maneuver Entry} & \textbf{Resume Cruise} \\ \hline
                    & \textcolor{blue}{\textbf{Power}}                & \textbf{12'' MP}        & 20'' MP                \\ \hline
\textcolor{blue}{G} & \textcolor{blue}{\textbf{G}}as (fuel selectors) & On                      & On                     \\
                    & \textcolor{blue}{\textbf{G}}as (boost pumps)    & On                      & Off                    \\ \hline
\textcolor{blue}{U} & \textcolor{blue}{\textbf{U}}ndercarriage        & Up                      & Up                     \\ \hline
\textcolor{blue}{M} & \textcolor{blue}{\textbf{M}}ixtures             & Rich                    & Cruise Lean            \\ \hline
\textcolor{blue}{P} & \textcolor{blue}{\textbf{Propellers}}           & Full forward below 90 KIAS & 2400 RPM            \\ \hline
\textcolor{blue}{C} & \textcolor{blue}{\textbf{C}}owl Flaps           & L - Closed / R - Open   & Closed                 \\ \hline
\textcolor{blue}{S} & \textcolor{blue}{\textbf{S}}eat Belts           & Secure                  & Secure                 \\ \hline
\end{tabular}
\end{table}

Left throttle - leave 12''

Right throttle - move full forward

\textbf{At 85 KIAS:} Pitch up to horizon, losing 1 knot of airspeed per second.

Upon loss of directional control, stall warning, buffet or full rudder travel:

\textbf{IMMEDIATELY BEGIN RECOVERY!}

Recovery:
\begin{itemize}[label={}]
\item Simultaneously \textbf{lower pitch} to $\frac{1}{2}$ ground – $\frac{1}{2}$ sky while \textbf{reducing power} on the good engine and neutralizing the rudder.
\item After regaining control, ease in full power on the good engine and reestablish\\\textbf{\textcolor{blue}{85 KIAS BLUE LINE}}.
\item Return both engines to 20''/2400.
\item Perform post-maneuver GUMP-CS
\end{itemize}

\subsection{Airman Certification Standards: \vmc Demonstration}

See table on following page.

\begin{table}[]
\begin{tabular}%
  {>{\raggedleft\arraybackslash}p{0.15\linewidth}%
   >{\raggedright\arraybackslash}p{0.8\linewidth}%
  }
\textbf{Task}                                                       & \textbf{B. \vmc Demonstration (AMEL, AMES)}                                                                                                                                                                                               \\ \hline
\textit{References:}                                                & \textit{FAA-H-8083-2, FAA-H-8083-3, FAA-H-8083-25; FAA-P-8740-66; POH/AFM}                                                                                                                                                               \\
\textbf{Objective:}                                                 & To determine the applicant exhibits satisfactory knowledge, risk management, and skills associated with \vmc demonstration.                                                                                                               \\
\textit{\textbf{Note:}}                                             & \textit{See Appendix 2: Safety of Flight and Appendix 3: Aircraft, Equipment, and Operational Requirements \& Limitations for information related to this Task.}                                                                         \\ \hline
\textbf{Knowledge:}                                                 & The applicant demonstrates understanding of:                                                                                                                                                                                             \\
\textit{CA.X.B.K1}                                                  & Factors affecting \vmc and how \vmc differs from stall speed ($V_S$).                                                                                                                                                                         \\
\textit{CA.X.B.K2}                                                  & \vmc (red line), \vyse (blue line), and safe single-engine speed ($V_{SSE}$).                                                                                                                                                                   \\
\textit{CA.X.B.K3}                                                  & Cause of loss of directional control at airspeeds below \vmc.                                                                                                                                                                             \\
\textit{CA.X.B.K4}                                                  & Proper procedures for maneuver entry and safe recovery.                                                                                                                                                                                  \\ \hline
\textbf{\begin{tabular}[c]{@{}r@{}}Risk\\ Management:\end{tabular}} & The applicant is able to identify, assess, and mitigate risk associated with:                                                                                                                                                            \\
\textit{CA.X.B.R1}                                                  & Configuring the airplane.                                                                                                                                                                                                                \\
\textit{CA.X.B.R2}                                                  & Maneuvering with one engine inoperative.                                                                                                                                                                                                 \\
\textit{CA.X.B.R3}                                                  & Distractions, task prioritization, loss of situational awareness, or disorientation.                                                                                                                                                     \\ \hline
\textbf{Skills:}                                                    & The applicant exhibits the skill to:                                                                                                                                                                                                     \\
\textit{CA.X.B.S1}                                                  & Configure the airplane in accordance with the manufacturer’s recommendations, in the absence of the manufacturer’s recommendations, then at safe single-engine speed ($V_{SSE}$/\vyse), as appropriate, and:                                   \\
\textit{CA.X.B.S1a}                                                 & a. Landing gear retracted                                                                                                                                                                                                                \\
\textit{CA.X.B.S1b}                                                 & b. Flaps set for takeoff                                                                                                                                                                                                                 \\
\textit{CA.X.B.S1c}                                                 & c. Cowl flaps set for takeoff                                                                                                                                                                                                            \\
\textit{CA.X.B.S1d}                                                 & d. Trim set for takeoff                                                                                                                                                                                                                  \\
\textit{CA.X.B.S1e}                                                 & e. Propellers set for high revolutions per minute (rpm)                                                                                                                                                                                  \\
\textit{CA.X.B.S1f}                                                 & f. Power on critical engine reduced to idle and propeller windmilling                                                                                                                                                                    \\
\textit{CA.X.B.S1g}                                                 & g. Power on operating engine set to takeoff or maximum available power                                                                                                                                                                   \\
\textit{CA.X.B.S2}                                                  & Establish a single-engine climb attitude with the airspeed at approximately 10 knots above $V_{SSE}$.                                                                                                                                         \\
\textit{CA.X.B.S3}                                                  & Establish a bank angle not to exceed 5° toward the operating engine, as required for best performance and controllability.                                                                                                               \\
\textit{CA.X.B.S4}                                                  & Increase the pitch attitude slowly to reduce the airspeed at approximately 1 knot per second while applying increased rudder pressure as needed to maintain directional control.                                                         \\
\textit{CA.X.B.S5}                                                  & Recognize and recover at the first indication of loss of directional control, stall warning, or buffet.                                                                                                                                  \\
\textit{CA.X.B.S6}                                                  & Recover promptly by simultaneously reducing power sufficiently on the operating engine, decreasing the angle of attack as necessary to regain airspeed and directional control, and without adding power on the simulated failed engine. \\
\textit{CA.X.B.S7}                                                  & Recover within 20° of entry heading.                                                                                                                                                                                                     \\
\textit{CA.X.B.S8}                                                  & Advance power smoothly on the operating engine and accelerate to $V_{SSE}$/\vyse, as appropriate, ±5 knots during recovery.                                                                                                                   
\end{tabular}
\end{table}

\newpage

\section{Effects of Configuration Demonstration (Drag Demo)}
\subsection{Maneuver Checklist}

\textbf{Perform clearing turns prior to maneuver. At least 3000' AGL.}

Set up maneuver:

\begin{table}[H]
\centering
\begin{tabular}{|c|l|c|c|}
\hline
                    &                                                 & \textbf{Maneuver Entry} & \textbf{Resume Cruise} \\ \hline
                    & \textcolor{blue}{\textbf{Power}}                & \textbf{15'' MP}        & 20'' MP                \\ \hline
\textcolor{blue}{G} & \textcolor{blue}{\textbf{G}}as (fuel selectors) & On                      & On                     \\
                    & \textcolor{blue}{\textbf{G}}as (boost pumps)    & On                      & Off                    \\ \hline
\textcolor{blue}{U} & \textcolor{blue}{\textbf{U}}ndercarriage        & Up (for now)            & Up                     \\ \hline
\textcolor{blue}{M} & \textcolor{blue}{\textbf{M}}ixtures             & Rich                    & Cruise Lean            \\ \hline
\textcolor{blue}{P} & \textcolor{blue}{\textbf{Propellers}}           & 2400 RPM (for now)      & 2400 RPM               \\ \hline
\textcolor{blue}{C} & \textcolor{blue}{\textbf{C}}owl Flaps           & R - Closed / L - Open   & Closed                 \\ \hline
\textcolor{blue}{S} & \textcolor{blue}{\textbf{S}}eat Belts           & Secure                  & Secure                 \\ \hline
\end{tabular}
\end{table}

\emph{Memory aid: Airspeed $\pm$ 5. Gear, flaps, propeller.}

Right throttle \textbf{10''} / Right propeller \textbf{detent}.

Left propeller at 2400 / Left throttle – \textbf{20''}

Establish \vyse \textbf{85 kts \textcolor{blue}{BLUE LINE}} and note VSI – use as a reference value

Pitch for \textbf{80 kts}, stabilize and note VSI (reference -100)

Pitch again for \textbf{85 kts} and stabilize VSI at reference

Pitch for \textbf{90 kts}, stabilize, and note VSI (reference -100)

Pitch again for \textbf{85 kts} and stabilize VSI at reference

Maintain \textbf{85 kts} for the rest of the demonstration.

Lower undercarriage, stabilize and note VSI (reference - 250) \textbf{Gear –250}

Lower flaps, stabilize and note VSI (reference - 600) \textbf{Flaps –350}

Windmill propeller (throttle to idle, Prop full forward)\\
- stabilize and note VSI (reference - 900) \textbf{Propeller –300}

Raise gear, stabilize and note VSI (reference - 650)

Recovery:
\begin{itemize}[label={}]

\item \textbf{Power 20''/2400}
\item Flaps Up above \textbf{71 KIAS}
\item Gear Up above \textbf{85 KIAS}
\item Perform post-maneuver GUMP-CS
\end{itemize}

\subsection{Airman Certification Standards: Drag Demo}

%See table on following page.

\begin{table}[]
\begin{tabular}%
  {>{\raggedleft\arraybackslash}p{0.15\linewidth}%
   >{\raggedright\arraybackslash}p{0.8\linewidth}%
  }
\textbf{Task}                                                       & \textbf{C. Demonstration of Effects of Various Airspeeds and Configurations during Engine Inoperative Performance (AMEL/AMES)}                                                                                                                      \\ \hline
\textit{References:}                                                & \textit{FAA-H-8083-2, FAA-H-8083-3, FAA-H-8083-9, FAA-H-8083-25; FAA-P-8740-66; POH/AFM}                                                                                                                                                                \\
\textbf{Objective:}                                                 & To determine the applicant understands the effects of various airspeeds and configurations during engine inoperative performance, can apply that knowledge, manage associated risks, demonstrate appropriate skills, and provide effective instruction. \\
\textit{\textbf{Note:}}                                             & \textit{See Appendix 2: Safety of Flight and Appendix 3: Aircraft, Equipment, and Operational Requirements \& Limitations for information related to this Task.}                                                                                        \\ \hline
\textbf{Knowledge:}                                                 & The applicant demonstrates instructional knowledge by describing and explaining:                                                                                                                                                                        \\
\textit{AI.XIII.C.K1}                                               & Purpose for and elements of demonstration of effects of various airspeeds and configurations during engine inoperative performance.                                                                                                                     \\
\textit{AI.XIII.C.K2}                                               & Selection of appropriate altitude for the demonstration.                                                                                                                                                                                                \\
\textit{AI.XIII.C.K3}                                               & Proper entry procedure to include pitch attitude, bank attitude, and airspeed.                                                                                                                                                                          \\
\textit{AI.XIII.C.K4}                                               & Effects on performance of airspeed changes at, above, and below \vyse.                                                                                                                                                                                   \\
\textit{AI.XIII.C.K5}                                               & Effects on performance of various configurations including:                                                                                                                                                                                             \\
\textit{AI.XIII.C.K5a}                                              & a. Landing gear extended                                                                                                                                                                                                                                \\
\textit{AI.XIII.C.K5b}                                              & b. Wing flaps extended                                                                                                                                                                                                                                  \\
\textit{AI.XIII.C.K5c}                                              & c. Landing gear and wing flaps extended                                                                                                                                                                                                                 \\
\textit{AI.XIII.C.K5d}                                              & d. Windmilling propeller on the inoperative engine                                                                                                                                                                                                      \\
\textit{AI.XIII.C.K6}                                               & Airspeed control throughout the demonstration.                                                                                                                                                                                                          \\
\textit{AI.XIII.C.K7}                                               & Smooth control technique and coordination throughout the demonstration.                                                                                                                                                                                 \\
\textit{AI.XIII.C.K8}                                               & Common errors related to this Task.                                                                                                                                                                                                                     \\ \hline
\textbf{\begin{tabular}[c]{@{}r@{}}Risk\\ Management:\end{tabular}} & The applicant explains and teaches how to identify and manage risk associated with:                                                                                                                                                                     \\
\textit{AI.XIII.C.R1}                                               & Altitude selection.                                                                                                                                                                                                                                     \\
\textit{AI.XIII.C.R2}                                               & Entry and recovery procedures.                                                                                                                                                                                                                          \\
\textit{AI.XIII.C.R3}                                               & Loss of control or stall.                                                                                                                                                                                                                               \\
\textit{AI.XIII.C.R4}                                               & Configuring the airplane.                                                                                                                                                                                                                               \\
\textit{AI.XIII.C.R5}                                               & Collision hazards.                                                                                                                                                                                                                                      \\
\textit{AI.XIII.C.R6}                                               & Distractions, task prioritization, loss of situational awareness, or disorientation.                                                                                                                                                                    \\ \hline
\textbf{Skills:}                                                    & The applicant demonstrates and simultaneously explains how to:                                                                                                                                                                                          \\
\textit{AI.XIII.C.S1}                                               & Demonstrate, describe, and explain effects of various airspeeds and configurations during engine inoperative performance.                                                                                                                               \\
\textit{AI.XIII.C.S2}                                               & Demonstrate smooth control inputs when transitioning between various airspeeds and configurations, which include:                                                                                                                                       \\
\textit{AI.XIII.C.S2a}                                              & a. Landing gear extended                                                                                                                                                                                                                                \\
\textit{AI.XIII.C.S2b}                                              & b. Wing flaps extended                                                                                                                                                                                                                                  \\
\textit{AI.XIII.C.S2c}                                              & c. Landing gear and wing flaps extended                                                                                                                                                                                                                 \\
\textit{AI.XIII.C.S2d}                                              & d. Windmilling propeller on the inoperative engine                                                                                                                                                                                                      \\
\textit{AI.XIII.C.S3}                                               & Maintain appropriate airspeed, attitude, and altitude combinations for the various configurations.                                                                                                                                                      \\
\textit{AI.XIII.C.S4}                                               & Return to normal cruise flight at the altitude and heading specified by the evaluator.                                                                                                                                                                  \\
\textit{AI.XIII.C.S5}                                               & Analyze and correct common errors related to this Task.                                                                                                                                                                                                
\end{tabular}
\end{table}
