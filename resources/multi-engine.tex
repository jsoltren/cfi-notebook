% Define some commonly used terms.
% We did include xcolor, didn't we?
\newcommand{\vyse}{\textcolor{blue}{$V_{yse}$ }}
\newcommand{\vmc}{\textcolor{red}{$V_{mc}$ }}
\newcommand{\vxse}{$V_{xse}$ }

\chapter*{Multi Engine Airplanes:\\An Introduction}

\begin{center}
\includegraphics[width=0.9\linewidth]{pca-logo}
\end{center}

Do you want to fly the ``big iron''? Are you looking for more performance, capability,
and reliability? Do you enjoy engine management (and maintenance) so much that you want to
double the fun? If so, the multi engine airplane is for you.

This chapter is meant to prepare the prospective multi engine pilot or instructor for the
fun that is to come.

The material in this section is based upon FAA resources, as well as the Multiengine Manual
that Pilot's Choice Aviation (PCA) in Georgetown, TX distributes to incoming multi engine students.
This document could not exist without Beth Ann Jenkins and Stephanie Fernihough, the authors of the
2017 PCA Multiengine Manual and some notes beyond it.

The Multiengine Manual was reformatted into \LaTeX{} in Spring of 2025. At that time, it was
updated to reflect eight years of students successfully using this document to become multi
engine pilots. It also reflects the latest FAA testing documents as of this writing.

This Manual focused on the Beechcraft Duchess BE76, the primary multi engine trainer used at PCA.

\section*{Disclaimer}

The information contained in this publication is subject to change.

Aeronautical information, regulations, and aircraft information change regularly, therefore those relevant
publications should be referred to for any critical information.

The information in this manual is to be utilized for training purposes only. Always
refer to your aircraft's POH, AFM, and other certified documentation before flight!

\chapter{Single Engine Aerodymanics}

%Keep it above \vyse or bad things happen.
%Keep it above \vmc or really bad things happen.

With a few exceptions, flying a multiengine airplane in normal operations is similar to flying a
complex single. Most of the differences between single-engine and multi-engine flying relate to
emergency situations. Specifically, we are concerned with the airplane’s flight characteristics
when only one engine is fully operating.

The discussion to follow will focus on two key elements of multi-engine flying: performance and controllability.

Note: \emph{performance} and \emph{controllability} are complementary. As one increases, the other decreases, and vice versa.

\section{The Engine-Inoperative Condition}

\subsection{Asymmetric Thrust}

Engines on conventional twins are mounted on the wings. Unlike a single-engine airplane, the engine thrust is not
directed along the longitudinal axis of the airplane. Rather, each engine’s thrust produces a moment that attempts to
yaw the airplane around its vertical axis. On the Duchess (one counter-rotating prop) when both engines are
producing equal thrust, these moments balance each other out and the net thrust has no yawing moment. When one
engine is at reduced or zero thrust, there is a net yawing moment that will lead to a loss of directional control if not
counteracted.

Just like in a single, yawing moments (such as propeller left-turning tendencies in a climb) are counteracted with
rudder. When an engine fails in a multi-engine airplane, the yaw that occurs must be balanced out with enough
rudder pressure to keep the airplane straight. Rudder effectiveness is a function of airspeed – more air flowing over
the rudder airfoil gives it the ability to produce more horizontal lift.

\subsection{Accelerated Slipstream}

Because the engines on a conventional twin are wing-mounted, additional lift is produced by the accelerated
slipstream of the propeller wash over the wing surface. The loss of thrust on one wing results in a loss of lift on that
wing which produces an imbalance of lift between the two wings, leading to a rolling moment toward the
inoperative engine. This rolling tendency must be counteracted with aileron deflection.

\subsection{Summary}

Because of the above listed factors (asymmetric thrust and accelerated slipstream), both produced by the operating
engine, there is a tendency for the airplane to \emph{both} roll \emph{and} yaw into the inoperative engine!

\section{Engine-Inoperative Performance}

\subsection{Loss of Horsepower}

A common misconception is that with one engine out, a twin will have half the climb performance
that it would with both engines. In reality, for aircraft with a maximum gross weight of less than
6,000 pounds, there is no
requirement that they be capable of level flight or climb for \emph{any} weight or flight condition! The only requirement is
that the rate of climb or descent be determined. Many light twins are not capable of holding altitude with one engine.

The Duchess has two 180-HP engines for a total of 360 HP, and requires about 140 HP to maintain level flight.
Losing one engine drastically cuts the horsepower available for climb performance:

\begin{table}[h]
\centering
\begin{tabular}{cl}
\textbf{360}   & \textbf{total HP available}            \\
\textbf{(140)} & \textbf{HP for level flight}           \\ \hline
\textbf{220}   & \textbf{HP left for climb performance} \\
\textbf{(180)} & \textbf{HP -- loss of an engine}       \\ \hline
\textbf{40}    & \textbf{HP now available for climb}
\end{tabular}
\caption{Single engine performance for the Duchess.}
\end{table}

This means we now have only approximately 20\% (40/220) climb performance remaining. In addition,
it should be
stressed that the airplane must be cleaned up to climb. Anything that creates drag will
require additional horsepower
and will decrease the airplane’s climb performance.

Further, realize that 180 HP is the rated horsepower for sea-level standard conditions. Depending on density altitude
(pressure altitude and temperature) effective horsepower may be less than 180 HP. This means that you 
\emph{may not be able to maintain altitude with only one engine}. Maintaining \vyse (blue line) will give you a best rate of climb or the
least rate of descent.

\subsection{\vyse (\textcolor{blue}{Blue Line})}


\vyse is the maximum rate of climb (or minimum rate of sink) airspeed for a single-engine configuration.
It represents
the maximum lift over drag ratio ($L/D_{max}$) with one engine operating, and may be likened to the best glide speed in a
single-engine airplane. At slower airspeeds, induced drag becomes more prominent. At faster speeds, parasite drag
becomes more prominent.

\vyse is the minimum speed to use during all phases of flight and is to be exceeded until committed to land on short
final. \vyse is the minimum speed above which you can commit to a continued takeoff. \vyse is the minimum speed to
use during emergencies involving an engine failure. \vyse is marked by a \textcolor{blue}{Blue Line} on the airspeed indicator (85
KIAS on the Duchess).

\subsection{Drag Factors}

With one engine inoperative, several factors will determine whether or not you'll be able to maintain altitude, climb
or descend. These drag factors increase the horsepower required for level flight, and eat into the excess horsepower
which could be used for climb. All figures are approximate and will vary with density altitude:

\vbox{%
\begin{enumerate}
    \item Not at \vyse -- High or low by 5 knots: \textbf{100 fpm descent}
    \item Gear Down: \textbf{250 fpm descent}
    \item Full Flaps: \textbf{350 fpm descent} (flaps @ 20 = 150 fpm descent)
    \item Critical engine windmilling: \textbf{300 fpm descent}
\end{enumerate}}

Single engine goarounds may be impossible and \textbf{shall not be attempted with flap settings beyond 20 degrees}.

Each twin has a single engine service ceiling and an absolute single engine ceiling:

\begin{itemize}
\item The \textbf{single-engine service ceiling} (Duchess: 6000 ft @ ISA) is the maximum \emph{density altitude} the airplane can sustain a 50 fpm climb with max power on the good engine in the clean configuration.
\item The \textbf{single-engine absolute ceiling} (Duchess: approximately 7800 ft @ ISA) is the maximum \emph{density altitude} the airplane can maintain on one engine with max power in the clean configuration. This is also the altitude where \vyse
and \vxse meet.
\end{itemize}

\subsection{Engine-Inoperative Controllability}

In a single-engine airplane, keeping the aircraft under control (avoiding a stall) is critical. Even if performance is
below that required to maintain level flight, we accept a descent and a controlled landing rather than try to hold off
the descent and get into a stall/spin situation. While stalls are a concern in multi-engine aircraft, another important
consideration is the possible loss of directional control if airspeed is not managed correctly.

With all this in mind, it can be said that the battle for controllability is one between engine and rudder. Anything
that increases the difference in thrust between the two engines will decrease controllability, and anything that makes
the rudder more able to counteract the thrust difference will increase controllability.

\subsection{Critical Engine and Critical Engine Factors}

The critical engine is the engine whose failure most adversely affects the performance and controllability of the
airplane. In general, one of the engines will have a larger yaw moment and the airspeed needs to be higher in order
to balance it out. When the airplane has counter-rotating props (such as the Duchess) there is no critical engine. On
most twins, both propellers rotate clockwise when viewed from the cockpit. On these aircraft, the left engine is
critical. The reasons for this are explained below.

The following discussion assumes a conventional light twin, with two clockwise-rotating propellers. On such
airplanes, the critical engine is the left engine, because the left-turning tendencies of the right engine add to its
asymmetric thrust. The left turning tendencies are discussed below. (PAST)

\subsubsection{P factor}

The descending blade produces more thrust than the ascending blade. The descending blade on the right engine has
a longer moment arm (A2) than on the left engine (A1). This produces greater asymmetric thrust when the right
engine is operating than when the left engine is operating.

\begin{center}
\includegraphics[width=0.9\linewidth]{pfactor}
\end{center}

\subsubsection{Accelerated Slipstream}

The loss of induced airflow created by the propeller over the dead engine wing results in a
loss of lift on that wing. This loss of lift causes a roll towards the dead engine and will
require additional aileron deflection into the operating engine. Due to P-factor, the
accelerated slipstream of the right engine has a longer moment-arm (A2) than the left
engine (A1) because the descending (greater-thrust) propeller blade is outboard.

\begin{center}
\includegraphics[width=0.5\linewidth]{accel1}
\end{center}

\begin{center}
\includegraphics[width=0.5\linewidth]{accel2}
\end{center}

\subsubsection{Spiraling Slipstream}

Both engines produce spiraling slipstreams, but the left engine’s spiraling slipstream is directed towards the rudder, making it more effective. The right engine’s spiraling slipstream is directed away from the rudder. In the event of
left engine failure, the rudder becomes less effective due to the loss of the critical engine’s spiraling slipstream.
Therefore, with critical engine failure maintaining directional control requires more rudder authority.

\begin{center}
\includegraphics[width=0.5\linewidth]{spiraling}
\end{center}

\subsubsection{Torque}

Torque is the opposite reaction to the clockwise turning of the propellers. Both engines produce a rolling tendency
to the left. With the right engine operating (critical engine inoperative), torque adds to the yaw/roll produced by
asymmetric thrust. With the left engine operating, torque counteracts the yaw/roll produced by asymmetric thrust.

\begin{center}
\includegraphics[width=0.9\linewidth]{torque}
\end{center}

\subsection{$V_{MC}$}

\vmc is defined as the minimum speed at which you can maintain directional control
with ithe sudden loss of the critical engine. The actual speed at which you will
lose directional control will vary depending on conditions on a
given day. The \emph{published \vmc airspeed} is dictated by a set of conditions in
14 CFR (FAR) 23.149. \vmc is marked with a \textbf{\textcolor{red}{red line}} on the airspeed indicator.
For the Duchess, this airspeed is \textcolor{red}{65 KIAS}.

On August 30, 2017, a substantial rewrite of 14 CFR (FAR) 23 went into effect. This
rewrite eliminated 23.149 from the current regulations. However, it remains the
certification basis for many of the light piston twin airplanes we fly today. For
light twins certificated under the new Part 23, such as the Tecnam P2006, we would
see similar requirements under 14 CFR (FAR) 23.2xxx. Since this document focuses on
the Duchess, we restrict our discussion to the ``old'' 14 CFR (FAR) 23, which is
archived on ecfr.gov.

\subsection{Determination of $V_{MC}$}

Under the ``old'' 14 CFR Part 23, manufactures of multiengine aircraft were required to demonstrate and publish a \vmc
(minimum control airspeed) under the following specific conditions, which are either set for Standardization (S), or are the Worst Case (W):

\vbox{%
\begin{enumerate}
    \item \textbf{Standard Day} (S) (15\degree C and 29.92" at sea level)
    \item \textbf{Maximum sea level takeoff weight} (S)
    \item \textbf{The most rearward allowable center of gravity} (W)
    \item \textbf{The critical engine failed} (W) and the propeller is
        \begin{enumerate}
            \item Windmilling, or
            \item Feathered, if the aircraft has an auto feather system (rare in light twins)
        \end{enumerate}
    \item \textbf{Takeoff at maximum available power on the good engine} (W)
    \item \textbf{Landing gear up} (W)
    \item \textbf{Flaps in the takeoff position} (S)
    \item \textbf{No more than 5\degree{} of bank into the good engine} (S)
\end{enumerate}}

A further factor is the pilot ability, Part 23 states that recovery from loss of directional
control at \vmc should not require more than average pilot technique and the recovery should
be accomplished within 20\degree{} of original heading.

Apart from the 5\degree{} bank, max gross weight, standard day, and flaps, the remaining factors
are all worst case, and they lead to the highest \vmc to be published by the manufacturer
(65 KIAS for the Duchess), or are related to the takeoff scenario.

When considering \vmc, realize that lower \vmc is better. Anything that lowers \vmc
will increase controllability at lower airspeed, giving more margin for error in
single-engine operation.

\subsubsection{Conditions that the FAA Requires for Published \vmc (14 CFR 23.149)}

What follows is a list of conditions required to be met by manufacturers in
determining the published \vmc value for certification. Understanding the
regulatory criteria is important for understanding how existing conditions and
aircraft control influence single-engine controllability.

The conditions required for certification represent the worst-case scenario
for controllability: an engine failure shortly after takeoff, with the
aircraft in the climb-out configuration. However, not every condition required for
certification is necessarily worst-case (W); some conditions are specified primarily
for standardization purposes (S).

\subsubsection{Standard conditions (ISA: 15\degree C and 29.92” Hg) (Standardization)}

\vmc decreases with increase density altitude. Any condition that decreases
power on the operating engine such as increased altitude, low air density or
high temperature will in turn mean less thrust, which creates less yaw, so \vmc
will decrease. The opposite is also true for any condition that increases power,
such as lower altitude, high density or low temperature, which will increase \vmc.

\emph{Memory aid: Hot = Good}

\subsubsection{Maximum sea-level power on operating engine (Worst case)}

Maximum power on the good engine increases \vmc due to increased asymmetric thrust.

\subsubsection{Critical engine windmilling or auto-feathered if installed (Worst case)}

The windmilling propeller creates drag, which is asymmetric. Therefore, more rudder
authority will be required to offset this asymmetric drag. \vmc is higher
with a windmilling propeller on the inoperative engine.

\subsubsection{Landing gear retracted (Worst case)}

The gear and gear doors extended tend to act like keels on a boat and resist rolling and yawing tendencies by
shifting the center of gravity down the vertical axis of the airplane. Additionally, on a tricycle-gear airplane, the
main gear are located aft of the center of gravity and produce \emph{stabilizing drag} when extended, like a drag chute
would. \vmc is \emph{lower with gear down}.

\subsubsection{Flaps in the takeoff configuration (Standardization)}

A number of considerations determine the relationship between flap setting and \vmc. With flaps extended a lesser
angle of attack is necessary to produce the same amount of lift. Therefore, P-factor is less as well as yaw.
Additionally, flaps increase drag aft of the C.G., providing a stabilizing effect. However, deploying flaps creates
additional lift on the wing with the operating engine since lift increases with the same airspeed. Therefore it is not
straightforward to say that \vmc changes one way or the other with flaps deployed, and this relationship may vary
depending on the airplane.

The Duchess procedures call for flaps fully retracted for takeoff.

\subsubsection{Maximum 5\degree{} bank into good engine (Standardization)}

The maximum bank allowed by the regulations for \vmc determination is five degrees. Any sideslip toward the good
engine increases controllability due to increased rudder effectiveness – the sideslip results in weathervaning
tendencies toward the operating engine. Likewise, sideslip toward the inoperative engine decreases controllability
by introducing a weathervaning moment away from the operating engine. Specifying a maximum of five degrees
limits the manufacturers to a realistic bank angle.

It should be remembered that a bank angle of three degrees towards the good engine with the ball 1/2 off-center
results in minimum drag and maximum climb. Five degrees of bank towards the good engine actually results in a
sideslip toward the good engine, increasing drag but increasing controllability.

\subsubsection{Maximum gross weight (Standardization)}

\vmc is determined at max gross weight. Primarily this is a reference point for standardization purposes.

When the airplane is banked, a sideslip occurs because a component of weight is acting along the wing (similar to
the idea of a wing-down crosswind approach). The spanwise component of weight and sideslip is greater at a higher
weight than a lower weight. Because of this, when the airplane is banked into the operative engine beyond the
zero-sideslip angle, \vmc increases with the decrease in weight, and vice versa.

\emph{Memory aid: Weight increases side slip effectiveness and lowers \vmc.}

\subsubsection{Most adverse CG (usually aft legal CG limit) (Worst case)}

As the CG moves aft, the moment arm between the rudder and C.G. is shortened, producing less leverage for the
rudder. The further aft the C.G., the more rudder authority is required to offset the asymmetric thrust, requiring
greater airspeed. \vmc is higher with an aft C.G.

\subsubsection{Ground effect negligible (Worst case)}

In ground effect there is a reduction in induced drag, so if an engine failure should occur while in ground effect a
lower than normal angle of attack would be required to create the same amount of lift as when out of ground effect.
A lower angle of attack would decrease the effect of P-factor, reduce yaw, and lower \vmc. Operating out of ground
effect results in a higher \vmc than in ground effect.

\begin{center}
\includegraphics[width=0.9\linewidth]{vmc-conditions}
\end{center}

\emph{Memory aid: Any increase in yaw or decrease in rudder authority \textbf{increases} \vmc!}

\emph{Memory aid: In the least favorable (aft) CG, the arm to the rudder is reduced, so the rudder is less effective.}

\section{Altitude vs. $V_{MC}$ and Stall Speed}

As density altitude increases, \vmc will decrease due to less power output from the operating engine (you'll lose
directional control at a slower airspeed). Indicated stall speed remains relatively constant for all density altitudes.
Thus, it is easily possible for \vmc to be lower than stall speed. When this happens a possible spin could develop
during \vmc demonstrations or during other single-engine operation, real or simulated.

If loss of directional control occurs during single-engine operation, \textbf{IMMEDIATELY} reduce power on the good
engine and lower the nose to regain airspeed.

\begin{center}
\includegraphics[width=0.8\linewidth]{vmc-stall}
\end{center}

\section{The Zero-Sideslip Condition}

As previously stated, a twin with only one engine operating must counteract the yaw and roll produced by
asymmetric thrust with rudder and aileron. The asymmetric thrust and the \textbf{horizontal lift} produced by the rudder
result in a net sideslip toward the \textbf{inoperative} engine when the wings of the airplane are held level,
causing a turn \emph{away} from the operating engine. This sideslip increases drag and degrades performance
just as a sideslip would in a single.

To overcome this sideslip, the airplane must be banked into the operative engine. The sideslip caused by the bank
angle and ball half-centered cancels out the sideslip created by the engine and rudder resulting in a
\textbf{zero-sideslip condition}. The zero-sideslip condition reduces
drag and therefore improves climb performance (or minimizes rate
of descent). In the Duchess, the bank angle that results in zero sideslip is approximately three degrees.

\begin{center}
\includegraphics[width=0.6\linewidth]{zero-sideslip}
\end{center}

\emph{Memory aid: Heavier means good engine is less effective, which reduces \vmc.}

\emph{Memory aid: Tail induces centripetal force which turns the airplane. So, you have to counteract that turn
with a bank angle of \emph{3\degree}}.

\chapter{Limitations}

\section{Duchess Speeds}


\begin{table}[h]
\centering
\begin{tabular}{lll}
\textcolor{red}{\textbf{$V_{MC}$}}      & \textbf{65}      & Minimum control speed.                                           \\
\textbf{$V_{S0}$}                       & \textbf{60}      & Stall: landing configuration.                                    \\
\textbf{$V_{S1}$}                       & \textbf{70}      & Stall: clean configuration.                                      \\
\textbf{$V_{XSE}$}                      & \textbf{85}      & Best rate of climb, one engine.                                  \\
\textcolor{blue}{\textbf{$V_{YSE}$}}    & \textbf{85}      & Best angle of climb, one engine.                                 \\ \hline
\multicolumn{1}{|l}{\textbf{$V_{SSE}$}} & \textbf{71}      & \multicolumn{1}{l|}{Intentional one engine inoperative speed.}   \\ \hline
\textbf{$V_A$}                          & \textbf{132}     & Maneuvering speed (at Max gross weight).                         \\
\textbf{$V_{LO}$}                       & \textbf{140/112} & Landing gear operating extend/retract.                           \\
\textbf{$V_{LE}$}                       & \textbf{140}     & Max landing gear extended.                                       \\
\textbf{$V_{FE}$}                       & \textbf{120/110} & Max flaps extended 20/35.                                        \\
\textbf{$V_{NO}$}                       & \textbf{154}     & Max structural cruising speed.                                   \\
\textbf{$V_{NE}$}                       & \textbf{194}     & Never exceed speed.                                              \\
\textbf{$V_R$}                          & \textbf{71}      & Rotation speed. For training purposes, $V_R$ is increased to 80. \\
\textbf{$V_X$}                          & \textbf{71}      & Best angle of climb.                                             \\
\textbf{$V_Y$}                          & \textbf{85}      & Best rate of climb.                                              \\
\textbf{}                               & \textbf{140}     & Emergency descent. \emph{Do not exceed!}                         \\
\textbf{}                               & \textbf{100}     & Emergency landing gear extension (free fall), maximum.           \\
\textbf{}                               & \textbf{95}      & Best glide: 12:1 with propellers feathered.                      \\
\textbf{}                               & \textbf{25}      & Maximum demonstrated crosswind component.
\end{tabular}
\end{table}

\section{Weight Limits}

\begin{table}[h]
\centering
\begin{tabular}{ll}
\textbf{Maximum Ramp Weight}              & 3916 Pounds \\
\textbf{Maximum Takeoff Weight}           & 3900 Pounds \\
\textbf{Maximum Landing Weight}           & 3900 Pounds \\
\textbf{Maximum Zero Fuel Weight}         & 3500 Pounds \\
\textbf{Maximum Baggage Compartment Load} & 200 Pounds 
\end{tabular}
\end{table}

\chapter{Emergency Procedures}

\textbf{KNOW PROCEDURES BY MEMORY!}

\section{Engine Failure in Flight (above 3000' AGL)}

\subsection{Procedures}

\textbf{Maintain directional control!}

\textbf{ALL AVAILABLE POWER:}
\begin{enumerate}
    \item \textbf{PITCH FOR \textcolor{blue}{BLUE LINE} - \vyse (85 KIAS),\\OR maintain altitude at a higher airspeed if able.}
    \item \textbf{\textcolor{red}{MIXTURES} SET (max available power)}
    \item \textbf{\textcolor{blue}{PROPELLERS} Full Forward}
    \item \textbf{THROTTLES Full Power}
\end{enumerate}

\textbf{CLEAN UP:}
\begin{enumerate}
    \item \textbf{FLAPS UP (or as required)}
    \item \textbf{GEAR UP (or as required)}
    \item \textbf{Trim and bank into good engine.}
\end{enumerate}

\textbf{IDENTIFY/VERIFY:}
\begin{enumerate}
    \item \textbf{IDENTIFY Dead foot = dead engine.}
    \item \textbf{VERIFY Power idle on dead engine. No change in performance - verified.}
\end{enumerate}

\textbf{FIX OR FEATHER:}
\begin{itemize}
    \item \textbf{DECISION Based upon situation/altitude (Restart or Feather?)}
\end{itemize}

Rarely do engines fail suddenly and completely (fuel starvation is the exception). If an engine is running poorly, but
developing some power, \textbf{you are better off letting it run} (above 3000 AGL) until you sort out the problem. The
decision to feather should be made with some deliberation. A catastrophic engine failure would require feathering
the engine (avoiding a potential fire), however, a rough running engine should not be feathered as any horsepower it
is producing potentially helps.

\textbf{The exception is during a critical phase of flight}, such as initial climb-out,
approach or landing. During these phases of flight the propeller on the problem engine should be feathered
immediately, as there is not enough time to safely perform the fix procedures. Maintaining aircraft control is
\emph{the priority} and you should land as soon and as safely as possible.

\textbf{FIX:}

\textbf{CHECK (\textcolor{red}{Red Items}) - only on affected engine.}
\begin{enumerate}
    \item \textbf{Fuel - ON}
    \item \textbf{Carburetor Heat - ON}
    \item \textbf{Mixture - SET (max available power)}
    \item \textbf{Boost Pumps - ON}
    \item \textbf{Magnetos - CYCLE Left - Right - Both}
\end{enumerate}

\textbf{\textcolor{red}{WARNING} Feathering the wrong engine is incredibly dangerous!
Work methodically to make certain the correct engine is feathered.}

\textbf{FEATHER:}
\begin{enumerate}
    \item \textbf{Feather Propeller on inoperative engine.}
    \item \textbf{Power as needed on good engine to maintain altitude/airspeed.\\Minimum speed: \textcolor{blue}{BLUE LINE 85 KIAS}.}
\end{enumerate}

\emph{Memory aid: 3\degree{} bank for slip.}

\textbf{SHUTDOWN AND SECURE ENGINE:}
\begin{itemize}
    \item \textbf{Mixture - Idle Cutoff}
    \item \textbf{Fuel Selector - OFF}
    \item \textbf{Cowl Flaps - Open on operative engine, closed on inoperative engine.}
    \item \textbf{Fuel Pump - OFF}
    \item \textbf{Magnetos - OFF}
    \item \textbf{Alternator Switch - OFF}
    \item \textbf{Notify ATC.}
    \item \textbf{Land as soon as practical.}
\end{itemize}

\textbf{\textcolor{red}{** NOW REFER TO CHECKLIST. **}}



